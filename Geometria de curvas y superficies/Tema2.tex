\section{Superficies}
\subsection{Superficies regulares}
\subsubsection{Noción de superficie regular}

\begin{defn}
	Sea $S \subset \R^3$ un conjunto de puntos, es una superficie regular $\iff \forall p \in S$
	\[\exists U \subset \R^2 \, \we \, \exists W \subset R^3 \tex{ abiertos} \, \we \, \exists \appl{\mathbb{X}}{U}{\R^3} : \mathbb{X}(U) = W \cap S \tex{ de manera que}\]
	\begin{enumerate}
		\item (diferenciabilidad) $\mathbb{X}$ es $C^\infty$.
		\item (regularidad) $\ds \forall (u, v) \in U : \left(\odv{\mathbb{X}}{u} \times \odv{\mathbb{X}}{v}\right)(u, v) \neq 0$. Es decir, $\odif{\mathbb{X}}$ es de rango 2.
		\item (homeomorfismo) $\mathbb{X}$ es inyectiva y $\mathbb{X}^{-1}$ es continua.
	\end{enumerate}
\end{defn}

A tales $\mathbb{X}$ se les llama cartas, parametrizaciones, mapas, o sistema (local) de coordenadas.

Si $\mathbb{X}$ es carta de $S$ y $\mathbb{X}(U) = S$ (es decir, la carta $\mathbb{X}$ cubre todo $S$), se dice que $\mathbb{X}$ es carta global o parametrización global de $S$.

\allbold{Coordenadas y curvas coordenadas}

Si $q \in X(U)$, entonces $q$ se escribe (de manera única) $q = \mathbb{X}(u_0, v_0)$ con $(u_0, v_0) \in U$. \\
A $(u_0, v_0)$ se les llama las $\mathbb{X}$-coordenadas de $q$.

Las curvas que describe $\mathbb{X}$ cuando $v$ se deja constante y cuando $u$ se deja constante son las curvas coordenadas: son $u$-curvas y $v$-curvas, respectivamente y denotamos:
\begin{itemize}
	\item $\X_u(u_0, v_0)$ por el vector velocidad de la $u$-curva en $\mathbb{X}(u_0, v_0)$.
	\item $\X_v(u_0, v_0)$ por el vector velocidad de la $v$-curva en $\mathbb{X}(u_0, v_0)$.
\end{itemize}

\allbold{Sobre las condiciones de superficie regular}

\begin{enumerate}
	\item Sobre la condición de diferenciabilidad, podemos escribir $\X(u, v) = (x(u, v), y(u, v), z(u, v))$ y entonces las funciones coordenadas de $\X$, es decir, $x, y, z$ son $C^\infty$.
	\item Sobre la condición de regularidad, escribiendo $\X_u = (x_u, y_u, z_u)$ y $\X_v = (x_v, y_v, z_v)$, son equivalentes:
	      \begin{itemize}
		      \item $\forall (u, v) \in U : $ la matriz $\odif{\X}(u, v)$ es inyectiva.
		      \item $\forall (u, v) \in U : \X_u(u, v)$ y $\X_v(u, v)$ son linealmente independientes.
		      \item $\forall (u, v) \in U : \X_u(u, v) \times \X_v(u, v) \neq 0$.
		      \item $\ds \forall (u, v) \in U : \left( \abs{\pdv{{(x, y)}}{{(u,v)}}(u,v)} \neq 0 \ve \abs{\pdv{{(x, z)}}{{(u,v)}}(u,v)} \neq 0 \ve \abs{\pdv{{(y, z)}}{{(u,v)}}(u,v)} \neq 0 \right)$.
	      \end{itemize}
	\item Sobre la condición de homeomorfismo, la inyectividad de $\X$ permite ubicar sin ambigüedad los puntos de $\X(U) = W\cap S$ cercanos a $p$ con parámetros $(u, v)$.

	      Que $\X^{-1}$ sea continua es una condición laboriosa de verificar, pero si ya sabemos que $S$ es superficie regular y tenemos un candidato a carta $\X$ que cumple las condiciones anteriores, entonces $\X^{-1}$ es continua automáticamente.
\end{enumerate}

Por tanto, para comprobar que un conjunto $S$ es una superficie regular, hay que encontrar una carta $\X$ que cumpla las condiciones de superficie regular.
\begin{enumerate}
	\item Encontrar una aplicación $\X$ de tipo $C^\infty$ cuya imagen de un abierto $U \subset \R^2$ sea (una parte de) $S$.
	\item Comprobar que $\forall (u, v) \in U : \X_u(u, v)$ y $\X_v(u, v)$ son linealmente independientes.
	\item Comprobar que $\X$ es inyectiva.
	\item Comprobar que $\X(u_n, v_n) \xrightarrow{n \to \infty} \X(u,v) \implies (u_n, v_n) \xrightarrow{n \to \infty} (u, v)$.
\end{enumerate}

Usando el teorema de la función implícita, podemos demostrar que el conjunto de puntos que satisfacen una ecuación de la forma $F(x, y, z) = a$ (con $F$ $C^\infty$) es una superficie regular si $\nabla F \neq 0$ en los puntos de la superficie.
\[\forall (x,y,z) \in F^{-1}(a) : F_z(x,y,z) \neq 0 \implies \exists f \tex{ de tipo } C^\infty : F(x, y, f(x, y)) = a\]
Entonces definimos $\X(u, v) = (u, v, f(u, v))$ que cumple las condiciones de carta.

\subsection{Superficies parametrizadas}

La noción de superficie parametrizada es más general que la que hemos
manejado de superficie regular. Es extensión directa de la noción de curva regular.Deberíamos quizás adjetivar superficie parametrizada regular, pero no.

\begin{defn}
	Sea $\appl{X}{U}{\R^3}$ con $U \subset \R^2$ abierto, $\left(\X, U\right)$ es una superficie parametrizada
	\[\iff \big(\forall (u, v) \in U : \left(\X_u \times \X_v\right)(u, v) \neq 0\big) \we \big(\X \tex{ es de tipo } C^\infty\big)\]
\end{defn}

\begin{obs}
	\begin{itemize}
		\item La traza de la superficie es $\X(U)$.
		\item Solo hay una carta $\X$ para la superficie y no hay condiciones de homeomorfismo, solo de diferenciabilidad y regularidad.
		\item $\X$ puede no ser inyectiva y $\X^{-1}$ puede no ser continua. De esta manera, incluimos superficies con auto-intersecciones y con ``bordes''.
	\end{itemize}
\end{obs}

\subsubsection{Superficies generadas por curvas}

\begin{enumerate}
	\item \allbold{Cilindros:} Sea $\forall t \in I : \alpha(t) = \left(a(t), b(t)\right)$ una curva en $\R^2$.

	      Definimos un cilindro parametrizado mediante $\forall t \in I, u \in \R : \X(t, u) = \left(a(t), b(t), u\right)$.

	      La aplicación es de tipo $C^\infty$ y se tiene que $\X_t = \left(a', b', 0\right)$ y $\X_u = \left(0, 0, 1\right)$, que son linealmente independientes. Por tanto, $(\X, I \times \R)$ es una superficie parametrizada.
	\item \allbold{Superficie de tangentes:} Sea $\gamma(s)$ una curva birregular p.p.a. definida en $s\in I$.

	      Definimos la superficie de tangentes mediante $\forall s \in I, u>0 : \X(s, u) = \gamma(s) + u \gamma'(s)$.

	      Observamos que $\forall s \in \R : \alpha(t) \defeq \X(s, t)$ es una (semir)recta tangente a $\gamma$ en $\gamma(s)$.

	      La aplicación es de tipo $C^\infty$ y se tiene que $\X_s = \gamma'(s) + u\gamma''(s)$ y $\X_u = \gamma'(s)$.
	      \[\implies \norm{(X_s \times \X_u)(u,v)} = u\norm{\gamma''(s) \times \gamma'(s)} = u \kappa(s) \neq 0 \tex{ por ser $\gamma$ birregular}\]
	      Por tanto, $(\X, I \times \R_{>0})$ es una superficie parametrizada.
	\item \allbold{Superficie de revolución:} Sea $\alpha(t) = \left(a(t), b(t)\right)$ con $a(t) > 0$ una curva regular en $\R^2$ definida para $t \in I$.

	      Definimos la superficie de revolución mediante $\X(\theta, t) = \left(a(t) \cos \theta, a(t) \sin \theta, b(t)\right)$ en $(\theta, t) \in [-\pi, \pi) \times I$.

	      La aplicación es de tipo $C^\infty$ y se tiene que
	      \[\X_\theta(\theta, t) = \left(-a(t) \sin \theta, a(t) \cos \theta, 0\right) \, \we \, \X_t(\theta, t) = \left(a'(t) \cos \theta, a'(t) \sin \theta, b'(t)\right)\]
	      \[\implies \norm{\X_\theta \times \X_t} = \norm{\left(-a(t) b'(t) \cos \theta, b(t)a'(t)\sin\theta, a(t)a'(t)\right)} = a(t) \norm{\alpha'(t)} \neq 0\]
	      Por tanto, como $\alpha$ es una curva regular, $(\X, [-\pi, \pi) \times I)$ es una superficie parametrizada.
\end{enumerate}

\subsection{Complementos}

\subsubsection{Las superficies regulares son localmente gráficas}

Sea $S$ una superficie regular y sea $p$ un punto de $S$.

Entonces, existe un abierto $W$ de $p$ en $\R^3$ de manera que $W\cap S$ es la gráfica de una función de tipo $C^\infty$ de una de las tres formas siguientes: $z = f(x, y)$, $y = f(x, z)$ o $x = f(y, z)$.

Esto se  debe, nuevamente, al teorema de la función implícita.

\subsubsection{Los cambios de carta son difeomorfismos}

Sean $\appl{\X}{U}{\R^3}$ y $\appl{\mathbb{Y}}{V}{\R^3}$ cartas en un punto $p$ de una superficie regular $S$.

Denotamos $A = \X(U) \cap \mathbb{Y}(V)$ y sean $\bar{U} = \X^{-1}(A)$ y $\bar{V} = \mathbb{Y}^{-1}(A)$.

Los puntos de $A$ se pueden describir mediante $\X$-coordenadas en $\bar{U}$ o $\mathbb{Y}$-coordenadas en $\bar{V}$.

La función $\appl{\Phi}{\bar{U}}{\bar{V}}$ definida por $\Phi = \mathbb{Y} \circ \X^{-1}$ es el difeomorfismo \allbold{cambio de carta}.
\begin{itemize}
	\item $\X$ y $\mathbb{Y}$ son funciones inyectivas $\implies$ son biyectivas en sus imágenes $\implies \Phi$ es biyectiva.
	\item $\X$ y $\mathbb{Y}$ de tipo $C^\infty \implies \Phi$ y $\Phi^{-1}$ son de tipo $C^\infty$.
\end{itemize}

\subsubsection{Sobre la condición de $\mathbb{X}^{-1}$ continua}

Sea $S$ una superficie regular y sea $\appl{\X}{U}{\R^3}$ una aplicación tal que $\X(U) = W \cap S$. Suponemos que $\X$
\begin{itemize}
	\item Cumple la condición de diferenciabilidad (es de tipo $C^\infty$).
	\item Cumple la condición de regularidad (el rango de $\odif{\X}$ es 2).
	\item Es inyectiva.
\end{itemize}
Se concluye que $\X^{-1}$ es continua, y por tanto, $\X$ es una carta de $S$.

\subsubsection{Superficies parametrizadas y superficies regulares}

Sea $\appl{\X}{U}{\R^3}$ una superficie parametrizada y sea $(u_0, v_0) \in U$. Entonces se puede tomar $V$ entorno de $(u_0, v_0)$ tal que $(u_0, v_0) \in V \subset U$ de manera que $S=\X(V)$ es una superficie regular.

Es más, $\appl{\X}{V}{\R^3}$ es parametrización global de $S$.

\subsection{Vectores y plano tangente}

\subsubsection{Vector tangente}

\begin{defn}
	Sea $S$ una superficie regular y sea $p \in S$. Un vector $v \in \R^3$ es tangente a $S$ en $p \iff \Big(\exists \appl{\alpha}{I \ni 0}{S}$ curva $: \alpha(0) = p \we \alpha(I)\subset S \Big) \we v = \alpha'(0)$.

	Es decir, el que $v$ sea un vector tangente a la superficie regular $S$ en el punto $p$ significa que $v$ es vector tangente (velocidad) a una curva por $p$ con traza en $S$.
\end{defn}

Sea $\appl{\X}{U}{\R^3}$ una carta de una superficie regular $S$ y sea $p = \X(u_0, v_0)$.

Sea $v$ un vector tangente a $S$ en $p$ y sea $\alpha(t) = \X(u(t), v(t))$ la curva con traza contenida en $S$ y tal que $\alpha(0) = p$ y $\alpha'(0) = v$.
\[\implies v= \alpha'(0) = u'(0)\cdot\X_u(u_0, v_0) + v'(0)\cdot\X_v(u_0, v_0) \implies v \in \mathcal{L}\{\X_u(u_0, v_0), \X_v(u_0, v_0)\}\]
Recíprocamente, dada una combinación lineal de $\{\X_u(u_0, v_0), \X_v(u_0, v_0)\}$, consideramos la curva $\alpha(t) = \X(u_0 + ta, v_0 + tb)$, entonces:
\[\alpha'(t) = a\X_u(u_0 + ta, v_0 + tb) + b\X_v(u_0 + ta, v_0 + tb) \implies \alpha'(0) = a\X_u(u_0, v_0) + b\X_v(u_0, v_0) = v\]

Es decir, $v$ es vector tangente a $S$ en $p \iff v \in \mathcal{L}\{\X_u(u_0, v_0), \X_v(u_0, v_0)\}$.

\subsubsection{Plano tangente}

\begin{defn}
	Sea $S$ una superficie regular y sea $\X$ una carta con $p = \X(u_0, v_0)$, \\$T_pS$ es el plano tangente a $S$ en $\ds p \iff T_pS = p + \mathcal{L}\{\X_u(u_0, v_0), \X_v(u_0, v_0)\}$.
\end{defn}

\allbold{Vector normal a $T_pS$}

El vector $\X_u(u_0, v_0) \times \X_v(u_0, v_0)$ es no nulo y perpendicular a $T_pS$, así que $T_pS$ es el plano que pasa por $p$ y tiene como vector normal a $\X_u(u_0, v_0) \times \X_v(u_0, v_0)$.

\begin{defn}[Vector normal unitario]
	Sea $S$ una superficie regular y sea $p \in S$.

	Definimos $\ds N\defeq \frac{\X_u(u_0, v_0) \times \X_v(u_0, v_0)}{\norm{\X_u(u_0, v_0) \times \X_v(u_0, v_0)}}$ como el vector normal unitario a $S$ en $p$.
\end{defn}
