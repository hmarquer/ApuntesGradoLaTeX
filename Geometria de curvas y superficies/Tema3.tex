\section{Primera forma fundamental}

\subsection{Primera forma cuadrática fundamental}

\begin{defn}
	Sea $S$ una superficie regular y $p\in S$, la aplicación $I_p$ es la forma cuadrática (cuadrática fundamental) en el plano tangente $T_pS \iff \forall v \in T_pS : I_p(v) = \langle v,v \rangle = \norm{v}^2$.
\end{defn}

\begin{obs}
	$\forall p \in S : I_p$ es una forma cuadrática definida positiva.

	$I_p$ es la forma cuadrática asociada a la forma bilineal simétrica $(u,v) \in T_pS \times T_pS \mapsto \langle u,v \rangle$.
\end{obs}

Sea $\appl{\X}{U\subset \R^2}{\R^3}$ carta de $S$ en $p=\X(u_0, v_0)$. Vamos a expresar la primera forma fundamental $I_p$ de $S$ en $p$ en coordenadas con respecto a la base natural $\beta = \{\X_u(u_0, v_0), \X_v(u_0, v_0)\}$ de $T_pS$ cuyos vectores son las velocidades de las curvas coordenadas en $p$.

Para todo $(u, v) \in U$ definimos:
\[\begin{aligned}
		E(u, v) & \defeq \langle \X_u(u, v), \X_u(u, v) \rangle = \norm{\X_u(u, v)}^2                    \\
		F(u, v) & \defeq \langle \X_u(u, v), \X_v(u, v) \rangle = \langle \X_v(u, v), \X_u(u, v) \rangle \\
		G(u, v) & \defeq \langle \X_v(u, v), \X_v(u, v) \rangle = \norm{\X_v(u, v)}^2
	\end{aligned}\]

Las funciones $E, F, G$ son $C^\infty$ en $U$.
\begin{itemize}
	\item $E(u_0, v_0)$ es el cuadrado de la rapidez de la curva coordenada $\X(u, v_0)$ en $\X(u_0, v_0)$, cuyo vector velocidad es $\X_u(u_0, v_0)$.
	\item $G(u_0, v_0)$ es el cuadrado de la rapidez de la curva coordenada $\X(u_0, v)$ en $\X(u_0, v_0)$, cuyo vector velocidad es $\X_v(u_0, v_0)$.
	\item Si se tiene $F\equiv 0$, entonces las curvas coordenadas se cortan perpendicularmente en cada intersección.
\end{itemize}
Se tiene que $\forall v \in T_pS : \exists a, b \in \R : v = a\X_u(u_0, v_0) + b\X_v(u_0, v_0)$, y entonces:
\[
	\begin{aligned}
		I_p(v) & = \langle v, v \rangle = \langle a\X_u(u_0, v_0) + b\X_v(u_0, v_0), a\X_u(u_0, v_0) + b\X_v(u_0, v_0) \rangle                                               \\
		       & = a^2\langle \X_u(u_0, v_0), \X_u(u_0, v_0) \rangle + 2ab\langle \X_u(u_0, v_0), \X_v(u_0, v_0) \rangle + b^2\langle \X_v(u_0, v_0), \X_v(u_0, v_0) \rangle \\
		       & = a^2E(u_0, v_0) + 2abF(u_0, v_0) + b^2G(u_0, v_0)
	\end{aligned}\]
Por tanto, en forma matricial: $\ds I_p(v) = \begin{pmatrix}
		E(u_0, v_0) & F(u_0, v_0) \\
		F(u_0, v_0) & G(u_0, v_0)
	\end{pmatrix}\begin{pmatrix}a\\b\end{pmatrix} \tex{ con } v = (a, b)_{\beta}$.

Esta matriz simétrica es la matriz de la forma cuadrática $I_p$ respecto de la base $\beta$.
\[I_p \tex{ def. positiva } \implies \det{\begin{pmatrix}
			E(u_0, v_0) & F(u_0, v_0) \\
			F(u_0, v_0) & G(u_0, v_0)
		\end{pmatrix}} = E(u_0, v_0)G(u_0, v_0) - F^2(u_0, v_0) > 0\]

\subsection{Longitudes, ángulos, áreas}

\subsubsection{Longitudes}

Sea $\appl{\alpha}{I}{\R^3}$ una curva y $a, b \in I, a < b$, se tiene que la longitud de la traza de $\alpha$ entre $\alpha(a)$ y $\alpha(b)$ es longitud $\ds = L(\alpha) = \int_{a}^{b} \norm{\alpha'(t)} \odif{t}$.

Si la traza de $\alpha$ está contenida en una superficie regular $S$, de hecho, en $\X(U)$, entonces:
\[\forall t \in I : \alpha(t) = \X(u(t), v(t)) \implies \alpha'(t) = u'(t)\X_u(u(t), v(t)) + v'(t)\X_v(u(t), v(t))\]
\[I_{\alpha(t)}(\alpha'(t)) = \norm{\alpha'(t)}^2 = \left[u'(t)\right]^2E(u(t), v(t)) + 2u'(t)v'(t)F(u(t), v(t)) + \left[v'(t)\right]^2G(u(t), v(t))\]
Por tanto, la longitud de $\alpha$ se escribe, obviando la variable $t$:
\[L(\alpha) = \int_{a}^{b} \sqrt{(u')^2E(u,v) + 2u'v'F(u,v) + (v')^2G(u,v)} \odif{t}\]
Con cierto abuso de notación se escribe $\odif{s}^2 = E \odif{u}^2 + 2F\odif{u}\odif{v} + G\odif{v}^2$ y se le llama primera forma fundamental.

\begin{defn}[Primera forma fundamental]
	Sea $\appl{\alpha}{I}{S}$ una curva regular en una superficie regular $S$ y sea $\appl{\X}{U\subset \R^2}{\R^3}$ una carta de $S$ tal que $\alpha(I) \subset \X(U)$ y $\alpha(t) = \X(u(t), v(t))$.

	La primera forma fundamental de $S$ en $\alpha(t)$ es la forma cuadrática $I_{\alpha(t)}$ evaluada en $\alpha'(t)$ y se denota por $\odif{s}^2 = E\odif{u}^2 + 2F\odif{u}\odif{v} + G\odif{v}^2$.
\end{defn}

\subsubsection{Ángulos}

Sean $\alpha_1$ y $\alpha_2$ dos curvas regulares en una superficie regular $S$ tales que $\alpha_1(t_1) = p = \alpha_2(t_2)$.

Se tiene que el ángulo $\omega$ entre $\alpha_1$ y $\alpha_2$ en $p$ cumple $\ds \cos\omega = \frac{\langle \alpha_1'(t_1), \alpha_2'(t_2) \rangle}{\norm{\alpha_1'(t_1)}\norm{\alpha_2'(t_2)}}$.

Sea $\X(u,v)$ una carta de $S$ en $p$, tenemos $\begin{cases}
		\alpha_1(t) = \X(u_1(t), v_1(t)) \\
		\alpha_2(t) = \X(u_2(t), v_2(t))
	\end{cases} \hspace{-0.5cm}$.
\[\implies \begin{cases}
		\alpha_1'(t) = u_1'(t)\X_u(u_1(t), v_1(t)) + v_1'(t)\X_v(u_1(t), v_1(t)) \\
		\alpha_2'(t) = u_2'(t)\X_u(u_1(t), v_1(t)) + v_2'(t)\X_v(u_1(t), v_1(t))
	\end{cases}\hspace{-0.5cm}\]
\[(u_1(t_1), v_1(t_1)) = (u_0, v_0) = (u_2(t_2), v_2(t_2)) \implies \begin{cases}
		\alpha_1'(t_1) = u_1'(t_1)\X_u(u_0, v_0) + v_1'(t_1)\X_v(u_0, v_0) \\
		\alpha_2'(t_2) = u_2'(t_2)\X_u(u_0, v_0) + v_2'(t_2)\X_v(u_0, v_0)
	\end{cases}\]
\[\implies \langle \alpha_1'(t_1), \alpha_2'(t_2) \rangle = \begin{aligned}
		  & u_1'(t_1)u_2'(t_2)E(u_0, v_0) + (u_1'(t_1)v_2'(t_2) + v_1'(t_1)u_2'(t_2))F(u_0, v_0) \\
		+ & v_1'(t_1)v_2'(t_2)G(u_0, v_0)
	\end{aligned}\]
Por tanto, obviando las variables $t_1, t_2$:
\[\cos\omega = \frac{Eu_1'u_2' + F(u_1'v_2' + v_1'u_2') + Gv_1'v_2'}{\sqrt{(u_1')^2E + 2u_1'v_1'F + (v_1')^2G}\sqrt{(u_2')^2E + 2u_2'v_2'F + (v_2')^2G}}\]
Las funciones $E$, $F$ y $G$ se evalúan en $(u_0, v_0) = (u_1(t_1), v_1(t_1)) = (u_2(t_2), v_2(t_2))$. \\Las derivadas $u_1'$ y $v_1'$ se evalúan en $t_1$, mientras que las derivadas $u_2'$ y $v_2'$ se evalúan en $t_2$.

\subsubsection{Áreas}

Sea $S$ una superficie regular con $\appl{X}{U\subset\R^2}{V\subset S}$ carta de $S$.
\[\implies \tex{Área de }\X(U) = \iint_U \norm{\X_u(u, v)\times \X_v(u, v)} \odif{u} \odif{v}\]
Como se tiene que $\forall v, w \in \R^3 :\norm{v \times w}^2=\norm{v}^2 \norm{w}^2 - {\langle v, w \rangle}^2$, se deduce que
\[\norm{\X_u \times \X_v} = \sqrt{EG -F^2} \implies \tex{Área}(\X(U)) = \iint_U \sqrt{E(u,v)G(u,v) - F^2(u,v)} \odif{u} \odif{v}\]

\subsection{Loxodromas}

\begin{defn}[Loxodroma]
	Sea $\appl{\alpha}{I\subset\R^2}{\R^3}$ una curva contenida en una esfera, $\alpha$ es un loxodroma $\iff$ corta a los meridianos de la esfera con ángulo $\beta$ fijo.
\end{defn}

Si consideramos la parametrización de la esfera $S$ de radio $R$ en coordenadas esféricas dada por $\X(\theta, \varphi) = (R\cos\theta \sin\varphi, R\sin\theta \sin\varphi, R\cos\varphi)$ con $\theta \in (0, 2\pi)$ y $\varphi \in (0, \pi)$, tenemos que la primera forma fundamental viene dada por: $E(\theta, \varphi)=R^2\sin^2\phi$, $F(\theta, \varphi)=0$ y $G(\theta, \varphi)=R^2$.

Fijamos un $\beta \in [0, \sfrac{\pi}{2})$ y un punto $p = \X(\theta_0, \varphi_0)$ en la esfera. Sea $\gamma(t) = \X(\theta_0, t)$ el meridiano que pasa por $p$ y $\omega = \X(\theta(t), \varphi(t))$ una loxodroma que corta a $\gamma$ con ángulo $\beta$ y que $\omega(t_0) = p$.

Dado que $\odv{}{t} \theta_0 = 0 \we \odv{}{t} t = 1 \we \odv{}{t} \theta(t) = \theta'(t) \we \odv{}{t} \varphi(t) = \varphi'(t)$, se tiene que:
\[\implies \langle \omega'(t), \gamma'(t) \rangle =	R^2\sin^2\varphi \cdot 0 \cdot \theta'(t) + 0 \cdot (0\cdot \varphi'(t) + 1\cdot \theta'(t)) + R^2\cdot 1\cdot \varphi'(t) = R^2\varphi'(t)\]
\[\begin{aligned}
		\norm{\varphi'(t)} & = \sqrt{0^2\cdot R^2\sin^2\varphi_0 + 2 \cdot 0 \cdot 1 \cdot 0 + 1^2\cdot R^2} = R                                                                          \\
		\norm{\omega'(t)}  & = \sqrt{(\theta'(t))^2 R^2 \sin^2\varphi_0 + 2\theta'(t)\varphi'(t) \cdot 0 + (\varphi'(t))^2 R^2} = R\sqrt{(\theta'(t))^2\sin^2\varphi_0 + (\varphi'(t))^2}
	\end{aligned}\]
\[\implies \cos\beta = \frac{\cancel{R^2}\varphi'(t)}{\cancel{R^2}\sqrt{(\theta'(t))^2\sin^2\varphi_0 + (\varphi'(t))^2}} = \frac{\varphi'(t)}{\sqrt{(\theta'(t))^2\sin^2\varphi_0 + (\varphi'(t))^2}}\]

Pero queremos que $\omega$ corte a todos los meridianos con el mismo ángulo $\beta$. Observando que, en la fórmula anterior $\varphi_0 = \varphi(t_0)$, concluimos que las funciones $\theta(t)$ y $\varphi(t)$ que determinan la curva $\omega$ deben cumplir la siguiente ecuación direncial:
\[\forall t : \boxed{\cos\beta = \frac{\varphi'(t)}{\sqrt{(\theta'(t))^2\sin^2(\varphi(t)) + (\varphi'(t))^2}}}\]
\[\implies \frac{(\varphi'(t))^2}{\cos^2\beta} = \sin^2(\varphi(t))(\theta'(t))^2 + (\varphi'(t))^2 \implies \varphi'(t)\left(\frac{1}{\cos^2\beta} - 1\right) = \sin^2(\varphi(t))(\theta'(t))^2\]
\[\implies (\varphi'(t))^2\tan^2\beta = \sin^2 (\varphi(t))(\theta'(t))^2 \implies \frac{(\varphi'(t))^2}{\sin^2(\varphi(t))} = \frac{(\theta'(t))^2}{\tan^2\beta}\implies \frac{\varphi'(t)}{\sin(\varphi(t))} = \pm\frac{\theta'(t)}{\tan\beta}\]
\[\implies \boxed{\tan\beta \cdot \ln\left(\tan\left(\frac{\varphi(t)}{2}\right)\right) = \pm \theta(t) + C}\]

\subsection{Isometrías}

\begin{defn}
	Sea $S$ una superficie regular, $S_*$ es una trozo de $S \iff S_* \subset S$ y $S_*$ es una superficie regular.
\end{defn}

\allbold{Nota:} La intersección de un abierto de $\R^3$ con una superficie regular es un trozo de la superficie.

\begin{defn}
	Sean $S$ y $S_*$ dos superficies regulares, $\appl{f}{S}{S_*}$ es una isometría
	\[\iff\begin{cases}
			f \tex{ es un difeomorfismo, i.e. $f$ es suave, biyectiva y con inversa suave} \\
			f \tex{ preserva longitudes }\forall C \subset S \tex{ curva regular} : L(f(C)) = L(C)
		\end{cases}\]
\end{defn}

\begin{teo}
	No existe una isometría entre un trozo de una esfera y un trozo de un plano.
	\begin{dem}
		Asumimos que una isometría preserva también ángulos y áreas. Por contradicción, asumimos que existe una isometría $f$ entre un trozo de una esfera y un trozo de un plano.

		Tomamos un triángulo $T$ de vértices $A, B, C$ dentro del trozo de un plano y su imagen $T'$ de vértices $A', B', C'$ en el trozo de la esfera. Los segmentos que conforman los lados del triángulo son las curvas de longitud mínima entre los vértices. En el caso del plano, rectas, y en el caso de la esfera, circunferencias máximas.

		Esto quiere decir que los lados del triángulo deben transformarse en trozos de circunferencias máximas en la esfera. Como se conservan los ángulos tenemos que
		\[\pi = \hat{A}+\hat{B}+\hat{C} = \hat{A'}+\hat{B'}+\hat{C'} = \pi + \frac{\tex{Área}(T')}{R^2} > \pi \contr\]
		Aquí se ha usado el teorema Girard (la suma de los ángulos de un triángulo en la esfera unidad es $\pi + \tex{Área del triángulo}$).
	\end{dem}
\end{teo}

\begin{lem}
	Sean $S$ y $S_*$ dos superficies regulares y $\appl{f}{S}{S_*}$ una aplicación de tipo $C^\infty$.
	\[f\tex{ preserva longitudes} \iff f \tex{ preserva la primera forma fundamental }\]
	\begin{dem}
		($\Longleftarrow$) Sea $\appl{\X}{U}{S}$ una carta de $S$ tal que $\X(U) = S$ y sea $\appl{\alpha}{I}{S}$ una curva regular en $S$ tal que $\forall t \in (a, b) : \alpha(t) = \X(u(t), v(t))$.

		Como se conserva la primera forma fundamental, se tiene que
		\[\begin{aligned}
				I_{\alpha(t)}(\alpha'(t)) & = \left[u'(t)\right]^2E(u(t), v(t)) + 2u'(t)v'(t)F(u(t), v(t)) + \left[v'(t)\right]^2G(u(t), v(t))       \\
				                          & = \left[u'(t)\right]^2E^*(u(t), v(t)) + 2u'(t)v'(t)F^*(u(t), v(t)) + \left[v'(t)\right]^2G^*(u(t), v(t)) \\
				                          & = I^*_{(f\circ \alpha)(t)}((f\circ\alpha)'(t))
			\end{aligned}\]
		Donde $E^*, F^*, G^*$ componen la primera forma fundamental de $S_*$ con la carta $f\circ\X$.

		Dado que $\norm{\alpha'(t)}^2= I_{\alpha(t)}(\alpha(t))$, se tiene que
		\[L(\alpha) = \int_{a}^{b} \norm{\alpha'(t)} \odif{t} = \int_{a}^{b} \sqrt{I_{\alpha(t)}(\alpha'(t))} \odif{t} = \int_{a}^{b} \sqrt{I^*_{(f\circ \alpha)(t)}((f\circ\alpha)'(t))} \odif{t} = L(f\circ\alpha)\]
		Es decir, $f$ preserva longitudes.

		($\Longrightarrow$) Sea $\appl{\X}{U}{S}$ una carta de $S$ y $\appl{\alpha}{I}{S}$ una curva regular en $S$ tal que $\forall t \in (a, b) : \alpha(t) = \X(u(t), v(t))$. Supongamos que $f$ preserva longitudes y, por contradicción, que alguno de los coeficientes de la primera forma fundamental es distinto en $I_{\alpha(t)}(\alpha'(t))$ y $I^*_{(f\circ \alpha)(t)}((f\circ\alpha)'(t))$.

		Asumimos primero que $E(u_0, v_0) < E^*(u_0, v_0)$. Por continuidad, existe una bola $B_\varepsilon \subset U$ con $(u_0, v_0) \in B_\varepsilon$ tal que $\forall (u, v) \in B_\varepsilon : E(u, v) < E^*(u, v)$. Tomamos
		\[C\defeq \{\alpha(t) \defeq \X(u_0+t, v_0) : -\varepsilon < t < \varepsilon\} \subset S\]
		\[\implies \norm{\alpha'(t)}^2 = \norm{\pdv{\X}{u}(u_0t, v_0)}^2=E(u_0+t, v_0) < E^*(u_0+t, v_0) = \norm{(f\circ\alpha)'(t)}^2\]
		Por tanto, al considerar longitudes, se tiene que $L(\alpha) < L(f\circ\alpha)$.
		\[L(C) = \int_{-\varepsilon}^{\varepsilon} \norm{\alpha'(t)} \odif{t} < \int_{-\varepsilon}^{\varepsilon} \norm{(f\circ\alpha)'(t)} \odif{t} = L(f\circ C) \contr\]

		Si $G(u_0, v_0) \neq G^*(u_0, v_0)$, se toma $C\defeq \{\alpha(t) \defeq \X(u_0, v_0+t) : -\varepsilon < t < \varepsilon\}$ y se llega a una contradicción análoga.

		Finalmente, si $E(u_0, v_0) = E^*(u_0, v_0)$ y $G(u_0, v_0) = G^*(u_0, v_0)$ pero $F(u_0, v_0) < F^*(u_0, v_0)$, se toma $C\defeq \{\alpha(t) \defeq \X(u_0+t, v_0+t) : -\varepsilon < t < \varepsilon\}$.
		\[\alpha' = \pdv{\X}{u} + \pdv{\X}{v} = \X_u + \X_v \implies \norm{\alpha'}^2 = E + 2F + G < E^* + 2F^* + G^* = \norm{(f\circ\alpha)'}^2\]
		Por tanto y de manera análoga, $L(\alpha) < L(f\circ\alpha) \implies L(C) < L(f\circ C) \contr$.
	\end{dem}
\end{lem}

\begin{cor}
	Sea $S$ una superficie regular y $\appl{f}{S}{S^*}$ una isometría.
	\begin{itemize}
		\item $f$ preserva longitudes en $S$ independientemente de la carta que se tome.
		\item $f$ preserva ángulos y áreas.
	\end{itemize}
\end{cor}

\begin{lem}
	Sea $S$ una superficie regular y $\appl{f}{S}{S^*}$ una aplicación de tipo $C^\infty$ que preserva la primera forma fundamental.
	\[\implies \forall p \in S : \exists S_1 \ni p \tex{ trozo de }S : \left. f\right|_{S_1} \tex{es una isometría}\]
	\begin{dem}
		Se tiene el siguiente diagrama:
		\[\begin{tikzcd}
				p \in \hspace{-2.3cm} & S \arrow[r, "f"] & S^* & \hspace{-2.3cm} \ni f(p) \\
				\X^{-1}(p) \in \hspace{-1.25cm} & U \arrow[u, "\X"] \arrow[r, "g"] & V  \arrow[u, "\mathbb{Y}"'] & \hspace{-1.3cm} \ni \mathbb{Y}^{-1}(f(p))
			\end{tikzcd}\]
		\[\tex{Se tiene que }\norm{\pdv{(f\circ \X)}{u} \times \pdv{(f\circ \X)}{v}}^2 = E^*G^* - F^* = EG - F = \norm{\X_u \times \X_v}^2 > 0\]
		Por tanto, la matriz diferencial de $f\circ \X$ en $\X^{-1}(p)$ y la matriz diferencial de $\mathbb{Y}$ en $\mathbb{Y}^{-1}(f(p))$ tienen rango 2.

		Entonces, la matriz diferencial de $g\defeq \mathbb{Y}^{-1}\circ f\circ \X$ en $\X^{-1}(p)$ tiene rango 2. Por el teorema de la función inversa, $g$ es un difeomorfismo local en un entorno de $\X^{-1}(p)$.

		Por tanto, $\exists S_1 \ni f(p) : \forall q \in S^* : {\left({\left. f \right|}_{S_1}\right)}^{-1}(q) = \big(\X \circ g^{-1}\circ \mathbb{Y}^{-1}\big)(q)$ que es de clase $C^\infty$. Entonces, $\left. f\right|_{S_1}$ es una isometría.
	\end{dem}
\end{lem}

\begin{defn}
	Sea $S$ una superficie regular, $\appl{f}{S}{S^*}$ es una isometría local
	\[\iff f\tex{ es de clase $C^\infty$ y preserva la primera forma fundamental}\]
\end{defn}

\begin{lem}
	Sean $f$ y $g$ dos isometrías, entonces $f\circ g$ es una isometría.
	\begin{dem}
		Trivialote.
	\end{dem}
\end{lem}

\begin{defn}Sean $S$ y $S^*$ dos superficies regulares.
	\begin{itemize}
		\item $S$ y $S^*$ son isométricas $\iff \exists \appl{f}{S}{S^*}$ isometría.
		\item $S$ y $S^*$ son localmente isométricas
		      \[\begin{aligned}
				      \iff & \forall p \in S : \exists S_1 \ni p \tex{ trozo de $S$ isométrico a algún trozo de $S^*$}   \\
				      \we  & \forall q \in S^* : \exists S_2 \ni q \tex{ trozo de $S^*$ isométrico a algún trozo de $S$}
			      \end{aligned}\]
	\end{itemize}
\end{defn}

\begin{obs}
	Tanto ser isométricas como ser localmente isométricas es una relación de equivalencia dentro del conjunto de superficies regulares.
\end{obs}

\subsubsection{Isometrías con formas}
¿Cómo saber si dos superficies regulares son (localmente) isométricas o no? Veamos un resultado general.
\begin{lem}
	Sean $S$ y $S^*$ dos superficies regulares y sean $\appl{\X}{U}{S}$ y $\appl{\mathbb{Y}}{A}{S^*}$ sus respectivas cartas.
	\[\exists \appl{f}{S}{S^*} \tex{ isometría} \iff \exists \appl{g}{U}{A} \tex{ difeomorfismo tal que}\]
	La primera forma fundamental de $\X$ $(E, F, G)$ es la misma que la de $\mathbb{Y}\circ g$ $(E_1, F_1, G_1)$.

	\begin{dem}
		Analicemos la primera forma fundamental de $\mathbb{Y}\circ g$ suponiendo que $\forall (u, v) \in U : g(u, v) = (a(u,v), b(u,v))$:
		\[\begin{aligned}
				E_1 & = \norm{\pdv{(\mathbb{Y}\circ g)}{u}}^2 = a_u^2 \norm{\mathbb{Y}_a(a, b)}^2 + 2a_ub_u\left\langle\mathbb{Y}_a(a, b), \mathbb{Y}_b(a, b) \right\rangle  + b_u^2 \norm{\mathbb{Y}_b(a, b)}^2                     \\
				    & = a_u^2 E^*(a, b) + 2a_ub_u F^*(a,b) + b_u^2 G^*(a,b)                                                                                                                                                          \\
				F_1 & = \left\langle \pdv{(\mathbb{Y}\circ g)}{u}, \pdv{(\mathbb{Y}\circ g)}{v} \right\rangle = \left\langle a_u\mathbb{Y}_a(a, b) + b_u\mathbb{Y}_b(a,b), a_v\mathbb{Y}_a(a, b) + b_v\mathbb{Y}_b(a,b)\right\rangle \\
				    & = a_ua_v\norm{\mathbb{Y}_a(a,b)}^2 + (a_ub_v+b_ua_v)\langle \mathbb{Y}_a(a,b), \mathbb{Y}_b(a,b)\rangle + b_ub_v\norm{\mathbb{Y_b}(a,b)}^2                                                                     \\
				    & = a_ua_vE^*(a, b) + (a_ub_v+b_ua_v)F^*(a,b) + b_ub_vG^*(a, b)                                                                                                                                                  \\
				G_1 & = \norm{\pdv{(\mathbb{Y}\circ g)}{v}}^2 = a_v^2 \norm{\mathbb{Y}_a(a, b)}^2 + 2a_vb_v\left\langle\mathbb{Y}_a(a, b), \mathbb{Y}_b(a, b) \right\rangle  + b_v^2 \norm{\mathbb{Y}_b(a, b)}^2                     \\
				    & = a_v^2 E^*(a, b) + 2a_vb_v F^*(a,b) + b_v^2 G^*(a,b)                                                                                                                                                          \\
			\end{aligned}\]
		Donde $(E^*, F^*, G^*)$ componen la primera forma fundamental de $\mathbb{Y}$.
		\[\implies \begin{cases}
				E & = E_1 \\
				F & = F_1 \\
				G & = G_1 \\
			\end{cases} \iff \begin{cases}
				E & = a_u^2 E^*(a, b) + 2a_ub_u F^*(a,b) + b_u^2 G^*(a,b)         \\
				F & = a_ua_vE^*(a, b) + (a_ub_v+b_ua_v)F^*(a,b) + b_ub_vG^*(a, b) \\
				G & = a_v^2 E^*(a, b) + 2a_vb_v F^*(a,b) + b_v^2 G^*(a,b)
			\end{cases}\]
		\[\tex{Se tiene el siguiente diagrama: }\begin{tikzcd}
				& S \arrow[r, "f"] & S^* \\
				(u, v) \in \hspace{-1.25cm} & U \arrow[u, "\X"] \arrow[r, "g"] & A  \arrow[u, "\mathbb{Y}"'] & \hspace{-1.3cm} \ni (a, b)
			\end{tikzcd}\]

		$(\Longrightarrow)$ Como $f$ es una isometría y $\X$ e $\mathbb{Y}$ son cartas, en particular son difeomorfismos.
		\\ Por tanto, $g = \mathbb{Y}^{-1}\circ f \circ \X$ es un difeomorfismo.

		Además, se tiene que $f\circ \X = \mathbb{Y} \circ g$ y como $f$ es una isometría, la primera forma fundamental de $S^*$ por la carta $f\circ \X$ es la misma que la de $S$ por la carta $\X$. Por tanto, la primera forma fundamental de $\mathbb{Y} \circ g$ es la misma que la de $\X$.

		$(\Longleftarrow)$ Suponemos que $g$ es difeomorfismo tal que se cumple el sistema de arriba. Entonces, tenemos que $f = \mathbb{Y} \circ g \circ \X^{-1}$ es un difeomorfismo por ser composición de ellos.

		Además, se vuelve a cumplir que $f\circ \X = \mathbb{Y}\circ g$ y como la primera forma fundamental de $S^*$ por la carta $\mathbb{Y}\circ g$ es la misma que la de $S$ por la carta $\X$, se concluye que $f$ preserva la primera forma fundamental. Por tanto, $f$ preserva también longitudes.

		Como $f$ es un difeomorfismo que preserva longitudes, $f$ es una isometría.
	\end{dem}
\end{lem}

Es decir, para demostrar que dos superficies regulares son isométricas, basta con encontrar un ``cambio de carta'' que mantenga la primera forma fundamental.

El sistema de ecuaciones del lema también puede escribirse en forma matricial:
\[\begin{pmatrix}
		E & F \\
		F & G
	\end{pmatrix} = \begin{pmatrix}
		a_u & a_v \\
		b_u & b_v
	\end{pmatrix}^T \begin{pmatrix}
		E^*(a,b) & F^*(a,b) \\
		F^*(a,b) & G^*(a,b)
	\end{pmatrix} \begin{pmatrix}
		a_u & a_v \\
		b_u & b_v
	\end{pmatrix} = (\tex{D}g)^T \begin{pmatrix}
		E^*(g) & F^*(g) \\
		F^*(g) & G^*(g)
	\end{pmatrix} \tex{D}g\]
Que proviene del cambio de carta:
\[\begin{pmatrix}
		\odif{a} \\
		\odif{b}
	\end{pmatrix} = \begin{pmatrix}
		a_u & a_v \\
		b_u & b_v
	\end{pmatrix} \begin{pmatrix}
		\odif{u} \\
		\odif{v}
	\end{pmatrix} \implies \begin{pmatrix}
		\odif{u} \\
		\odif{v}
	\end{pmatrix} = \begin{pmatrix}
		a_u & a_v \\
		b_u & b_v
	\end{pmatrix}^{-1} \begin{pmatrix}
		\odif{a} \\
		\odif{b}
	\end{pmatrix}\]

\begin{teo}[Janet 1926 - Cartan 1927]
	Sean $\appl{E, F, G}{U \subset \R^2}{\R}$ funciones de clase $C^\infty$ con $U$ abierto tales que $E > 0$ y $EG - F^2 > 0$ en $U$.

	$\implies \forall (u_0, v_0) \in U : \exists V \subset U : \begin{cases}
			(u_0, v_0) \in V \tex{ abierto de } \R^2                                         \\
			\exists \appl{\X}{V}{\R^3} \tex{ carta de }S=\X(V) \tex{ superficie regular con} \\
			\phantom{\exists \appl{\X}{V}{\R^3}\tex{ }} E = \norm{\X_u}^2 \we F = \langle \X_u, \X_v \rangle \we G = \norm{\X_v}^2
		\end{cases}$
\end{teo}

\subsection{Aplicación conformes}

Ya hemos visto que no podemos encontrar una isometría de un trozo de una esfera en un trozo de un plano. Sin embargo, ¿podemos encontrar una aplicación que haga que un dibujo en la esfera se vea de forma ``similar'' en el plano?

\begin{defn}
	Sea $\appl{f}{S}{S^*}$ un difeomorfismo, es conforme $\iff f$ preserva ángulos entre cualquier pareja de curvas regulares.
\end{defn}

\begin{lem}
	Sea $\appl{f}{S}{S^*}$ una aplicación de tipo $C^\infty$ entre dos superficies $S$ y $S^*$.
	\[f \tex{ es conforme} \iff \exists \appl{\lambda}{U}{\R} : \odif{s_*}^2 = \lambda^2 \odif{s}^2 \tex{ donde}\]
	$\odif{s}^2 = E\odif{u}^2 + 2F\odif{u}\odif{v} + G\odif{v}^2$ y $\odif{s_*}^2 = E^*\odif{u}^2 + 2F^*\odif{u}\odif{v} + G^*\odif{v}^2$ son las primeras formas fundamentales de $S$ y $S^*$ respectivamente.
	\begin{dem}
		$(\Longleftarrow)$ Trivial, ya que el coseno del ángulo entre dos curvas era:
		\[\cos\omega = \frac{Eu_1'u_2' + F(u_1'v_2' + v_1'u_2') + Gv_1'v_2'}{\sqrt{(u_1')^2E + 2u_1'v_1'F + (v_1')^2G}\sqrt{(u_2')^2E + 2u_2'v_2'F + (v_2')^2G}}\]
		con $(u_1', v_1')$ y $(u_2', v_2')$ los vectores tangentes a las curvas en $U$. Por tanto, si sustituimos $u_1' = \lambda u_1$ y $v_1' = \lambda v_1$ y $u_2' = \lambda u_2$ y $v_2' = \lambda v_2$, se cancela el factor $\lambda$ y el coseno del ángulo se mantiene.

		$(\Longrightarrow)$ Podemos escribir $\begin{cases}
				t_1 = \tan w_1 = {v_1'}/{u_1'} \\
				t_2 = \tan w_2 = {v_2'}/{u_2'}
			\end{cases}$ con $w_1$ y $w_2$ los ángulos que forman las curvas con respecto al eje $u$ en el plano $uv$. Entonces (si $u_1' >0$ y $u_2' > 0$):
		\[\cos\omega = \frac{E + F(t_1 + t_2) + Gt_1t_2}{\sqrt{E + 2Ft_1 + Gt_1^2}\sqrt{E + 2Ft_2 + Gt_2^2}}\]
		Si miramos cuándo las curvas son ortogonales en $S$ queda:
		\[\cos \omega = 0 \iff E + F(t_1 + t_2) + Gt_1t_2 = 0\iff \tan w_2 = \frac{-E/G -F/G\tan w_1}{F/G + \tan w_1}\]
		Como $f$ preserva los ángulos, las imágenes de las curvas también son ortogonales en $S^*$:
		\[\frac{-E^*/G^* -F^*/G^*\tan w_1}{F^*/G^* + \tan w_1} = \frac{-E/G -F/G\tan w_1}{F/G + \tan w_1} \implies \frac{E^*}{G^*} = \frac{E}{G}\]
	\end{dem}
\end{lem}

\begin{defn}
	Sea $\appl{f}{S}{S^*}$ una aplicación conforme, $\appl{\lambda}{U}{\R}$ es el factor de dilatación de $f \iff \odif{s_*}^2 = \lambda^2 \odif{s}^2$ donde $\odif{s}^2$ y $\odif{s_*}^2$ son las primeras formas fundamentales de $S$ y $S^*$ respectivamente.

	Las distancias en $S^*$ cerca del punto corriespondiente a $p \in S$ aumentan por un factor $\lambda(p)$.
\end{defn}

\subsubsection{Aplicación conforme al plano}

Ahora queremos encontrar una aplicación conforme de $S$ en un plano. Es decir, queremos encontrar $a(u, v)$ y $b(u, v)$ tales que
\[\odif{s}^2 = E\odif{u}^2 + 2F\odif{u}\odif{v} + G\odif{v}^2 = \lambda^2 \left[\odif{a}^2 + \odif{b}^2\right] = \lambda^2 \odif{s_*}^2\]
para cierto $\lambda$. Vemos que si se cumple esto, entonces $\odif{s}^2=0 \iff \odif{s_*}^2 = 0$.
\[\odif{s_*}^2 = 0 \iff \odif{a}^2 + \odif{b}^2 = 0 \iff (\odif{a} + i \odif{b})(\odif{a} - i \odif{b}) = 0\iff \odif{z}\odif{\bar{z}} = 0\]
donde $z = a + ib$ y $\bar{z}=a-ib$ es su conjugado. Si hacemos lo mismo con $\odif{s}^2$ obtenemos:
\[\odif{s}^2 = 0 \iff E\odif{u}^2 + 2F\odif{u}\odif{v} + G\odif{v}^2 = 0 \iff E\left(\frac{\odif{u}}{\odif{v}}\right)^2 + 2F\frac{\odif{u}}{\odif{v}} + G = 0\]
Por tanto, obetenemos $\ds \odif{s}^2= 0 \iff \boxed{\frac{\odif{u}}{\odif{v}} = -\frac{F}{E} \pm i \frac{\sqrt{EG - F^2}}{E}} \quad (*)$, $(EG-F^2> 0)$. Además, podemos escribir
\[\begin{aligned}
		\odif{s}^2 & = E^2 \odif{v}^2 \left[\frac{\odif{u}}{\odif{v}} - \left(\frac{-F}{E} + i \frac{\sqrt{EG-F^2}}{E}\right)\right]\left[\frac{\odif{u}}{\odif{v}} - \left(\frac{-F}{E} - i \frac{\sqrt{EG-F^2}}{E}\right)\right] \\
		           & = E^2 \left[\odif{u} - \left(\frac{-F}{E} + i \frac{\sqrt{EG-F^2}}{E}\right)\odif{v}\right]\left[\odif{u} - \left(\frac{-F}{E} - i \frac{\sqrt{EG-F^2}}{E}\right)\odif{v}\right]                              \\
		           & = E^2 \left[\odif{u} + \frac{F}{E} \odif{v} + i \frac{\sqrt{EG-F^2}}{E}\odif{v}\right]\left[\odif{u} + \frac{F}{E} \odif{v} - i \frac{\sqrt{EG-F^2}}{E}\odif{v}\right]
	\end{aligned}\]

Podemos resolver $(*)$ en ciertos casos por separación de variables y, ene general, poniendo $u = a_0 + a_1 v + a_2 v^2 + \dots$ con $a_j \in \C$ y agrupando por grados (si $E, F, G$ están definidas con series de Taylor).

Supongamos que las soluciones son $u = g_\pm(v) + \tex{cte} = g_1(v)\pm ig_2(v) + \tex{cte}$, entonces
\[u-g_\pm(v) = \tex{cte} \implies \odif{(u-g_\pm(v))} = \odif{u} - g'_\pm (v)\odif{v} = \odif{u} + \frac{F}{E}\odif{v} \pm i \frac{\sqrt{EG-F^2}}{E}\odif{v}\]
\[\implies \odif{s}^2 = E^2 \odif{\left(u + g_1(v) + i g_2(v)\right)}\odif{\left(u - g_1(v) - ig_2(v)\right)}\]
Por lo que al hacer el cambio $\begin{cases}
		U = u -g_1(v) + ig_2(v) \\
		V = u - g_1(v) - ig_2(v)
	\end{cases} \hspace{-0.5cm} \implies \boxed{\odif{s}^2 = E^2 \odif{U}\odif{V}}$.

Ahora, como tenemos que $\odif{s_*}^2 = \odif{a}^2 + \odif{b}^2 = \odif{z}\odif{\bar{z}}$, si ponemos $U = z$ y $V = \bar{z}$, obtenemos $\odif{s}^2 = E^2 \odif{s_*}^2$. Por tanto, hemos encontrado una aplicación conforme de $S$ en un plano con $a(u, v) = u - g_1(v)$ y $b(u, v) = g_2(v)$.

En realidad, tenemos muchas más aplicaciones conformes de una superficie en un plano y quedan (todas) dadas por:
\[\begin{cases}
		A = f(U) \\
		B = f_*(V)
	\end{cases} \hspace{-0.5cm} \implies \odif{s_*}^2 = \frac{\abs{f'(u-g_1(v) + ig_2(v))}^2}{E^2(u, v)}\odif{s}^2\]
con $f$ dada por su serie de Taylor y $f_*$ dada por la misma serie con coeficientes conjugados.

\begin{obs}
	Podemos componer aplicaciones conformes entre superficies y el plano para obtener aplicaciones conformes entre superficies generales.
\end{obs}

