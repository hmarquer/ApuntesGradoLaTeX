\section{Segunda forma fundamental - Pablo}
\subsection{El operador de forma}

Sea $S$ una superficie regular y $p \in S$ un punto, sea $T_pS$ el plano tangente a $S$ en $p$ y $N$ un vector unitario normal al $T_pS$.

Sea $\appl{\X}{U\subset\R^2}{\R^3}$ una carta local de $S$ en $p$ tal que $\X(u_0,v_0) = p$,\\
entonces $\{\X_u(u_0, v_0), \X_v(u_0, v_0)\}$ es una base de $T_pS$ y $N$ (salvo signo) viene dado por
\[\begin{aligned}
		N \colon S \subset \R^3 & \loongrightarrow \R^3                                                      \\
		\X(u, v) = p            & \loongmapsto   N(p)=\frac{\X_u \times \X_v}{\norm{\X_u \times \X_v}}(u, v)
	\end{aligned}\]
El objetivo es estudiar la forma de la superficie $S$ cerca del punto $p$, donde ``forma'' se refiere a cómo se curva la superficie. Esto equivale a estudiar la variación del vector normal $N$ en distintas direcciones.

\begin{defn}[Operador de forma]
	Sea $S$ una superficie regular y $p \in S$ un punto, $\appl{F_p}{T_pS}{\R}$ es el operador de forma de $S$ en $\ds p \iff \forall w \in T_pS : \F_p(w) = -\left.\odv{}{t}N(\alpha(t))\right|_{t=0}$
	donde $\alpha$ es una curva regular en $S$ tal que $\alpha(0) = p$ y $\alpha'(0) = w$.
\end{defn}

\subsubsection{Expresión del operador de forma en una carta}

Sea $\appl{\X}{U\subset\R^2}{\R^3}$ una carta local de $S$ superficie regular en $p\in S$ tal que $\X(u_0,v_0) = p$, sea $w \in T_pS = \mathcal{L}\{\X_u(u_0, v_0), \X_v(u_0, v_0)\} \implies w = a\X_u(u_0, v_0) + b\X_v(u_0, v_0)$.

Ahora tomamos una curva regular $\alpha(t) = \X(u(t), v(t))$ en $S$ tal que $\alpha(0) = p$ y $\alpha'(0) = w$. \\
Es decir, $(u(0), v(0)) = (u_0, v_0)$ y $(u'(0), v'(0)) = (a, b)$.

Por otro lado tenemos $(N\circ \alpha)(t) = N(\X(u(t), v(t))) = (N\circ \X)(u(t), v(t))$
\[\tex{y el diagrama queda } \begin{tikzcd}[column sep=huge]
		& S \subset \R^3 \arrow{r}{N} & \R^3 \\
		I\subset \R \arrow{ru}{\alpha} \arrow{r}{(u, v)} & U \subset \R^2 \arrow{u}{\X} \arrow{ur}{N\circ \X} &
	\end{tikzcd}\]
\[\implies \F_p(w) = -\left.\odv{}{t}N(\alpha(t))\right|_{t=0} = - (N\circ \X)_u(u(0), v(0))u'(0) - (N\circ \X)_v(u(0), v(0))v'(0)\]
\[\implies \F_p(a\X_u(u_0, v_0) + b\X_v(u_0, v_0)) = -a(N\circ \X)_u(u_0, v_0) - b(N\circ \X)_v(u_0, v_0)\]
O, simplificando notación, $\ds \boxed{\F_p(a\X_u + b\X_v) = -a(N\circ \X)_u - b(N\circ \X)_v}$.

\begin{obs} De la expresión anterior se deduce que
	\begin{itemize}
		\item $\F_p(w)$ no depende de las curva $\alpha$ de la definición original.
		\item Dado $w\in T_pS$, para calcular $\F_p(w)$ basta con conocer las derivadas de $N\circ \X$ en $(u_0, v_0)$.
		\item De hecho, $\F_p(\X_u) = - (N\circ \X)_u$ y $\F_p(\X_v) = - (N\circ \X)_v$.
		\item $\F_p$ es lineal porque $\forall \lambda_1, \lambda_2 \in \R, w_1, w_2 \in T_pS : \F_p(\lambda_1w_1 + \lambda_2w_2) = \lambda_1\F_p(w_1) + \lambda_2\F_p(w_2)$.
		\item $\forall w \in T_pS : \F_p(w) \in T_pS$.
		\item $\appl{\F_p}{T_pS}{T_pS}$ es una aplicación lineal autoadjunta, es decir,
		      \[\forall w_1, w_2 \in T_pS : \langle \F_p(w_1), w_2 \rangle = \langle w_1, \F_p(w_2) \rangle\]
	\end{itemize}
\end{obs}

\subsubsection{Expresión matricial del operador de forma}

Tomando la base $\{\X_u, \X_v\}$ de $T_pS$, tenemos que $\begin{cases}
		\F_p(\X_u) = - (N\circ \X)_u \eqdef A \X_u + B\X_v \\
		\F_p(\X_v) = - (N\circ \X)_v \eqdef C \X_u + D\X_v
	\end{cases}$.
\[\forall w = a\X_u + b\X_v \in T_pS : \F_p(w) = x\X_u + y\X_v = a \F_p(\X_u) + b \F_p(\X_v)\]
\[\implies \begin{pmatrix}
		x \\ y
	\end{pmatrix} = \begin{pmatrix}
		A & C \\ B & D
	\end{pmatrix} \begin{pmatrix}
		a \\ b
	\end{pmatrix} \iff \begin{pmatrix}
		\F_p(w)_1 \\ \F_p(w)_2
	\end{pmatrix} = \begin{pmatrix}
		\F_p(\X_u)_1 & \F_p(\X_v)_1 \\
		\F_p(\X_u)_2 & \F_p(\X_v)_2
	\end{pmatrix} \begin{pmatrix}
		w_1 \\ w_2
	\end{pmatrix}\]
donde $\F_p(w) = (\F_p(w)_1, \F_p(w)_2) \we \F_p(\X_u) = (\F_p(\X_u)_1, \F_p(\X_u)_2) \we F_p(\X_v) = (\F_p(\X_v)_1, \F_p(\X_v)_2)$ y $w = (w_1, w_2)$ todos en base $\{\X_u, \X_v\}$.

Para hallar $A, B, C, D$ consideramos los siguientes productos escalares:
\[\begin{aligned}
		\F_p(\X_u) \cdot \X_u & = (A\X_u + B\X_v) \cdot \X_u = AE+BF \\
		\F_p(\X_u) \cdot \X_v & = (A\X_u + B\X_v) \cdot \X_v = AF+BG \\
		\F_p(\X_v) \cdot \X_u & = (C\X_u + D\X_v) \cdot \X_u = CE+DF \\
		\F_p(\X_v) \cdot \X_v & = (C\X_u + D\X_v) \cdot \X_v = CF+DG
	\end{aligned}\]
Conviene definir las siguientes funciones $\appl{e, f, g}{U}{\R}$:
\[\begin{aligned}
		e \defeq \F_p(\X_u) \cdot \X_u & = -(N\circ \X)_u \cdot \X_u                             \\
		f \defeq \F_p(\X_u) \cdot \X_v & = -(N\circ \X)_u \cdot \X_v = -(N\circ \X)_v \cdot \X_u \\
		g \defeq \F_p(\X_v) \cdot \X_v & = -(N\circ \X)_v \cdot \X_v
	\end{aligned}\]
(Si cambiamos $N$ por $-N$, entonces todas las funciones cambian de signo).
\[\implies \begin{pmatrix}
		e & f \\ f & g
	\end{pmatrix} = \begin{pmatrix}
		E & F \\ F & G
	\end{pmatrix} \begin{pmatrix}
		A & C \\ B & D
	\end{pmatrix}\implies \begin{pmatrix}
		A & C \\ B & D
	\end{pmatrix} = \begin{pmatrix}
		E & F \\ F & G
	\end{pmatrix}^{-1} \begin{pmatrix}
		e & f \\ f & g
	\end{pmatrix}\]
\[\implies \tex{matriz de } \F_p = \begin{pmatrix}
		\F_p(\X_u)_1 & \F_p(\X_u)_2 \\
		\F_p(\X_v)_1 & \F_p(\X_v)_2
	\end{pmatrix} = \frac{1}{EG-F^2} \begin{pmatrix}
		G & -F \\ -F & E
	\end{pmatrix} \begin{pmatrix}
		e & f \\ f & g
	\end{pmatrix}\]
Esto nos da fórmulas para $A, B, C, D$ en función de $E, F, G$ y $e, f, g$.

\subsubsection{Fórmulas alternativas para \textit{e, f, g}}
Calcular estas funciones por su definición suele ser laborioso. Observamos que
\[(N\circ \X) \cdot \X_u \equiv 0 \implies (N\circ \X)_u \cdot \X_u + (N\circ \X) \cdot \X_{uu} \equiv 0\]
y, por tanto, podemos escribir $e = - (N\circ \X)_u \cdot \X_u =  (N\circ \X) \cdot \X_{uu}$ y, en general, se tiene que
\[e = (N\circ \X) \cdot \X_{uu} \quad\we\quad f = (N\circ \X) \cdot \X_{uv} \quad\we\quad g = (N\circ \X) \cdot \X_{vv}\]

\subsection{Formas fundamentales}
Extendemos la definición de primera forma  fundamental que usábamos:
\begin{defn}[Primera forma fundamental]
	Sea $S$ una superficie regular y $p \in S$ un punto, $\appl{I_p}{T_pS\times T_pS}{\R}$ es la primera forma fundamental de $S$ en $p$ \[\iff \forall w, w' \in T_pS : I_p(w, w') = \langle w, w' \rangle\]
	Es una forma bilineal simétrica.\\
	Si tomamos una carta $\X$ de $S$ en $p$ y $\{\X_u, \X_v\}$ como base de $T_pS$, entonces
	\[I_p(w, w') = w_1w_1'E + (w_1w_2' + w_2w_1')F + w_2w_2'G\]
\end{defn}

\begin{defn}[Segunda forma fundamental]
	Sea $S$ una superficie regular y $p \in S$ un punto, $\appl{II_p}{T_pS\times T_pS}{\R}$ es la segunda forma fundamental de $S$ en $p$ \[\iff \forall w, w' \in T_pS : II_p(w, w') = \langle \F_p(w), w' \rangle\]
	Es también una forma bilineal simétrica.\\
	Si tomamos una carta $\X$ de $S$ en $p$ y $\{\X_u, \X_v\}$ como base de $T_pS$, entonces
	\[II_p(w, w') = w_1w_1'e + (w_1w_2' + w_2w_1')f + w_2w_2'g\]
\end{defn}

\subsubsection{Expresiones matriciales de las formas fundamentales}
\[I_p(w, w') = \begin{pmatrix} w_1 & w_2 \end{pmatrix} \begin{pmatrix}
		E & F \\ F & G
	\end{pmatrix} \begin{pmatrix}
		w_1' \\ w_2'
	\end{pmatrix} \quad\we\quad II_p(w, w') = \begin{pmatrix} w_1 & w_2 \end{pmatrix} \begin{pmatrix}
		e & f \\ f & g
	\end{pmatrix} \begin{pmatrix}
		w_1' \\ w_2'
	\end{pmatrix}\]
\[\F_p(w) = \begin{pmatrix}
		A & C \\ B & D
	\end{pmatrix}\begin{pmatrix}
		w_1 \\ w_2
	\end{pmatrix}\tex{ donde } \begin{pmatrix}
		e & f \\ f & g
	\end{pmatrix} = \begin{pmatrix}
		E & F \\ F & G
	\end{pmatrix} \begin{pmatrix}
		A & C \\ B & D
	\end{pmatrix}\]

\subsubsection{Diagonalización del operador de forma}

Como $\appl{\F_p}{T_pS}{T_pS}$ es una aplicación lineal autoadjunta,
\begin{enumerate}
	\item Si $\{\til{e}_1, \til{e}_2\}$ es una base ortonormal de $T_pS$, entonces la matriz de $\F$ en esa base es simétrica.
	\item Podemos (orto)diagonalizar $\F_p$, i.e., $\exists$ una base ortonormal de $T_pS$ en la que la matriz de $\F_p$ es diagonal.
\end{enumerate}

Por tanto, $\exists \kappa_1, \kappa_2 \in \R, e_1, e_2 \in T_pS$ tales que $e_1 \perp e_2 \we \F_p(e_1) = \kappa_1e_1 \we \F_p(e_2) = \kappa_2e_2$ donde $k_1, k_2$ son autovalores de $\F_p$ y $e_1, e_2$ son (una base de) autovectores de $\F_p$.

A $\kappa_1, \kappa_2$ se les llama curvaturas principales y a $\pm e_1, \pm e_2$ direcciones principales. Se pueden obtener diagonalizando la matriz de $\F_p$ en la base $\{\X_u, \X_v\}$.

En la base ortonormal $\{e_1, e_2\}$ (de autovectores, de direcciones principales), si dos vectores $w, w' \in T_pS$ se escriben como $w = xe_1 + ye_2$ y $w'=x'e_1 + y'e_2$:
\[\begin{aligned}
	\implies \F_p(w) &= x\F_p(e_1) + y\F_p(e_2) = x\kappa_1e_1 + y\kappa_2e_2\\
	I_p(w, w') &= \langle w, w' \rangle = xx' + yy'\\
	II_p(w, w') &= \langle \F_p(w), w' \rangle = x\kappa_1x' + y\kappa_2y'
\end{aligned}\]
Más aún, si $w \in T_pS$ se escribe como
\begin{center}
\begin{tabular}{c|c|c}
	& $w = xe_1 + ye_2$ & $w = w_1\X_u + w_2\X_v$ \\
	\hline
	$I_p(w, w) = \norm{w}^2$ & $x^2 + y^2$ & $w_1^2E + 2w_1w_2F + w_2^2G$ \\
	$II_p(w, w) = \langle \F_p(w), w \rangle$ & $\kappa_1x^2 + \kappa_2y^2$ & $w_1^2e + 2w_1w_2f + w_2^2g$
\end{tabular}
\end{center}

\subsection{Curvatura normal}
\begin{defn}[Curvatura normal]
	Sea $S$ una superficie regular y $p \in S$ un punto, $\appl{k_p}{T_pS}{\R}$ es la curvatura normal de $S$ en $p$ \[\iff \forall w \in T_pS \tex{ unitario} : k_p(w) = \F_p(w) \cdot w = II_p(w, w)\]
\end{defn}

\pagebreak
\section{Curvatura de superficies - Ubis}

Vamos a buscar máximos y mínimos de $z = f(x,y)$. Si $\nabla f = 0$, podemos estar ante un máximo, mínimo, punto de silla o punto singular. Para estudiar la curvatura, miro la curvas resultantes al seccionar la superficie por cada plano perpendicular al tangente.

(Ojo: la curvatura de una de esas curvas no tiene por qué dar información sobre la curvatura de la superficie).

\begin{itemize}
	\item Queremos estudiar todas las curvaturas a la vez e interesa ver si son positivas (sentido de $\ds N=\frac{\X_u\times\X_v}{\norm{\X_u\times\X_v}}$) o negativas al coger siempre la misma normal.
	\item En general, vamos a rotar la superficie para obtener una gráfica con plano tangente horizontal. Ahí se calcula la curvatura escalar proque se ha convertido un punto $p$ cualquiera en un crítico. ¡Los giros no afectan a la curvatura!
\end{itemize}

\subsection{Curvatura con formas cuadráticas}

Sea $\appl{f}{\R^2}{\R^3}$ dada por su expresión de Taylor hasta grado 2 en el $(0,0,0)$:
\[\begin{aligned}
		f(x, y) & \approx f(0,0) + f_x(0,0)x + f_y(0,0)y + \frac{1}{2}f_{xx}(0,0)x^2 + f_{xy}(0,0)xy + \frac{1}{2}f_{yy}(0,0)y^2 \\
		        & \approx 0 + 0x + 0y + \frac{ax^2 + 2bxy + cy^2}{2}
	\end{aligned}\]
\[\implies \pdv[2]{f}{x} (x, y) = a \we \pdv{f}{x, y} (x, y) = b \we \pdv[2]{f}{y} (x, y) = c \implies f(x, y) \approx \begin{pmatrix} x & y \end{pmatrix} \begin{pmatrix}
		\sfrac{a}{2} & \sfrac{b}{2} \\ \sfrac{b}{2} & \sfrac{c}{2}
	\end{pmatrix} \begin{pmatrix}
		x \\ y
	\end{pmatrix}\]
Definimos el plano $\ds \frac{x}{\cos{\theta_0}}=\frac{y}{\sin{\theta_0}}$ perpendicular al plano tangente a la superficie en $(0,0,0)$ puesto que $\nabla f(0,0) = 0$ y lo intersecamos con la superficie para obtener la curva $\appl{\alpha}{\R}{\R^3}$ dada por $\alpha(r) = (r\cos\theta_0, r\sin\theta_0, z_{\theta_0}(r))$ donde
\[z_{\theta_0}(r) = \frac{a\cos^2\theta_0 + 2b\cos\theta_0\sin\theta_0 + c\sin^2\theta_0}{2} r^2\]
\[\implies \begin{cases}
		z_{\theta_0}'(r) = (a\cos^2\theta_0 + 2b\cos\theta_0\sin\theta_0 + c\sin^2\theta_0)r \\
		z_{\theta_0}''(r) = a\cos^2\theta_0 + 2b\cos\theta_0\sin\theta_0 + c\sin^2\theta_0
	\end{cases}\]
Y como ya sabemos que $\ds \kappa = \frac{\norm{\alpha' \times \alpha''}}{\norm{\alpha'}^3} =\frac{\abs{z_{\theta_0}''}}{(1 + z_{\theta_0}'^2)^{\sfrac{3}{2}}}$, tenemos que en el punto $(0,0,0)$
\[\begin{aligned}
		\kappa(\theta_0) & = \frac{\abs{a\cos^2\theta_0 + 2b\cos\theta_0\sin\theta_0 + c\sin^2\theta_0}}{(1 + (a\cos^2\theta_0 + 2b\cos\theta_0\sin\theta_0 + c\sin^2\theta_0)^2(0)^2)^{\sfrac{3}{2}}} \\
		                 & = \abs{a\cos^2\theta_0 + 2b\cos\theta_0\sin\theta_0 + c\sin^2\theta_0}
	\end{aligned}\]

Esta fórmula nos indica cómo se curva $S$ en el punto $(0,0,0)$ en función del plano perpendicular al plano tangente que escojamos, indicado por el valor de $\theta_0$.

Podemos encontrar el máximo y el mínimo de $\kappa'(\theta_0)$ si derivamos respecto de $r$ y despejamos.
\[\implies \kappa'(\theta_0) = 2(c - a)\cos\theta_0\sin\theta_0 + 2b(\cos^2\theta_0 - \sin^2\theta_0)\]
\[\implies \kappa'(\theta_0)=0 \iff 2(a-c)\cos\theta_0\sin\theta_0 = 2b(\cos^2\theta_0 - \sin^2\theta_0)\]
\[\iff (a-c)\sin(2\theta_0) = b\cos(2\theta_0) \iff \tan(2\theta_0) = \frac{b}{a-c}\]
\[\implies \theta_0 = \begin{cases}
		-\frac{\pi}{4} \ve +\frac{\pi}{4}                                                     & \tex{ si } a=c \we b\ne0 \\
		\arctan\left(\frac{b}{a-c}\right) \ve \arctan\left(\frac{b}{a-c}\right)+\frac{\pi}{2} & \tex{ si } a\neq c
	\end{cases}\]
Es decir, encontramos dos soluciones que pertenecen a planos perpendiculares entre sí \\(a no ser que $a=c \we b=0$, en cuyo caso la curvatura es la misma en todos los planos perpendiculares al plano tangente).

Estudiemos ahora la relación que tiene esto con los autovalores de la matriz jacobiana $\ds \tex{J}f (0,0,0) = \begin{pmatrix} a & b \\ b & c \end{pmatrix}$. Si llamamos $(x(r), y(r)) = (r\cos\theta_0, r\sin\theta_0)$, entonces
\[\kappa(\theta_0) = a\left(\frac{x(r)}{r}\right)^2 + 2b \frac{x(r)}{r} \frac{y(r)}{r} + c\left(\frac{y(r)}{r}\right)^2= \frac{ax^2(r) + 2bx(r)y(r) + cy^2(r)}{x^2(r) + y^2(r)}\]
\[\implies \tex{(obviando la dependencia de $r$) } \kappa(x, y) =  \frac{
		\begin{pmatrix} x & y\end{pmatrix} \begin{pmatrix} a & b \\ b & c \end{pmatrix} \begin{pmatrix} x \\ y \end{pmatrix}
	}{
		\begin{pmatrix} x & y \end{pmatrix} \begin{pmatrix} x \\ y \end{pmatrix}
	}
	\implies \kappa(\vec{w}) = \frac{\vec{w}^T Q \vec{w}}{\norm{\vec{w}}^2}\]
con $Q$ matriz simétrica de autovectores ortonormales $\vec{w}_1, \vec{w}_2$ y autovalores $\lambda_1, \lambda_2$.
\[w=a_1\vec{w}_1 + a_2\vec{w}_2 \implies \kappa(w) = \frac{a_1^2\lambda_1 + a_2^2\lambda_2}{a_1^2 + a_2^2} = \lambda_1\frac{a_1^2}{a_1^2 + a_2^2} + \lambda_2\frac{a_2^2}{a_1^2 + a_2^2}\leq \max\{\lambda_1, \lambda_2\}\]

Como la curvatura máxima y mínima son ortogonales, podemos girar los ejes para que coincidan con $X$ e $Y$: $\ds z = f(x, y) = \kappa_1 \frac{x^2}{2} + \kappa_2 \frac{y^2}{2} \tex{ donde } \kappa_1 \geq \kappa_2$ de modo que
\[\begin{pmatrix}
		a & b \\ b & c
	\end{pmatrix} = \begin{pmatrix}
		\cos{(-\theta_1)} & -\sin{(-\theta_1)} \\ \sin{(-\theta_1)} & \cos{(-\theta_1)}
	\end{pmatrix} \begin{pmatrix}
		\kappa_1 & 0 \\ 0 & \kappa_2
	\end{pmatrix} \begin{pmatrix}
		\cos{\theta_1} & -\sin{\theta_1} \\ \sin{\theta_1} & \cos{\theta_1}
	\end{pmatrix}\]
donde $\vec{w}_1 = (\cos{\theta_1}, \sin{\theta_1})$ y $\vec{w}_2 = (-\sin{\theta_1}, \cos{\theta_1})$.

Las curvaturas $\kappa_1$ y $\kappa_2$ tienen el signo determinado por la orientación relativa de la normal de la curva que resulta al intersecar la superficie con el plano perpendicular al plano tangente con respecto a la normal de la superficie.

Si $\kappa_1 \cdot \kappa_2 > 0$, entonces las curvas de nivel de $z=f(x, y)$ son elipses;\\ si $\kappa_1 \cdot \kappa_2 < 0$, entonces son hipérbolas.

\begin{defn}
	Sea $S$ una superficie regular y $p\in S$ un punto. Sea $N = \frac{\X_u \times \X_v}{\norm{\X_u \times \X_v}}$ la normal unitaria a $S$ en $p$. $\appl{\hat{\kappa}_p}{T_pS}{\R}$ es la curvatura de $S$ en $p \iff \forall w \in T_pS : \hat{\kappa}_p(w) = \kappa_{\gamma}(t_0)$ donde $\gamma$ es la curva resultante al intersecar $S$ con el plano que contiene a $w$ y es perpendicular a $N$, $t_0$ es tal que $\gamma(t_0) = p$ y $\kappa_{\gamma}$ denota la curvatura de $\gamma$.

	Vemos que $\hat{\kappa}_p(-\vec{w}) = \hat{\kappa}_p(\vec{w})$ y que $\hat{\kappa}_p\left(\frac{\vec{w}}{\norm{\vec{w}}}\right) = \hat{\kappa}_p(\vec{w})$ porque $\forall a \in \R : \hat{\kappa}_p(a\vec{w}) = \hat{\kappa}_p(\vec{w})$.
\end{defn}

\begin{teo}
	Sean $S$ una curva regular $\implies \forall p \in S : \exists {w_1}, {w_2} \in T_pS $ ortogonales tales que $ \hat{\kappa}_p(w_1), \hat{\kappa}_p(w_2)$ son las curvaturas máx y mín en $p$.
\end{teo}
\begin{obs}
	\begin{itemize}
		\item Si la curvatura es constante, esos son los únicos extremos de $\hat{\kappa}_p$.
		\item Si $w$ forma un ángulo $\theta$ con $w_1$, entonces $\hat{\kappa}(w) = \hat{\kappa}_1 \cos^2\theta + \hat{\kappa}_2\sin^2\theta$.
		\item $w_1$ y $w_2$ son las direcciones principales y $\hat{\kappa}_1$ y $\hat{\kappa}_2$ las curvaturas principales.
		\item $\kappa_1 \cdot \kappa_2 > 0 \implies p$ elíptico; $\kappa_1 \cdot \kappa_2 = 0 \implies p$ parabólico; $\kappa_1 \cdot \kappa_2 < 0 \implies p$ umbílico.
	\end{itemize}
\end{obs}

\begin{cor}
	Sea $S$ una superficie regular y $p\in S$ un punto. Entonces, $\exists B$ bola alrededor de $p$ y $\varepsilon_0 > 0$ tales que $S\cap B \cap J$ es una curva que (hasta orden 2) es una elipse o una hipérbola donde $J$ es un plano paralelo a $T_pS$ a distancia menor que $\varepsilon_0$ de $p$.
\end{cor}

\subsection{Cálculo de curvatura en una carta cualquiera}

Sea $\appl{\X}{U\subset\R^2}{\R^3}$ una carta local de $S$ en $p$ tal que $\X(u_0, v_0) = p$, tomamos la normal unitaria $N = \frac{\X_u \times \X_v}{\norm{\X_u \times \X_v}}$. Sea $w = a\X_u + b\X_v \in T_pS$ y $\gamma(t) = \X(u(t), v(t))$ una curva regular parametrizada por longitud de arco en $S$ tal que $\gamma(0) = p$ y $\gamma'(0) = w$.
