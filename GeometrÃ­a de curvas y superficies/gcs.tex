\documentclass[12pt]{article}

\usepackage{../style}

\newcommand{\tngnt}{\operatorname{t}}
\newcommand{\nrml}{\operatorname{n}}

\title{ \normalsize
		\HRule{0.06cm} \\ [0.5cm] %\vspace{10pt}
		\LARGE \textbf{\uppercase{Geometría de curvas y superficies}
		\HRule{0.06cm} \\ \LARGE{Segundo del Grado en Matemáticas} \vfill}}
\date{1 de Febrero, 2024}
\author{\textbf{Hugo Marquerie} \\ 
		Profesor: \\
		Facultad de Ciencias - Universidad Autónoma de Madrid \\
		Segundo cuatrimestre 2023 - 2024}


\begin{document}

\onehalfspacing
\setlength{\abovedisplayskip}{0.15cm}
\setlength{\belowdisplayskip}{0.25cm}

\maketitle
\thispagestyle{empty}
\clearpage
\pagenumbering{arabic} 

\section{Tema 1: Curvas}

\subsection{Curvas regulares, trazas, velocidades y tangencias}
\begin{defn}[Curva]
	$\gamma$ es una curva $\iff \appl{\gamma}{I=(a,b)\subseteq\R}{\R^3}$ es $C^{\infty}$
	\begin{itemize}
		\item Para casi todo lo que sigue basta con que $\gamma \in C^2$. Para contrastar, en alguna discusión posterior, admitiremos curvas que son solo $C^0 \ve C^1$.
		\item El intervalo $I = (a,b)$ puede ser no acotado en $\R$.
	\end{itemize}
\end{defn}
\begin{defn}[Traza]
	Sea $\gamma$ una curva, $\trz{\gamma}=\{\gamma(t):t\in I\}\subset \R^3$. También se dice que $\gamma$ es parametrización de $\trz{\gamma}$.
\end{defn}

\begin{obs}
	Dos curvas distintas pueden tener la misma traza y corresponden a parametrizaciones diferentes. De hecho, si $\appl{\gamma}{I}{\R^3}$ es una curva y $\appl{g}{J}{I}$ es una función sobreyectiva $C^{\infty}$
	\[\implies \appl{\mu}{J}{\R^3} : \mu(u)=\gamma\left(g(u)\right)\tex{ es una curva de misma traza que } \gamma\]
\end{obs}
\begin{defn}[Velocidad y rapidez]
	Sea $\appl{\gamma}{I}{\R^3}$ una curva, el vector velocidad (velocidad) de $\gamma$ en $t$ es $\gamma'(t)=\nabla\gamma(t)\in \R^3$. La norma $\norm{\gamma'(t)}$ se conoce como la rapidez.
\end{defn}
\begin{defn}[Curva regular]
	Sea $\appl{\gamma}{I}{\R^3}$, es regular $\iff \forall t \in I : \gamma'(t)\ne 0$
\end{defn}
\begin{ejem}
	El conjunto de puntos $V=\{(x, y)\in \R^2:-1<x<1\we y=|x|\}$ no es traza de ninguna curva regular.\\
	\indent Pongamos que $\gamma(t)=(x(t), y(t))$ es curva regular con traza $V$, y que $\gamma(0)=(0,0)$. Entonces $y(t)$ tiene un mínimo en $t = 0$ y $y'(0) = 0$. Si fuera $x'(0) > 0$, entonces $x(t) > 0$ para $t \in (0, \delta)$ y, por tanto, $x(t) =y(t)$ para $t \in \left[0, \delta\right)$ y, por tanto $x'(0) = y'(0) = 0$. Contradicción: la curva $\gamma$ no es regular. También conduce a contradicción el que $x'(0) = 0$.
\end{ejem}
\begin{defn}[Recta tangente a una curva regular]
	Sea $\appl{\gamma}{I}{\R^3}$ una curva regular y $t_0\in I$, $\appl{\eta}{\R}{\R^3}$ es la recta tangente a $\gamma$ en $\gamma(t) \iff \eta(u)=\gamma(t_0)+u\gamma'(t_0)$
\end{defn}

\subsection{Longitud de curvas y reparametrizaciones}
\subsubsection{Poligonales}
Sea $\appl{\gamma}{I}{\R^3}$ una curva (no necesariamente regular) y $\left[c, d\right] \subset I$. Definimos una poligonal de $\gamma$ entre $\gamma(c)$ y $\gamma(d)$ tomando una partición de $\left[c, d\right]$:
\[c=t_0<t_1<\cdots<t_{n-1}<t_n=d\]
La poligonal $\Pi$ asociada a esa partición es la lista de segmentos:
\[\Pi=\left(\left[\gamma(t_0), \gamma(t_1)\right], \left[\gamma(t_1), \gamma(t_2)\right], \dots, \left[\gamma(t_{n-1}), \gamma(t_n)\right]\right)\]
La longitud de la poligonal $\Pi$ se denota $L(\Pi)$ se define entonces de forma natural como:
\[L(\Pi)=\sum_{j=1}^n \norm{\gamma(t_j)-\gamma(t_{j-1})}\]

\subsubsection{Definición de longitud de curva}
\begin{defn}[Longitud de una curva]
	Sea $\appl{\gamma}{I}{\R^3}$ una curva $C^0$, $\left[c,d\right] \subset  I$, se define la longitud como: \[\ds L(\gamma; c, d)=\sup \left\{L(\Pi) : \Pi\tex{ poligonal de } \gamma \tex{ entre }\gamma(c)\tex{ y }\gamma(d)\right\}\]
	En general, para una curva sólo continua este supremo puede ser $\infty$. Por ejemplo, para la curva copo de nieve de Von Koch.
\end{defn}

\subsubsection{Teorema de cálculo de longitud}
\begin{teo}
	Sea $\appl{\gamma}{I}{\R^3}$ una curva regular y $\left[c,d\right] \subset  I$
	\[\implies L(\gamma; c, d) = \int_c^d\norm{\gamma'(t)}\odif{t}\]
	\begin{dem}
		En la demostración denotamos
		\begin{itemize}
			\item $\ds L \defeq L(\gamma; c, d)$, que es supremo de las longitudes de las poligonales.
			\item $\ds J \defeq \int_c^d\norm{\gamma'(t)}\odif{t}$
		\end{itemize}
		Veamos que $\ds L\leq J \we J \leq L$:
		\begin{lem} \label{lem1}
			Sea $\appl{u}{I}{\R^3}$ una aplicación y sean $c, d \in I$, con $c<d$.
			\[\implies\norm{\int_c^du(t)\odif{t}} \leq \int_c^d\norm{u(t)}\odif{t}\]
		\end{lem}
		\begin{dem}
			Usamos que para cualquier $v$ se tiene $\ds\norm{v}=\sup\{|v\cdot w|:\norm{w}=1\}$. Pongamos $\ds v=\int_c^du(t)\odif{t}$ y sea $w$ con $\norm{w}=1$
			\[\implies v\cdot w=\int_c^d\left(u(t)\cdot w\right)\odif{t} \we \abs{v\cdot w}\leq\int_c^d\abs{u(t)\cdot w}  \odif{t} \leq  \int_c^d\norm{u(t)} \odif{t}\]
			Lo anterior es cierto para cualquier $w$ con $\norm{w}=1$.
		\end{dem}
		$\mathbf{\left[L\leq J\right]}$ Para una poligonal $\Pi$ con partición $t_0=c<t_1<\cdots<t_n=d$, se tiene que
		\[L(\Pi)=\sum_{j=1}^n\norm{\gamma(t_j)-\gamma(t_{j-1})} = \sum_{j=1}^n\norm{\int_{t_{j- 1}}^{t_j}\gamma'(t)\odif{t}}\]
		Por el Lema \ref{lem1}, se tiene \[\ds L(\Pi) \leq \sum_{j=1}^n\int_{t_{j-1}}^{t_j}\norm{\gamma'(t)}\odif{t} =   \int_{t_0}^{t_n}\norm{\gamma'(t)}\odif{t} = J \implies L \leq J\]
		$\mathbf{\left[L\leq J\right]}$ Fijemos $\varepsilon > 0$. Por la continuidad uniforme de $\gamma'\in C^0$, tenemos  $\delta>0$ tal que si $c\leq t\leq t'\leq d$ \\
		\centerline{\begin{itemize*}[itemjoin=\hspace{1cm}]
			\item son tales que $|t'-t|\leq \delta$
			\item entonces $\norm{\gamma(t')-\gamma(t)}\leq \varepsilon$
		\end{itemize*}}
		(El $\delta$ no depende de los puntos $t$ y $t'$, sólo de $\varepsilon$). \\
		Tomamos una partición $\ds t_0=c<t_1<\cdots <t_n=d$ para la que cumpla que $\ds t_j-t_{j-1}<\delta$ para $\ds j=1, \dots, n$. \\
		Sea $\Pi$ la poligonal dada por esa partición. Entonces:
		\[L(\Pi)=\sum_{j=0}^{n-1}\norm{\gamma(t_{j+1})-\gamma(t_j)}=\sum_{j=0}^{n-1}\norm{\int_{t_{j-1}}^{t_j}\gamma'(t)\odif{t}}\]
		Sea $\ds u_j\in \left[t_{j-1}, t_j\right]$ tal que $\ds\norm{\gamma'(u_j)}=\max_{t\in \left[t_{j-1}, t_j\right]}\norm{\gamma'(t)}$. \\
		Por un lado, $\ds J=\int_c^d\norm{\gamma'(t)}\odif{t} = \sum_{j=1}^n \int_{t_{j-1}}^{t_j}\norm{\gamma'(t)}\odif{t} \leq \sum_{j=1}^n\norm{\gamma'(u_j)}(t_j-t_{j-1}) \hspace{0.5cm} (\star)$ \\
		Por otro lado, \[\ds \gamma(t_{j+1})-\gamma(t_j) = \int_{t_j}^{t_{j+1}}\gamma'(t)\odif{t}=\left(\int_{t_j}^{t_{j+1}}(\gamma'(t)-\gamma'(u_j))\odif{t}\right)+\gamma'(u_j)(t_{j+1}-t_j)\]
		Así que \begin{equation*}
			\begin{split}
				\norm{\gamma(t_{j+1})-\gamma(t_j)} & \geq  \norm{\gamma'(u_j)}|t_{j+1}-t_j|-\norm{\int_{t_j}^{t_{j+1}}(\gamma'(t)-\gamma'(u_j))\odif{t}} \\ & \geq  \norm{\gamma'(u_j)}|t_{j+1}-t_j|-\varepsilon(t_{j+1}-t_j)
			\end{split}
		\end{equation*}
		usando el Lema \ref{lem1} y las condiciones de la partición.
		Por lo tanto, \begin{equation*}\begin{split}
			L(\Pi) &=\sum_{j=0}^{n-1}\norm{\gamma(t_{j+1}-\gamma(t_j))}
			\geq \sum_{j=0}^{n-1}\norm{\gamma'(u_j)}(t_{j+1}-t_j)-\varepsilon\sum_{j=0}^{n-1}(t_{j+1}-t_j) \geq J -\varepsilon(d-c)
		\end{split}\end{equation*}
		donde se ha usado $(\star)$. En suma
		\[\forall \varepsilon > 0 : J\leq L(\Pi) + \varepsilon(d-c)\geq L + \varepsilon(d-c) \implies J \leq L\]
	\end{dem}
\end{teo}
\begin{ejem}[Longitud gráfica de una función] 
	$\ds \forall t \in \left[c, d\right] : \gamma(t)=(t, f(t))$ donde $\appl{f}{\left[c, d\right]}{\R}$ es una función deribable.
	\[ \norm{\gamma'(t)}=\sqrt{1+(f'(t))^2} \implies \tex{Long gráfica }=\int_c^d \sqrt{1+(f'(t))^2} \odif{t}\]
\end{ejem}



\subsubsection{Reparametrizaciones}

Dada una curva $\appl{\gamma}{I}{\R^3}$, queremos recorrer su traza con otra curva.
\begin{defn}
	Sean $\appl{h}{J}{I}$ un difeomorfismo (biyección tal que tanto $h$ como $h^{-1}$ son $C^\infty$) y $\appl{\gamma}{I}{\R^3}$ una curva, $\appl{\eta}{J}{\R^3}$ es una reparametrización de $\gamma$
	\[ \iff \forall u \in J : \eta(u)=\gamma(h(u))\]
	\begin{itemize}[topsep=1pt, itemsep=1pt,parsep=3pt]
		\item $\eta$ es curva porque $\gamma$ lo es y $h$ es $C^\infty$.
		\item $\eta$ y $\gamma$ tienen la misma traza $(\trz{\eta}=\trz{\gamma})$.
		\item $\gamma$ es también reparametrización de $\eta$, pues $\gamma = \eta \circ h^{-1}$.
		\item $\eta$ regular $\iff \gamma$ regular, pues $\eta(u) = h'(u)\cdot\gamma'(h(u))$.
		\item En coordenadas, $\ds \frac{d}{du}\eta(u)=\left(x'(h(u))h'(u), y'(h(u))h'(u), z'(h(u))h'(u)\right)$
	\end{itemize}
\end{defn}
\noindent \textbf{Orientación:}
\begin{itemize}[topsep=0pt, itemsep=1pt,parsep=3pt]
	\item Si $h$ es función creciente, entonces $\eta$ recorre la traza en el mismo sentido que $\gamma$.
	\item Si $h$ es decreciente, entonces $\eta$ recorre la traza en sentido contrario a como lo hace $\gamma$.
\end{itemize}
Supongamos que $h$ es creciente. Sean $\left[p, q\right]\subset J$. Llamamos $h(p)=c<d=h(q)$. La curva $\eta$ recorre de $p$ a $q$ la misma traza que $\gamma$ desde $c$ a $d$. La longitud de esa porción de traza se puede calcular con $\eta$ o con $\gamma$:
\[L(\gamma; c, d)=L(\eta; p, q) \iff \int_c^d\norm{\gamma'(t)}\odif{t}=\int_p^qh'(u)\norm{\gamma'(h(u))}\odif{u}\]

\subsubsection{Parametrización por longitud de arco}
\begin{defn}[Curva parametrizada por longitud de arco]
	Sea $\appl{\eta}{I}{\R^3}$ una curva regular, es parametrizada por longitud de arco
	\[\iff \forall s_1, s_2 \in I : s_1<s_2 : s_2-s_1=L(\eta; s_1, s_2)\]
	\[\iff \forall s_1, s_2 \in I : s_1<s_2 : s_2-s_1=\int_{s_1}^{s_2}\norm{\eta'(d)}\odif{s} \iff \forall s \in I : \norm{\eta'(s)}=1\]
\end{defn}
\begin{teo}
	Toda curva regular se puede reparametrizar por longitud de arco.
	\begin{dem}
		Sea $\appl{\gamma}{I}{\R^3}$ una curva regular, fijamos $t_0\in I$ y defininimos: \[\forall t \in I : g(t)\defeq\int_{t_0}^t\norm{\gamma'(u)}\odif{u}\]
		\[\begin{cases}
			\tex{Para } t >t_0\tex{, se tiene que } g(t) \tex{ es la longitud de la curva } \gamma \tex{ entre } \gamma(t_0) \tex{ y } \gamma(t).\\
			\tex{Para } t <t_0\tex{, la función } g(t) \tex{ nos da el negativo de la longitud entre } \gamma(t_0) \tex{ y } \gamma(t).
		\end{cases}\]
		Queremos usar $s \defeq g(t)$ como etiqueta del punto $\gamma(t)$, en otras palabras, queremos despejar $t$ en términos de $s$, es decir, invertir $g$.
		\begin{itemize}
			\item Como $\gamma$ es regular, se tiene $g'(t)=\norm{\gamma'(t)}>0$. Así que $g$ es función creciente.
			\item Supongamos que $I=(a,b)$. Llamaremos $\ds c\defeq \lim_{t \rightarrow a}g(t)$ y $\ds d\defeq \lim_{t \rightarrow b}g(t)$ 
		\end{itemize}
		\[\implies \tex{La función $g$ es difeomorfismo entre $I=(a,b)$ y $J\defeq(c,d)$}\]
		\begin{lem}
			Sea $\appl{\alpha}{I}{\R^3}$ una función $C^\infty : \forall t\in I : \alpha(t)\ne 0 \implies \norm{\alpha(t)} \tex{es }C^\infty$
			\begin{dem}
				La función $\norm{\alpha(t)}^2=\alpha(t)\cdot\alpha(t)$ es $C^\infty$ y $\norm{\alpha(t)}$ es la raíz cuadrada de una función $C^\infty$ que no se anula, por tanto, es $C^\infty$.
			\end{dem}
		\end{lem}
		El lema nos da que $g'$ es $C^\infty$.
		Tomamos $\appl{h}{J}{I}$ dada por $h(s)\defeq g^{-1}(s)$, y consideramos la curva $\forall s \in J : \eta(s)\defeq\gamma(h(s))$. \\
		Se tiene que $\ds \eta'(s)=h'(s)\cdot\gamma'(h(s))=\frac{1}{g'(h(s))}\gamma'(h(s))=\frac{1}{\norm{\gamma'(h(s))}}\gamma'(h(s))$.
		\[\implies \forall s \in J : \norm{\eta'(s)}=1\implies \eta\tex{ está parametrizada por longitud de arco.}\]
	\end{dem}
\end{teo}
La relevancia del teorema de reparametrización por longitud de arco radica sobre todo en que nos permite suponer que hay tal reparametrización, no tanto en que sea un procedimiento para conseguirla.

\subsection{Curvatura y torsión}
La curvatura de una curva en un punto mide
\begin{itemize}[topsep=1pt, itemsep=1pt,parsep=3pt]
	\item cuán lejana está la curva de ser recta en ese punto.
	\item cuán rápidamente está cambiando de dirección en ese punto.
\end{itemize}
La torsión de una curva en un punto mide
\begin{itemize}[topsep=1pt, itemsep=1pt,parsep=3pt]
	\item cuán próxima está la curva a ser curva plana en ese punto.
	\item cuan rápidamente está cambiando su plano “osculador” en ese punto.
\end{itemize}
\vspace{0.4cm}
\begin{lem}
	\begin{enumerate}[topsep=0pt, itemsep=1pt,parsep=3pt]
		\item[] 
		\item Sea $\appl{u}{I}{\R^3}$ una aplicación derivable tal que $\norm{u(t)}$ es constante en $I$.
		\[\implies \forall t \in I : u'(t)\cdot u(t)=0 \iff \forall t \in I : u'(t) \perp u(t)\]
		\item Sean $\appl{u}{I}{\R^3}, \appl{v}{I}{\R^3}$ aplicaciones derivables tales que $u(t)\cdot v(t)=\tex{cte}$ en $I$.
		\[\implies \forall t \in I : u'(t) \cdot v(t)=-u(t)\cdot v'(t)\]
	\end{enumerate}
	Uso típico: $\norm{u(t)}=1 \we u(t)\cdot v(t)=0$.
	\begin{dem}
		La primera parte del lema se deduce de la segunda, más general, tomando $v=u$.
		Como el producto escalar $u(t)\cdot v(t)$ es constante, derivando se tiene que
		\[\forall t \in I : u'(t)\cdot v(t) + u(t)\cdot v'(t) = 0 \implies \forall t \in I : u'(t)\cdot v(t) = - u(t)\cdot v'(t)\]
	\end{dem}
\end{lem}

\begin{defn}[Vector tangente]
	Sea $\appl{\gamma}{I}{\R^3}$ una curva parametrizada por longitud de arco ($\forall s \in I : \norm{\gamma'(s)}=1$), $\tngnt(s)$ es el vector tangente a $\gamma$ en $s\in I \iff \boxed{\tngnt(s)\defeq \gamma'(s)}$
\end{defn}

\begin{defn}[Curvatura]
	Sea $\appl{\gamma}{I}{\R^3}$ una curva parametrizada por longitud de arco ($\forall s \in I : \norm{\gamma'(s)}=1$), $\kappa(s)$ es la curvatura de $\gamma$ en $s \in I \iff \boxed{\kappa(s) \defeq \norm{\tngnt'(s)}=\norm{\gamma''(s)}}$
\end{defn}

\begin{ejem}[Curvatura nula significa que la traza es (parte de) una recta]
	
\end{ejem}

\begin{prop}
	Una curva en el plano de curvatura constante $M > 0$ está contenida en una circunferencia de radio $\sfrac{1}{M}$
\end{prop}

\begin{defn}[Curva birregular]
	Sea $\appl{\gamma}{I}{\R^3}$ una curva regular, es birregular \[\ds\iff \forall s \in I : \gamma''(s) \ne 0 \iff\forall s \in I : \kappa(s)\ne 0\]
\end{defn}
\begin{prop}
	Una curva regular $\appl{\gamma}{I}{\R^3}$ (no necesariamente parametrizada por longitud de arco) es birregular \[\iff \tex{su reparametrización por longitud de arco es birregular.}\]
	\[\iff \forall i \in I : \gamma'(t)\tex{ y }\gamma''(t) \tex{ son linealmente independientes.}\]
	\begin{dem}
		
	\end{dem}
\end{prop}
\begin{defn}[Vector normal]
	Sea $\appl{\gamma}{I}{\R^3}$ una curva birregular parametrizada por longitud de arco, $\operatorname{n}(s)$ es el vector normal a $\gamma$ en $s$
	\[\iff\operatorname{n}(s)\defeq\frac{\tngnt'(s)}{\norm{\tngnt'(s)}}= \frac{\gamma''(s)}{\norm{\gamma''(s)}}=\frac{\gamma''(s)}{\kappa(s)}\]
	La aplicación $\appl{\operatorname{n}}{I}{\R^3}$ es $C^\infty$, pues $\kappa$ es $C^\infty$ y no se anula. \\
	Como $\norm{\tngnt}=1\implies \tngnt'(s)\perp \tngnt(s) \implies \boxed{\nrml(t) \perp \tngnt(s)}$ \\
	El vector $\nrml(s)$ es unitario y perpendicular a $\tngnt(s)$. Hay infinitos vectores en $\R^3$ (toda una circunferencia) con esa propiedad: $\nrml(s)$ es uno de ellos, pero canónico.
\end{defn}
\begin{defn}[Plano osculador]
	
\end{defn}
\begin{defn}[Vector binormal]
	
\end{defn}
\begin{defn}[Triedro de Frenet]
	
\end{defn}
\begin{defn}[Torsión]
	
\end{defn}
\begin{defn}[Planos]
	
\end{defn}

\subsection{Curvatura y torsión. Complementos}
\subsection{Teorema fundamental de la teoría de curvas}

Este es el teorema fundamental de la teoría de curvas.


\end{document}
