\section{Primera forma fundamental}

\subsection{Primera forma cuadrática fundamental}

\begin{defn}
	Sea $S$ una superficie regular y $p\in S$, la aplicación $I_p$ es la forma cuadrática (cuadrática fundamental) en el plano tangente $T_pS \iff \forall v \in T_pS : I_p(v) = \langle v,v \rangle = \norm{v}^2$.
\end{defn}

\begin{obs}
	$\forall p \in S : I_p$ es una forma cuadrática definida positiva.

	$I_p$ es la forma cuadrática asociada a la forma bilineal simétrica $(u,v) \in T_pS \times T_pS \mapsto \langle u,v \rangle$.
\end{obs}

Sea $\appl{\X}{U\subset \R^2}{\R^3}$ carta de $S$ en $p=\X(u_0, v_0)$. Vamos a expresar la primera forma fundamental $I_p$ de $S$ en $p$ en coordenadas con respecto a la base natural $\beta = \{\X_u(u_0, v_0), \X_v(u_0, v_0)\}$ de $T_pS$ cuyos vectores son las velocidades de las curvas coordenadas en $p$.

Para todo $(u, v) \in U$ definimos:
\[\begin{aligned}
		E(u, v) & \defeq \langle \X_u(u, v), \X_u(u, v) \rangle = \norm{\X_u(u, v)}^2                    \\
		F(u, v) & \defeq \langle \X_u(u, v), \X_v(u, v) \rangle = \langle \X_v(u, v), \X_u(u, v) \rangle \\
		G(u, v) & \defeq \langle \X_v(u, v), \X_v(u, v) \rangle = \norm{\X_v(u, v)}^2
	\end{aligned}\]

Las funciones $E, F, G$ son $C^\infty$ en $U$.
\begin{itemize}
	\item $E(u_0, v_0)$ es el cuadrado de la rapidez de la curva coordenada $\X(u, v_0)$ en $\X(u_0, v_0)$, cuyo vector velocidad es $\X_u(u_0, v_0)$.
	\item $G(u_0, v_0)$ es el cuadrado de la rapidez de la curva coordenada $\X(u_0, v)$ en $\X(u_0, v_0)$, cuyo vector velocidad es $\X_v(u_0, v_0)$.
	\item Si se tiene $F\equiv 0$, entonces las curvas coordenadas se cortan perpendicularmente en cada intersección.
\end{itemize}
Se tiene que $\forall v \in T_pS : \exists a, b \in \R : v = a\X_u(u_0, v_0) + b\X_v(u_0, v_0)$, y entonces:
\[
	\begin{aligned}
		I_p(v) & = \langle v, v \rangle = \langle a\X_u(u_0, v_0) + b\X_v(u_0, v_0), a\X_u(u_0, v_0) + b\X_v(u_0, v_0) \rangle                                               \\
		       & = a^2\langle \X_u(u_0, v_0), \X_u(u_0, v_0) \rangle + 2ab\langle \X_u(u_0, v_0), \X_v(u_0, v_0) \rangle + b^2\langle \X_v(u_0, v_0), \X_v(u_0, v_0) \rangle \\
		       & = a^2E(u_0, v_0) + 2abF(u_0, v_0) + b^2G(u_0, v_0)
	\end{aligned}\]
Por tanto, en forma matricial: $\ds I_p(v) = \begin{pmatrix}
		E(u_0, v_0) & F(u_0, v_0) \\
		F(u_0, v_0) & G(u_0, v_0)
	\end{pmatrix}\begin{pmatrix}a\\b\end{pmatrix} \tex{ con } v = (a, b)_{\beta}$.

Esta matriz simétrica es la matriz de la forma cuadrática $I_p$ respecto de la base $\beta$.
\[I_p \tex{ def. positiva } \implies \det{\begin{pmatrix}
			E(u_0, v_0) & F(u_0, v_0) \\
			F(u_0, v_0) & G(u_0, v_0)
		\end{pmatrix}} = E(u_0, v_0)G(u_0, v_0) - F^2(u_0, v_0) > 0\]

\subsection{Longitudes, ángulos, áreas}

\subsubsection{Longitudes}

Sea $\appl{\alpha}{I}{\R^3}$ una curva y $a, b \in I, a < b$, se tiene que la longitud de la traza de $\alpha$ entre $\alpha(a)$ y $\alpha(b)$ es longitud $\ds = L(\alpha) = \int_{a}^{b} \norm{\alpha'(t)} \odif{t}$.

Si la traza de $\alpha$ está contenida en una superficie regular $S$, de hecho, en $\X(U)$, entonces:
\[\forall t \in I : \alpha(t) = \X(u(t), v(t)) \implies \alpha'(t) = u'(t)\X_u(u(t), v(t)) + v'(t)\X_v(u(t), v(t))\]
\[I_{\alpha(t)}(\alpha'(t)) = \norm{\alpha'(t)}^2 = \left[u'(t)\right]^2E(u(t), v(t)) + 2u'(t)v'(t)F(u(t), v(t)) + \left[v'(t)\right]^2G(u(t), v(t))\]
Por tanto, la longitud de $\alpha$ se escribe, obviando la variable $t$:
\[L(\alpha) = \int_{a}^{b} \sqrt{(u')^2E(u,v) + 2u'v'F(u,v) + (v')^2G(u,v)} \odif{t}\]
Con cierto abuso de notación se escribe $\odif{s}^2 = E \odif{u}^2 + 2F\odif{u}\odif{v} + G\odif{v}^2$ y se le llama primera forma fundamental.

\begin{defn}[Primera forma fundamental]
	Sea $\appl{\alpha}{I}{S}$ una curva regular en una superficie regular $S$ y sea $\appl{\X}{U\subset \R^2}{\R^3}$ una carta de $S$ tal que $\alpha(I) \subset \X(U)$ y $\alpha(t) = \X(u(t), v(t))$.

	La primera forma fundamental de $S$ en $\alpha(t)$ es la forma cuadrática $I_{\alpha(t)}$ evaluada en $\alpha'(t)$ y se denota por $\odif{s}^2 = E\odif{u}^2 + 2F\odif{u}\odif{v} + G\odif{v}^2$.
\end{defn}

\subsubsection{Ángulos}

Sean $\alpha_1$ y $\alpha_2$ dos curvas regulares en una superficie regular $S$ tales que $\alpha_1(t_1) = p = \alpha_2(t_2)$.

Se tiene que el ángulo $\omega$ entre $\alpha_1$ y $\alpha_2$ en $p$ cumple $\ds \cos\omega = \frac{\langle \alpha_1'(t_1), \alpha_2'(t_2) \rangle}{\norm{\alpha_1'(t_1)}\norm{\alpha_2'(t_2)}}$.

Sea $\X(u,v)$ una carta de $S$ en $p$, tenemos $\begin{cases}
		\alpha_1(t) = \X(u_1(t), v_1(t)) \\
		\alpha_2(t) = \X(u_2(t), v_2(t))
	\end{cases} \hspace{-0.5cm}$.
\[\implies \begin{cases}
		\alpha_1'(t) = u_1'(t)\X_u(u_1(t), v_1(t)) + v_1'(t)\X_v(u_1(t), v_1(t)) \\
		\alpha_2'(t) = u_2'(t)\X_u(u_1(t), v_1(t)) + v_2'(t)\X_v(u_1(t), v_1(t))
	\end{cases}\hspace{-0.5cm}\]
\[(u_1(t_1), v_1(t_1)) = (u_0, v_0) = (u_2(t_2), v_2(t_2)) \implies \begin{cases}
		\alpha_1'(t_1) = u_1'(t_1)\X_u(u_0, v_0) + v_1'(t_1)\X_v(u_0, v_0) \\
		\alpha_2'(t_2) = u_2'(t_2)\X_u(u_0, v_0) + v_2'(t_2)\X_v(u_0, v_0)
	\end{cases}\]
\[\implies \langle \alpha_1'(t_1), \alpha_2'(t_2) \rangle = \begin{aligned}
		  & u_1'(t_1)u_2'(t_2)E(u_0, v_0) + (u_1'(t_1)v_2'(t_2) + v_1'(t_1)u_2'(t_2))F(u_0, v_0) \\
		+ & v_1'(t_1)v_2'(t_2)G(u_0, v_0)
	\end{aligned}\]
Por tanto, obviando las variables $t_1, t_2$:
\[\cos\omega = \frac{Eu_1'u_2' + F(u_1'v_2' + v_1'u_2') + Gv_1'v_2'}{\sqrt{(u_1')^2E + 2u_1'v_1'F + (v_1')^2G}\sqrt{(u_2')^2E + 2u_2'v_2'F + (v_2')^2G}}\]
Las funciones $E$, $F$ y $G$ se evalúan en $(u_0, v_0) = (u_1(t_1), v_1(t_1)) = (u_2(t_2), v_2(t_2))$. \\Las derivadas $u_1'$ y $v_1'$ se evalúan en $t_1$, mientras que las derivadas $u_2'$ y $v_2'$ se evalúan en $t_2$.

\subsubsection{Áreas}

Sea $S$ una superficie regular con $\appl{X}{U\subset\R^2}{V\subset S}$ carta de $S$.
\[\implies \tex{Área de }\X(U) = \iint_U \norm{\X_u(u, v)\times \X_v(u, v)} \odif{u} \odif{v}\]
Como se tiene que $\forall v, w \in \R^3 :\norm{v \times w}^2=\norm{v}^2 \norm{w}^2 - {\langle v, w \rangle}^2$, se deduce que
\[\norm{\X_u \times \X_v} = \sqrt{EG -F^2} \implies \tex{Área}(\X(U)) = \iint_U \sqrt{E(u,v)G(u,v) - F^2(u,v)} \odif{u} \odif{v}\]

\subsection{Loxodromas}

\begin{defn}[Loxodroma]
	Sea $\appl{\alpha}{I\subset\R^2}{\R^3}$ una curva contenida en una esfera, $\alpha$ es un loxodroma $\iff$ corta a los meridianos de la esfera con ángulo $\beta$ fijo.
\end{defn}

Si consideramos la parametrización de la esfera $S$ de radio $R$ en coordenadas esféricas dada por $\X(\theta, \varphi) = (R\cos\theta \sin\varphi, R\sin\theta \sin\varphi, R\cos\varphi)$ con $\theta \in (0, 2\pi)$ y $\varphi \in (0, \pi)$, tenemos que la primera forma fundamental viene dada por: $E(\theta, \varphi)=R^2\sin^2\phi$, $F(\theta, \varphi)=0$ y $G(\theta, \varphi)=R^2$.

Fijamos un $\beta \in [0, \sfrac{\pi}{2})$ y un punto $p = \X(\theta_0, \varphi_0)$ en la esfera. Sea $\gamma(t) = \X(\theta_0, t)$ el meridiano que pasa por $p$ y $\omega = \X(\theta(t), \varphi(t))$ una loxodroma que corta a $\gamma$ con ángulo $\beta$ y que $\omega(t_0) = p$.

Dado que $\odv{}{t} \theta_0 = 0 \we \odv{}{t} t = 1 \we \odv{}{t} \theta(t) = \theta'(t) \we \odv{}{t} \varphi(t) = \varphi'(t)$, se tiene que:
\[\implies \langle \omega'(t), \gamma'(t) \rangle =	R^2\sin^2\varphi \cdot 0 \cdot \theta'(t) + 0 \cdot (0\cdot \varphi'(t) + 1\cdot \theta'(t)) + R^2\cdot 1\cdot \varphi'(t) = R^2\varphi'(t)\]
\[\begin{aligned}
		\norm{\varphi'(t)} & = \sqrt{0^2\cdot R^2\sin^2\varphi_0 + 2 \cdot 0 \cdot 1 \cdot 0 + 1^2\cdot R^2} = R                                                                          \\
		\norm{\omega'(t)}  & = \sqrt{(\theta'(t))^2 R^2 \sin^2\varphi_0 + 2\theta'(t)\varphi'(t) \cdot 0 + (\varphi'(t))^2 R^2} = R\sqrt{(\theta'(t))^2\sin^2\varphi_0 + (\varphi'(t))^2}
	\end{aligned}\]
\[\implies \cos\beta = \frac{\cancel{R^2}\varphi'(t)}{\cancel{R^2}\sqrt{(\theta'(t))^2\sin^2\varphi_0 + (\varphi'(t))^2}} = \frac{\varphi'(t)}{\sqrt{(\theta'(t))^2\sin^2\varphi_0 + (\varphi'(t))^2}}\]

Pero queremos que $\omega$ corte a todos los meridianos con el mismo ángulo $\beta$. Observando que, en la fórmula anterior $\varphi_0 = \varphi(t_0)$, concluimos que las funciones $\theta(t)$ y $\varphi(t)$ que determinan la curva $\omega$ deben cumplir la siguiente ecuación direncial:
\[\forall t : \boxed{\cos\beta = \frac{\varphi'(t)}{\sqrt{(\theta'(t))^2\sin^2(\varphi(t)) + (\varphi'(t))^2}}}\]
\[\implies \frac{(\varphi'(t))^2}{\cos^2\beta} = \sin^2(\varphi(t))(\theta'(t))^2 + (\varphi'(t))^2 \implies \varphi'(t)\left(\frac{1}{\cos^2\beta} - 1\right) = \sin^2(\varphi(t))(\theta'(t))^2\]
\[\implies (\varphi'(t))^2\tan^2\beta = \sin^2 (\varphi(t))(\theta'(t))^2 \implies \frac{(\varphi'(t))^2}{\sin^2(\varphi(t))} = \frac{(\theta'(t))^2}{\tan^2\beta}\implies \frac{\varphi'(t)}{\sin(\varphi(t))} = \pm\frac{\theta'(t)}{\tan\beta}\]
\[\implies \boxed{\tan\beta \cdot \ln\left(\tan\left(\frac{\varphi(t)}{2}\right)\right) = \pm \theta(t) + C}\]

\subsection{Isometrías}

\begin{defn}
	Sea $S$ una superficie regular, $S_*$ es una trozo de $S \iff S_* \subset S$ y $S_*$ es una superficie regular.
\end{defn}

\allbold{Nota:} La intersección de un abierto de $\R^3$ con una superficie regular es un trozo de la superficie.

\begin{defn}
	Sean $S$ y $S_*$ dos superficies regulares, $\appl{f}{S}{S_*}$ es una isometría
	\[\iff\begin{cases}
			f \tex{ es un difeomorfismo, i.e. $f$ es suave, biyectiva y con inversa suave} \\
			f \tex{ preserva longitudes }\forall C \subset S \tex{ curva regular} : L(f(C)) = L(C)
		\end{cases}\]
\end{defn}

\begin{teo}
	No existe una isometría entre un trozo de una esfera y un trozo de un plano.
	\begin{dem}
		Asumimos que una isometría preserva también ángulos y áreas. Por contradicción, asumimos que existe una isometría $f$ entre un trozo de una esfera y un trozo de un plano.

		Tomamos un triángulo $T$ de vértices $A, B, C$ dentro del trozo de un plano y su imagen $T'$ de vértices $A', B', C'$ en el trozo de la esfera. Los segmentos que conforman los lados del triángulo son las curvas de longitud mínima entre los vértices. En el caso del plano, rectas, y en el caso de la esfera, circunferencias máximas.

		Esto quiere decir que los lados del triángulo deben transformarse en trozos de circunferencias máximas en la esfera. Como se conservan los ángulos tenemos que
		\[\pi = \hat{A}+\hat{B}+\hat{C} = \hat{A'}+\hat{B'}+\hat{C'} = \pi + \frac{\tex{Área}(T')}{R^2} > \pi \contr\]
		Aquí se ha usado el teorema Girard (la suma de los ángulos de un triángulo en la esfera unidad es $\pi + \tex{Área del triángulo}$).
	\end{dem}
\end{teo}

\begin{lem}
	Sean $S$ y $S_*$ dos superficies regulares y $\appl{f}{S}{S_*}$ una aplicación de tipo $C^\infty$.
	\[f\tex{ preserva longitudes} \iff f \tex{ preserva la primera forma fundamental }\]
	\begin{dem}
		($\Longleftarrow$) Sea $\appl{\X}{U}{S}$ una carta de $S$ tal que $\X(U) = S$ y sea $\appl{\alpha}{I}{S}$ una curva regular en $S$ tal que $\forall t \in (a, b) : \alpha(t) = \X(u(t), v(t))$.

		Como se conserva la primera forma fundamental, se tiene que
		\[\begin{aligned}
				I_{\alpha(t)}(\alpha'(t)) & = \left[u'(t)\right]^2E(u(t), v(t)) + 2u'(t)v'(t)F(u(t), v(t)) + \left[v'(t)\right]^2G(u(t), v(t))       \\
				                          & = \left[u'(t)\right]^2E^*(u(t), v(t)) + 2u'(t)v'(t)F^*(u(t), v(t)) + \left[v'(t)\right]^2G^*(u(t), v(t)) \\
				                          & = I^*_{(f\circ \alpha)(t)}((f\circ\alpha)'(t))
			\end{aligned}\]
		Donde $E^*, F^*, G^*$ componen la primera forma fundamental de $S_*$ con la carta $f\circ\X$.

		Dado que $\norm{\alpha'(t)}^2= I_{\alpha(t)}(\alpha(t))$, se tiene que
		\[L(\alpha) = \int_{a}^{b} \norm{\alpha'(t)} \odif{t} = \int_{a}^{b} \sqrt{I_{\alpha(t)}(\alpha'(t))} \odif{t} = \int_{a}^{b} \sqrt{I^*_{(f\circ \alpha)(t)}((f\circ\alpha)'(t))} \odif{t} = L(f\circ\alpha)\]
		Es decir, $f$ preserva longitudes.

		($\Longrightarrow$) Sea $\appl{\X}{U}{S}$ una blabla... \textbf{TO BE CONTINUED}
	\end{dem}
\end{lem}

