\section{Primera forma fundamental}

\subsection{Primera forma cuadrática fundamental}

\begin{defn}
	Sea $S$ una superficie regular y $p\in S$, la aplicación $I_p$ es la forma cuadrática (cuadrática fundamental) en el plano tangente $T_pS \iff \forall v \in T_pS : I_p(v) = \langle v,v \rangle = \norm{v}^2$.
\end{defn}

\begin{obs}
	$\forall p \in S : I_p$ es una forma cuadrática definida positiva.

	$I_p$ es la forma cuadrática asociada a la forma bilineal simétrica $(u,v) \in T_pS \times T_pS \mapsto \langle u,v \rangle$.
\end{obs}

Sea $\appl{\X}{U\subset \R^2}{\R^3}$ carta de $S$ en $p=\X(u_0, v_0)$. Vamos a expresar la primera forma fundamental $I_p$ de $S$ en $p$ en coordenadas con respecto a la base natural $\beta = \{\X_u(u_0, v_0), \X_v(u_0, v_0)\}$ de $T_pS$ cuyos vectores son las velocidades de las curvas coordenadas en $p$.

Para todo $(u, v) \in U$ definimos:
\[\begin{aligned}
		E(u, v) & \defeq \langle \X_u(u, v), \X_u(u, v) \rangle = \norm{\X_u(u, v)}^2                    \\
		F(u, v) & \defeq \langle \X_u(u, v), \X_v(u, v) \rangle = \langle \X_v(u, v), \X_u(u, v) \rangle \\
		G(u, v) & \defeq \langle \X_v(u, v), \X_v(u, v) \rangle = \norm{\X_v(u, v)}^2
	\end{aligned}\]

Las funciones $E, F, G$ son $C^\infty$ en $U$.
\begin{itemize}
	\item $E(u_0, v_0)$ es el cuadrado de la rapidez de la curva coordenada $\X(u, v_0)$ en $\X(u_0, v_0)$, cuyo vector velocidad es $\X_u(u_0, v_0)$.
	\item $G(u_0, v_0)$ es el cuadrado de la rapidez de la curva coordenada $\X(u_0, v)$ en $\X(u_0, v_0)$, cuyo vector velocidad es $\X_v(u_0, v_0)$.
	\item Si se tiene $F\equiv 0$, entonces las curvas coordenadas se cortan perpendicularmente en cada intersección.
\end{itemize}
Se tiene que $\forall v \in T_pS : \exists a, b \in \R : v = a\X_u(u_0, v_0) + b\X_v(u_0, v_0)$, y entonces:
\[
	\begin{aligned}
		I_p(v) & = \langle v, v \rangle = \langle a\X_u(u_0, v_0) + b\X_v(u_0, v_0), a\X_u(u_0, v_0) + b\X_v(u_0, v_0) \rangle                                               \\
		       & = a^2\langle \X_u(u_0, v_0), \X_u(u_0, v_0) \rangle + 2ab\langle \X_u(u_0, v_0), \X_v(u_0, v_0) \rangle + b^2\langle \X_v(u_0, v_0), \X_v(u_0, v_0) \rangle \\
		       & = a^2E(u_0, v_0) + 2abF(u_0, v_0) + b^2G(u_0, v_0)
	\end{aligned}\]
Por tanto, en forma matricial: $\ds I_p(v) = \begin{pmatrix}
		E(u_0, v_0) & F(u_0, v_0) \\
		F(u_0, v_0) & G(u_0, v_0)
	\end{pmatrix}\begin{pmatrix}a\\b\end{pmatrix} \tex{ con } v = (a, b)_{\beta}$.

Esta matriz simétrica es la matriz de la forma cuadrática $I_p$ respecto de la base $\beta$.
\[I_p \tex{ def. positiva } \implies \det{\begin{pmatrix}
			E(u_0, v_0) & F(u_0, v_0) \\
			F(u_0, v_0) & G(u_0, v_0)
		\end{pmatrix}} = E(u_0, v_0)G(u_0, v_0) - F^2(u_0, v_0) > 0\]

\subsection{Longitudes, ángulos, áreas}

\subsubsection{Longitudes}

Sea $\appl{\alpha}{I}{\R^3}$ una curva y $a, b \in I, a < b$, se tiene que la longitud de la traza de $\alpha$ entre $\alpha(a)$ y $\alpha(b)$ es $\ds L(\alpha) = \int_{a}^{b} \norm{\alpha'(t)} \odif{t}$.