\section{Ejercicios}
\subsection{Hoja 1}

\noindent \textbf{14.*} Dos jugadores de tenis, $A$ y $B$, tienen probabilidades $p$ y $1-p$ de ganar un punto cuando sirve $A$.
\begin{enumerate}
    \item Halla la probabilidad de que A gane un juego en el que sirve y que en este momento está en situación de deuce.
    \item Halla la probabilidad de que A gane un juego en el que sirve.
\end{enumerate}

\noindent \textbf{21. Urna de Pólya}
\[\Omega= \tex{Listas infinitas de $0$s y $1$s} \we B_n=\tex{suceso "blanca" en }n\]
\begin{itemize}%a,b,c,d, ...
    \item $\ds P(B_1)=\frac{b}{a+b}$
    \item $\ds P(B_2)=P(B_2|B_1)P(B_1)+P(B_2|A_1)P(A_1)=P(B_2|B_1)\frac{b}{a+b}+P(B_2|A_1)\frac{a}{a+b}$ \\
    $\ds \implies P(B_2)=\frac{b+1}{a+b+1}\frac{b}{a+b}+\frac{b}{a+b+1}\frac{a}{a+b}=\frac{b(b+1)+ba}{(a+b)(a+b+1)}=\frac{b}{a+b}$
\end{itemize}
\begin{dem}
    Veamos por inducción que $\ds \forall n \in \N :P(B_n)=\frac{b}{a+b}$ \\
    Suponemos que $\ds \forall a,b \in \N: \forall k \in \{1, \dots, n\}:P(B_k)=P(B_1)=\frac{b}{a+b}$
    \[\implies P(B_{n+1})=P(B_{n+1}|B_1)P(B_1)+P(B_{n+1}|A_1)P(A_1)\]
    \[=P(B_{n+1}|B_1)\frac{b}{a+b}+P(B_{n+1}|A_1)\frac{a}{a+b}\]
    Por tanto, por hipótesis de inducción:
    \[\implies P(B_{n+1})=\frac{b+1}{a+b+1}\cdot\frac{b}{a+b} + \frac{b}{a+b+1}\cdot\frac{a}{a+b}=\frac{b}{a+b}\]
\end{dem}