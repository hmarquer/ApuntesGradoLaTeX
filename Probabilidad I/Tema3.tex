\fecha{12/03/2024}
\section{Variables aleatorias continuas}
Hasta ahora en $(\Omega, \F, P)$, una variable aleatoria $X$ discreta era una función $\appl{X}{\Omega}{\R}$ tal que $\exists N \subset \N : \abs{X(\Omega)}=\abs{N}$ y $P(X=k)=P(X^{-1}(k))$.
\begin{defn}[Variable aleatoria]
	Sea $(\Omega, \F, P)$ un espacio de probabilidad, la función $\appl{X}{\Omega}{\R}$ es una variable aleatoria
	\[\iff \forall x \in \R : \{\omega\in\Omega : X(\omega) \leq x\} \in \F\]
\end{defn}
\begin{prop}
	$X$ es v.a.d. $\implies X$ es variable aleatoria.
	\begin{dem}
		Puedo describir el suceso $\{X\leq x\}$ como unión numerable de sucesos
		\[\{\omega \in\Omega : X(\omega) \leq x\} = \bigcup_{y\in X(\Omega)} \{\omega \in \Omega : X(\omega) = y\}\]
		Como la unión numerable de sucesos es un suceso, $\{X\leq x\} \in \F$.
	\end{dem}
\end{prop}
\begin{defn}[Función de distribución]
	Sea $X$ una variable aleatoria, $\appl{F_X}{\R}{[0,1]}$ es su función de distribución $\iff \forall x \in X(\Omega) : \boxed{F_X(x) = P(X\leq x)}$
\end{defn}
Sea $X$ una v.a.d. que toma los valores $x_1, x_2, \cdots$ con probabilidades $p_1, p_2, \dots$ y con función de masa $p_X \ds \implies F_X(x) = \sum_{x_i \leq x} P(X=x_i)$. Es decir, es la función de masa acumulada.
\begin{lem}
	En $(\Omega, \F, P)$, sea $X$ una variable aleatoria con función de distribución $F_X$.
	\[\implies \begin{aligned}
		(1)& \lim_{x\to\infty} F_X(x) = 1 \we \lim_{x\to-\infty} F_X(x) = 0 \\
		(2)& F_X \text{ es no decreciente} \\
		(3)& F_X \text{ es continua por la derecha}
	\end{aligned}\]
	\begin{dem} %TODO: Hacer las demostraciones de (1) y (3) con detalle
		%$(1)$ $A_n = \{\omega \in \Omega : X(\omega) \leq n\}$ que es creciente 
		%\[\implies P\left(\bigcup_{n=1}^\infty A_n\right)=\lim_{n\to\infty} P\left(\bigcup_{j=1}^n A_j\right) = \]
		$(2)$ $x<y \implies \{X\leq x\} \subseteq \{X\leq y\} \implies F_X(x) \leq F_X(y)$.
	\end{dem}
\end{lem}
\begin{teo}
	Sea $\appl{F}{\R}{[0, 1]}$ una función que cumpla (1), (2) y (3) del lema anterior.
	\[\implies \exists! X \text{ variable aleatoria } : F_X = F\]
	\begin{dem} TODO: POR REVISAR
		Defino $X(\omega) = \inf\{x\in\R : F(x) \geq U(\omega)\}$ con $U(\omega) \sim \unif{[0,1]}$.
		\begin{itemize}
			\item $X(\omega) \leq x \iff \inf\{x\in\R : F(x) \geq U(\omega)\} \leq x \iff F(X(\omega)) \geq U(\omega) \iff X(\omega) \in \{x\in\R : F(x) \geq U(\omega)\}$
			\item $X(\omega) \leq x \implies F(X(\omega)) \geq U(\omega) \implies F(x) \geq F(X(\omega)) \geq U(\omega) \implies \{x\in\R : F(x) \geq U(\omega)\} \subseteq \{x\in\R : F(x) \geq F(X(\omega))\}$
		\end{itemize}
		Entonces, $X(\omega) = \inf\{x\in\R : F(x) \geq U(\omega)\} \implies F_X(x) = P(X\leq x) = P(U\leq F(x)) = F(x)$.
	\end{dem}
\end{teo}
\textbf{Moraleja:} Una variable aleatoria queda determinada por su función de distribución.