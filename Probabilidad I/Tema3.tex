\fecha{12/03/2024}
\section{Variables aleatorias continuas}
Hasta ahora en $(\Omega, \F, P)$, una variable aleatoria $X$ discreta era una función $\appl{X}{\Omega}{\R}$ tal que $\exists N \subset \N : \abs{X(\Omega)}=\abs{N}$ y $P(X=k)=P(X^{-1}(k))$.
\begin{defn}[Variable aleatoria]
	Sea $(\Omega, \F, P)$ un espacio de probabilidad, la función $\appl{X}{\Omega}{\R}$ es una variable aleatoria
	\[\iff \forall x \in \R : \{\omega\in\Omega : X(\omega) \leq x\} \in \F\]
\end{defn}
\begin{prop}
	$X$ es v.a.d. $\implies X$ es variable aleatoria.
	\begin{dem}
		Puedo describir el suceso $\{X\leq x\}$ como unión numerable de sucesos
		\[\{\omega \in\Omega : X(\omega) \leq x\} = \bigcup_{y\in X(\Omega)} \{\omega \in \Omega : X(\omega) = y\}\]
		Como la unión numerable de sucesos es un suceso, $\{X\leq x\} \in \F$.
	\end{dem}
\end{prop}
\begin{defn}[Función de distribución]
	Sea $X$ una variable aleatoria, $\appl{F_X}{\R}{[0,1]}$ es su función de distribución $\iff \forall x \in X(\Omega) : \boxed{F_X(x) = P(X\leq x)}$
\end{defn}
Sea $X$ una v.a.d. que toma los valores $x_1, x_2, \cdots$ con probabilidades $p_1, p_2, \dots$ y con función de masa $p_X \ds \implies F_X(x) = \sum_{x_i \leq x} P(X=x_i)$. Es decir, es la función de masa acumulada.
\begin{lem}
	En $(\Omega, \F, P)$, sea $X$ una variable aleatoria con función de distribución $F_X$.
	\[\implies \begin{aligned}
			(1) & \lim_{x\to\infty} F_X(x) = 1 \we \lim_{x\to-\infty} F_X(x) = 0 \\
			(2) & F_X \text{ es no decreciente}                                  \\
			(3) & F_X \text{ es continua por la derecha}
		\end{aligned}\]
	\begin{dem} %TODO: Hacer las demostraciones de (1) y (3) con detalle
		%$(1)$ $A_n = \{\omega \in \Omega : X(\omega) \leq n\}$ que es creciente 
		%\[\implies P\left(\bigcup_{n=1}^\infty A_n\right)=\lim_{n\to\infty} P\left(\bigcup_{j=1}^n A_j\right) = \]
		$(2)$ $x<y \implies \{X\leq x\} \subseteq \{X\leq y\} \implies F_X(x) \leq F_X(y)$.
	\end{dem}
\end{lem}
\begin{teo}
	Sea $\appl{F}{\R}{[0, 1]}$ una función que cumpla (1), (2) y (3) del lema anterior.
	\[\implies \exists! X \text{ variable aleatoria } : F_X = F\]
	\begin{dem} TODO: POR REVISAR
		Defino $X(\omega) = \inf\{x\in\R : F(x) \geq U(\omega)\}$ con $U(\omega) \sim \unif{[0,1]}$.
		\begin{itemize}
			\item $X(\omega) \leq x \iff \inf\{x\in\R : F(x) \geq U(\omega)\} \leq x \iff F(X(\omega)) \geq U(\omega) \iff X(\omega) \in \{x\in\R : F(x) \geq U(\omega)\}$
			\item $X(\omega) \leq x \implies F(X(\omega)) \geq U(\omega) \implies F(x) \geq F(X(\omega)) \geq U(\omega) \implies \{x\in\R : F(x) \geq U(\omega)\} \subseteq \{x\in\R : F(x) \geq F(X(\omega))\}$
		\end{itemize}
		Entonces, $X(\omega) = \inf\{x\in\R : F(x) \geq U(\omega)\} \implies F_X(x) = P(X\leq x) = P(U\leq F(x)) = F(x)$.
	\end{dem}
\end{teo}
\textbf{Moraleja:} Una variable aleatoria queda determinada por su función de distribución.
\fecha{13/03/2024}
Sea $X$ una variable aleatoria con función de distribución $F_X$ y $a,b\in \R$.
\[\implies P(a<X\leq b)=P(\{x\leq b\}\setminus \{x\leq a\}) = P(x\leq b) - P(x\leq a) = F_X(b)-F_X(a)\]
\begin{defn}[Variable aleatoria continua]
	Sea $X$ una variable aleatoria con función de distribución $F_X$, es continua (v.a.c.)
	\[\iff \exists f_X \geq 0 : \left(F_X(x) = P(X\leq x)=\int_{-\infty}^{x} f_X(y) \odif{y}\right) \we \left(\int_{-\infty}^{\infty} f_X(y) \odif{y}=1\right)\]
	$f_X$ se denomina la \textbf{función de densidad} de $X$.
\end{defn}

\begin{obs}
	\begin{enumerate}
		\item[]
		\item $\forall a \in \R : P(X=a) = 0$
		      \begin{dem}
			      Por continuidad de la probabilidad,
			      \[\begin{aligned}
					      \lbox{P(X=a)} & =\lim_{h \to 0^+} P(a-h<X\leq a+h) = \lim_{h\to0^+} F_X(a+h)-F_X(a-h)                                                                                                  \\
					                    & = \lim_{h \to 0^+} \left(\int_{-\infty}^{a} f_X(y) \odif{y} - \int_{-\infty}^{a-h} f_X(y) \odif{y}\right) = \lim_{h\to0^+} \int_{a-h}^{a+h} f_X(y) \odif{y} = \rbox{0}
				      \end{aligned}\]
		      \end{dem}
		\item Cálculo de probabilidades: $\ds\forall a \leq b : P(a\leq X\leq b) = \int_{a}^{b} f_X(y) \odif{y}$
		\item $\ds f_X(x) = \begin{cases}
				      \ds \odv{}{x} F_X(x), & \text{ si } F_X \text{ es derivable en } x \\
				      \ds \quad 0,          & \text{ en otro caso}
			      \end{cases}$
	\end{enumerate}
\end{obs}

\begin{ejem}
	\begin{enumerate}
		\item[]
		\item Para cualquier $f\geq 0$ tal que $\ds \int_{\-infty}^{\infty} f(x) \odif{x} = c \in \R$ tenemos una variable aleatoria continua.
		\item $\ds X\sim\U{0,1} \iff f_X(u)=\begin{cases}
				      1, & \text{ si } u\in[0,1] \\
				      0, & \text{ en otro caso}
			      \end{cases} \implies F_X(u) = \begin{cases}
				      0, & \text{ si } u<0       \\
				      u, & \text{ si } u\in[0,1] \\
				      1, & \text{ si } u>1
			      \end{cases}$
		\item $\ds X\sim\EXP{\lambda} \iff f_X(x) = \begin{cases}
				      \lambda e^{-\lambda x}, & \text{ si } x\geq 0  \\
				      0,                      & \text{ en otro caso}
			      \end{cases} = \lambda e^{-\lambda x} \cdot \mathbbm{1}_{\{x\geq 0\}}(x)$
		      \[\implies \forall x > 0 : F_X(x) = \int_{0}^{x} \lambda e^{-\lambda y} \odif{y} = \left[-e^{-\lambda y}\right]_{0}^{x} = 1-e^{-\lambda x}\]
		\item $\ds X\sim\N{\mu, \sigma^2} \iff f_X(x) = \frac{1}{\sqrt{2\pi}\sigma} e^{-\frac{(x-\mu)^2}{2\sigma^2}}$ \\
		      Pero la guay es $\ds X\sim\normal{0, 1} \iff \phi(x) \defeq f_X(x) = \frac{1}{\sqrt{2\pi}} e^{-\frac{x^2}{2}}$
		      \[\tex{Veamos que } \int_{-\infty}^{\infty} \frac{1}{\sqrt{2\pi}} e^{-\frac{x^2}{2}}\odif{y} = 1\]
		      \begin{dem}
			      \[I = \int_{-\infty}^{\infty} \frac{1}{\sqrt{2\pi}} e^{-\frac{x^2}{2}} \odif{x}\implies I^2 = \left(\int_{-\infty}^{\infty} \frac{1}{\sqrt{2\pi}} e^{-\frac{x^2}{2}}\odif{x}\right)\left(\int_{-\infty}^{\infty} \frac{1}{\sqrt{2\pi}} e^{-\frac{y^2}{2}} \odif{y}\right)\]
			      \[\implies \begin{aligned} I^2 & = \frac{1}{2\pi} \int_{-\infty}^{\infty} \int_{-\infty}^{\infty} e^{-\frac{1}{2}\left(x^2+y^2\right)} \odif{x} \odif{y} = \int_{0}^{2\pi} \int_{0}^{\infty} e^{-\frac{r^2}{2}} r \odif{r} \odif{\theta} \\
                    & =\end{aligned}\]
		      \end{dem}
	\end{enumerate}
\end{ejem}

\fecha{18/03/2024}
\subsection{Funciones / Transformaciones de v.a.c.}
Sea $X$ una v.a.c. con función de densidad $f_X$ y $\appl{g}{X(\Omega)}{\R}$ una función real. Consideramos $Y=g(X)$:
\begin{itemize}
	\item $Y$ es variable aleatoria $\iff \forall y \in \R : \{\omega \in \Omega : g(X(\omega)) \leq y\} \in \F$.\\
	      Esto es cierto para las $g$ ``habituales'' (continuas, monótonas, etc.), para más detalle, hay que esperar a teoría de la medida.
	\item ¡Cuidado! $Y$ puede no ser continua.
	\item Si $Y$ es v.a. $\implies Y$ tiene función de distribución
	      \[F_Y(y) = P(Y\leq y) = P(g(X)\leq y) = ???\]
	      \begin{ejem}
		      $g(x) = ax + b$ con $a\ne 0 \we b \in \R$.
		      \[\implies F_Y(y)= P(aX+b\leq y) = \begin{cases}
				      P\left(X \leq \frac{y-b}{a}\right) = F_X\left(\frac{y-b}{a}\right),       & \text{ si } a>0 \\
				      P\left(X \geq \frac{y-b}{a}\right) = 1 - P\left(X < \frac{y-b}{a}\right), & \text{ si } a<0
			      \end{cases}\]
		      Para $X$ v.a. continua, $\ds F_Y(y) = \begin{cases}
				      F_X\left(\frac{y-b}{a}\right),  & \text{ si } a>0 \\
				      1-F_X\left(\frac{y-b}{a}\right) & \text{ si } a<0
			      \end{cases}$.
		      \[\implies f_Y(y) =\odv{}{y} F_Y(y) = \begin{cases}
				      \frac{1}{a} f_X\left(\frac{y-b}{a}\right),  & \text{ si } a>0 \\
				      -\frac{1}{a} f_X\left(\frac{y-b}{a}\right), & \text{ si } a<0
			      \end{cases} = \frac{1}{\abs{a}} f_X\left(\frac{y-b}{a}\right)\]
	      \end{ejem}
\end{itemize}

\begin{teo}
	En $(\Omega, \F, P)$, sea $X$ una v.a.c. con función de densidad $f_X$ y $\appl{g}{\R}{\R}$ una función diferenciable y estrictamente creciente.
	\[\implies f_Y(y)=f_X\big(g^{-1}(y)\big) \frac{1}{g'\big(g^{-1}(y)\big)}\]
	De manera similar, si $g$ es estrictamente decreciente$\ds\implies f_Y(y)= - f_X\big(g^{-1}(y)\big) \frac{1}{g'\big(g^{-1}(y)\big)}$
	\begin{dem}
		\[\begin{aligned}
				F_Y(y) & = P(Y\leq y) = P(g(X)\leq y) = P(X\leq g^{-1}(y)) = F_X\big(g^{-1}(y)\big)                                                                                  \\
				f_Y(y) & = \odv{}{y} F_Y(y) = \odv{}{y} F_X\big(g^{-1}(y)\big) = f_X\big(g^{-1}(y)\big) \odv{}{y} g^{-1}(y) = f_X\big(g^{-1}(y)\big) \frac{1}{g'\big(g^{-1}(y)\big)}
			\end{aligned}\]
		En el caso de $g$ decreciente tenemos:
		\[\begin{aligned}
				F_Y(y) & = P(g(X)\leq y) = P(X\geq g^{-1}(y)) = 1 - P(X\leq g^{-1}(y)) = 1 - F_X\big(g^{-1}(y)\big)                                                  \\
				f_Y(y) & = -\odv{}{y} F_X\big(g^{-1}(y)\big) = -f_X\big(g^{-1}(y)\big) \odv{}{y} g^{-1}(y) = -f_X\big(g^{-1}(y)\big) \frac{1}{g'\big(g^{-1}(y)\big)}
			\end{aligned}\]
	\end{dem}
\end{teo}

\fecha{20/03/2024}
\begin{ejer}
	Sea $X$ una variable aleatoria con función de densidad $f_X(x)$ y $Y=g(X)$ con $g(x)=x^2$. ¿Cuál es la función de densidad de $Y$?
	\[F_Y(y) = P(X^2\leq y)=\begin{cases}
			y<0 \implies 0 \\
			y\geq 0 \implies P\left(-\sqrt{y}\leq X\leq \sqrt{y}\right) = F_X\left(\sqrt{y}\right)-F_X\left(-\sqrt{y}\right)
		\end{cases}\]
	\[\implies f_Y(y)=\begin{cases}
			y \leq 0 \implies 0 \\
			y > 0 \implies \ds \odv{}{y} F_X\left(\sqrt{y}\right)-\odv{}{y} F_X\left(-\sqrt{y}\right) = \frac{1}{2\sqrt{y}} f_X\left(\sqrt{y}\right) + \frac{1}{2\sqrt{y}} f_X\left(-\sqrt{y}\right)
		\end{cases}\]
\end{ejer}

\begin{ejem}
	Sea $\ds X\sim\normal{\mu, \sigma^2} \implies \forall x \in \R: f_X(x)=\frac{1}{\sqrt{2\pi}\sigma} e^{-\frac{(x-\mu)^2}{2\sigma^2}}$ e $Y=e^X$ (se denomina lognormal). Calculamos la derivada y la inversa de $g(x)=e^x$.
	\[\implies g'(x)=e^x \quad \we \quad  \odv{}{y} g^{-1} (y) = \ln{y} \quad \we \quad  \odv{}{y} g^{-1} (y) = \frac{1}{y}\]
	Como $g$ es estrictamente creciente, aplicamos el teorema anterior:
	\[f_Y(y) = \begin{cases}
			0,                                                                                 & \text{ si } y\leq 0 \\
			\ds \frac{1}{\sqrt{2\pi}\sigma} \frac{1}{y} e^{-\frac{(\ln{y}-\mu)^2}{2\sigma^2}}, & \text{ si } y>0     %TODO: gráfica
		\end{cases}\]
\end{ejem}

\begin{ejem}
	Sea $X\sim\EXP{\lambda}, \lambda > 0$ y $Y=3X+2$. Calculamos la función de densidad de $Y$.
	\[g'(x) = 3 \quad \we \quad \odv{}{y} g^{-1}(y) = \odv{}{y}\left(\frac{y-2}{3}\right) = \frac{1}{3}\]
	Como $g$ es estrictamente creciente, aplicamos el teorema anterior:
	\[f_Y(y) = f_X\left(\frac{y-2}{3}\right) \cdot \frac{1}{3}= \begin{cases}
			0,                                                 & \text{ si } y<2     \\
			\ds \frac{1}{3}\lambda e^{-\lambda \frac{y-2}{3}}, & \text{ si } y\geq 2
		\end{cases}\]
\end{ejem}

\subsection{Esperanzas de v.a.c.}
\begin{defn}[Esperanza]
	Sea $X$ una v.a.c. con función de densidad $f_X$. $E(X)$ es la esperanza de $X$
	\[\iff E(X) = \int_{-\infty}^{\infty} x \cdot f_X(x) \odif{x}\]
	Siempre que haya convergencia absoluta $\ds \iff \int_{-\infty}^{\infty} \abs{x} \cdot f_X(x) \odif{x} < \infty$.
\end{defn}

\begin{teo}
	Sea $X$ una v.a.c. con función de densidad $f_X$ y $\appl{g}{\R}{\R}$ una función diferenciable y estrictamente creciente.
	\[\implies E(g(X)) = \int_{-\infty}^{\infty} g(x) \cdot f_X(x) \odif{x}\]
	\begin{dem}
		\[\begin{aligned} %TODO: Por terminar
				E(g(X)) & = \int_{-\infty}^{\infty} y \cdot f_Y(y) \odif{y} = \int_{-\infty}^{\infty} y \cdot f_X\big(g^{-1}(y)\big) \frac{1}{g'\big(g^{-1}(y)\big)} \odif{y} = ??? \\
				        & = \int_{-\infty}^{\infty} g(x) \cdot f_X(x) \odif{x}
			\end{aligned}\]
	\end{dem}
\end{teo}

\begin{ejem}
\begin{enumerate}
\item[]
\item Sea $X\sim\unif{a,b}$, entonces $f_X(x)=\frac{1}{b-a}\cdot \mathbbm{1}_{[a,b]}(x)$.
\[\implies E(X) = \int_{a}^{b} x \cdot \frac{1}{b-a} \odif{x} = \frac{1}{b-a} \left[\frac{x^2}{2}\right]_{a}^{b} = \frac{b^2-a^2}{2(b-a)} = \frac{a+b}{2}\]
\[V(X) = E(X^2) - E(X)^2 = \int_{a}^{b} x^2 \cdot \frac{1}{b-a} \odif{x} - \left(\frac{a+b}{2}\right)^2 = \frac{1}{b-a} \left[\frac{x^3}{3}\right]_{a}^{b} - \left(\frac{a+b}{2}\right)^2 = \frac{b^3-a^3}{3(b-a)} - \left(\frac{a+b}{2}\right)^2 = \frac{(b-a)^2}{12}\]
\item Sea $X\sim\EXP{\lambda}, \lambda > 0$, entonces $f_X(x)=\lambda e^{-\lambda x} \cdot \mathbbm{1}_{[0,\infty)}(x)$.
\[\implies E(X)= \int_{0}^{\infty} x \cdot \lambda e^{-\lambda x} \odif{x} = \left[-x e^{-\lambda x}\right]_{0}^{\infty} + \int_{0}^{\infty} e^{-\lambda x} \odif{x} = 0 + \left[-\frac{1}{\lambda} e^{-\lambda x}\right]_{0}^{\infty} = \frac{1}{\lambda}\]
\[V(X) = E(X^2) - E(X)^2 = \int_{0}^{\infty} x^2 \cdot \lambda e^{-\lambda x} \odif{x} - \left(\frac{1}{\lambda}\right)^2 = \frac{1}{\lambda^2}\]
\end{enumerate}
\end{ejem}