\section{Curvatura de superficies}

Vamos a buscar máximos y mínimos de $z = f(x,y)$. Si $\nabla f = 0$, podemos estar ante un máximo, mínimo, punto de silla o punto singular. Para estudiar la curvatura, miro la curva que contenga al seccionar la superficie por cada plano perpendicular al tangente.

(Ojo: la curvatura de una de esas curvas no tiene por qué dar información sobre la curvatura de la superficie).

\begin{itemize}
	\item Queremos estudiar todas las curvaturas a la vez e interesa ver si son positivas (sentido de $\vec{n}$) o negativas al coger siempre la misma normal.
	\item En general, vamos a rotar la superficie para obtener una gráfica con plano tangente horizontal. Ahí se calcula la curvatura escalar proque se ha convertido un punto $p$ cualquiera en un crítico. ¡Los giros no afectan a la curvatura!
\end{itemize}