\section{Tema 1: Curvas}

\subsection{Curvas regulares, trazas, velocidades y tangencias}
\begin{defn}[Curva]
	$\gamma$ es una curva $\iff \appl{\gamma}{I=(a,b)\subseteq\R}{\R^3}$ es $C^{\infty}$
	\begin{itemize}
		\item Para casi todo lo que sigue basta con que $\gamma \in C^2$. Para contrastar, en alguna discusión posterior, admitiremos curvas que son solo $C^0 \ve C^1$.
		\item El intervalo $I = (a,b)$ puede ser no acotado en $\R$.
	\end{itemize}
\end{defn}
\begin{defn}[Traza]
	Sea $\gamma$ una curva, $\trz{\gamma}=\{\gamma(t):t\in I\}\subset \R^3$. También se dice que $\gamma$ es parametrización de $\trz{\gamma}$.
\end{defn}

\begin{obs}
	Dos curvas distintas pueden tener la misma traza y corresponden a parametrizaciones diferentes. De hecho, si $\appl{\gamma}{I}{\R^3}$ es una curva y $\appl{g}{J}{I}$ es una función sobreyectiva $C^{\infty}$
	\[\implies \appl{\mu}{J}{\R^3} : \mu(u)=\gamma\left(g(u)\right)\tex{ es una curva de misma traza que } \gamma\]
\end{obs}
\begin{defn}[Velocidad y rapidez]
	Sea $\appl{\gamma}{I}{\R^3}$ una curva, el vector velocidad (velocidad) de $\gamma$ en $t$ es $\gamma'(t)=\nabla\gamma(t)\in \R^3$. La norma $\norm{\gamma'(t)}$ se conoce como la rapidez.
\end{defn}
\begin{defn}[Curva regular]
	Sea $\appl{\gamma}{I}{\R^3}$, es regular $\iff \forall t \in I : \gamma'(t)\ne 0$
\end{defn}
\begin{ejem}
	El conjunto de puntos $V=\{(x, y)\in \R^2:-1<x<1\we y=|x|\}$ no es traza de ninguna curva regular.\\
	\indent Pongamos que $\gamma(t)=(x(t), y(t))$ es curva regular con traza $V$, y que $\gamma(0)=(0,0)$. Entonces $y(t)$ tiene un mínimo en $t = 0$ y $y'(0) = 0$. Si fuera $x'(0) > 0$, entonces $x(t) > 0$ para $t \in (0, \delta)$ y, por tanto, $x(t) =y(t)$ para $t \in \left[0, \delta\right)$ y, por tanto $x'(0) = y'(0) = 0$. Contradicción: la curva $\gamma$ no es regular. También conduce a contradicción el que $x'(0) = 0$.
\end{ejem}
\begin{defn}[Recta tangente a una curva regular]
	Sea $\appl{\gamma}{I}{\R^3}$ una curva regular y $t_0\in I$, $\appl{\eta}{\R}{\R^3}$ es la recta tangente a $\gamma$ en $\gamma(t) \iff \eta(u)=\gamma(t_0)+u\gamma'(t_0)$
\end{defn}

\subsection{Longitud de curvas y reparametrizaciones}
\subsubsection{Poligonales}
Sea $\appl{\gamma}{I}{\R^3}$ una curva (no necesariamente regular) y $\left[c, d\right] \subset I$. Definimos una poligonal de $\gamma$ entre $\gamma(c)$ y $\gamma(d)$ tomando una partición de $\left[c, d\right]$:
\[c=t_0<t_1<\cdots<t_{n-1}<t_n=d\]
La poligonal $\Pi$ asociada a esa partición es la lista de segmentos:
\[\Pi=\left(\left[\gamma(t_0), \gamma(t_1)\right], \left[\gamma(t_1), \gamma(t_2)\right], \dots, \left[\gamma(t_{n-1}), \gamma(t_n)\right]\right)\]
La longitud de la poligonal $\Pi$ se denota $L(\Pi)$ se define entonces de forma natural como:
\[L(\Pi)=\sum_{j=1}^n \norm{\gamma(t_j)-\gamma(t_{j-1})}\]

\subsubsection{Definición de longitud de curva}
\begin{defn}[Longitud de una curva]
	Sea $\appl{\gamma}{I}{\R^3}$ una curva $C^0$, $\left[c,d\right] \subset  I$, se define la longitud como: \[\ds L(\gamma; c, d)=\sup \left\{L(\Pi) : \Pi\tex{ poligonal de } \gamma \tex{ entre }\gamma(c)\tex{ y }\gamma(d)\right\}\]
	En general, para una curva sólo continua este supremo puede ser $\infty$. Por ejemplo, para la curva copo de nieve de Von Koch.
\end{defn}

\subsubsection{Teorema de cálculo de longitud}
\begin{teo}
	Sea $\appl{\gamma}{I}{\R^3}$ una curva regular y $\left[c,d\right] \subset  I$
	\[\implies L(\gamma; c, d) = \int_c^d\norm{\gamma'(t)}\odif{t}\]
	\begin{dem}
		En la demostración denotamos
		\begin{itemize}
			\item $\ds L \defeq L(\gamma; c, d)$, que es supremo de las longitudes de las poligonales.
			\item $\ds J \defeq \int_c^d\norm{\gamma'(t)}\odif{t}$
		\end{itemize}
		Veamos que $\ds L\leq J \we J \leq L$:
		\begin{lem} \label{lem1}
			Sea $\appl{u}{I}{\R^3}$ una aplicación y sean $c, d \in I$, con $c<d$.
			\[\implies\norm{\int_c^du(t)\odif{t}} \leq \int_c^d\norm{u(t)}\odif{t}\]
		\end{lem}
		\begin{dem}
			Usamos que para cualquier $v$ se tiene $\ds\norm{v}=\sup\{|v\cdot w|:\norm{w}=1\}$. Pongamos $\ds v=\int_c^du(t)\odif{t}$ y sea $w$ con $\norm{w}=1$
			\[\implies v\cdot w=\int_c^d\left(u(t)\cdot w\right)\odif{t} \we \abs{v\cdot w}\leq\int_c^d\abs{u(t)\cdot w}  \odif{t} \leq  \int_c^d\norm{u(t)} \odif{t}\]
			Lo anterior es cierto para cualquier $w$ con $\norm{w}=1$.
		\end{dem}
		$\mathbf{\left[L\leq J\right]}$ Para una poligonal $\Pi$ con partición $t_0=c<t_1<\cdots<t_n=d$, se tiene que
		\[L(\Pi)=\sum_{j=1}^n\norm{\gamma(t_j)-\gamma(t_{j-1})} = \sum_{j=1}^n\norm{\int_{t_{j- 1}}^{t_j}\gamma'(t)\odif{t}}\]
		Por el Lema \ref{lem1}, se tiene \[\ds L(\Pi) \leq \sum_{j=1}^n\int_{t_{j-1}}^{t_j}\norm{\gamma'(t)}\odif{t} =   \int_{t_0}^{t_n}\norm{\gamma'(t)}\odif{t} = J \implies L \leq J\]
		$\mathbf{\left[L\leq J\right]}$ Fijemos $\varepsilon > 0$. Por la continuidad uniforme de $\gamma'\in C^0$, tenemos  $\delta>0$ tal que si $c\leq t\leq t'\leq d$ \\
		\centerline{\begin{itemize*}[itemjoin=\hspace{1cm}]
			\item son tales que $|t'-t|\leq \delta$
			\item entonces $\norm{\gamma(t')-\gamma(t)}\leq \varepsilon$
		\end{itemize*}}
		(El $\delta$ no depende de los puntos $t$ y $t'$, sólo de $\varepsilon$). \\
		Tomamos una partición $\ds t_0=c<t_1<\cdots <t_n=d$ para la que cumpla que $\ds t_j-t_{j-1}<\delta$ para $\ds j=1, \dots, n$. \\
		Sea $\Pi$ la poligonal dada por esa partición. Entonces:
		\[L(\Pi)=\sum_{j=0}^{n-1}\norm{\gamma(t_{j+1})-\gamma(t_j)}=\sum_{j=0}^{n-1}\norm{\int_{t_{j-1}}^{t_j}\gamma'(t)\odif{t}}\]
		Sea $\ds u_j\in \left[t_{j-1}, t_j\right]$ tal que $\ds\norm{\gamma'(u_j)}=\max_{t\in \left[t_{j-1}, t_j\right]}\norm{\gamma'(t)}$. \\
		Por un lado, $\ds J=\int_c^d\norm{\gamma'(t)}\odif{t} = \sum_{j=1}^n \int_{t_{j-1}}^{t_j}\norm{\gamma'(t)}\odif{t} \leq \sum_{j=1}^n\norm{\gamma'(u_j)}(t_j-t_{j-1}) \hspace{0.5cm} (\star)$ \\
		Por otro lado, \[\ds \gamma(t_{j+1})-\gamma(t_j) = \int_{t_j}^{t_{j+1}}\gamma'(t)\odif{t}=\left(\int_{t_j}^{t_{j+1}}(\gamma'(t)-\gamma'(u_j))\odif{t}\right)+\gamma'(u_j)(t_{j+1}-t_j)\]
		Así que \begin{equation*}
			\begin{split}
				\norm{\gamma(t_{j+1})-\gamma(t_j)} & \geq  \norm{\gamma'(u_j)}|t_{j+1}-t_j|-\norm{\int_{t_j}^{t_{j+1}}(\gamma'(t)-\gamma'(u_j))\odif{t}} \\ & \geq  \norm{\gamma'(u_j)}|t_{j+1}-t_j|-\varepsilon(t_{j+1}-t_j)
			\end{split}
		\end{equation*}
		usando el Lema \ref{lem1} y las condiciones de la partición.
		Por lo tanto, \begin{equation*}\begin{split}
			L(\Pi) &=\sum_{j=0}^{n-1}\norm{\gamma(t_{j+1}-\gamma(t_j))}
			\geq \sum_{j=0}^{n-1}\norm{\gamma'(u_j)}(t_{j+1}-t_j)-\varepsilon\sum_{j=0}^{n-1}(t_{j+1}-t_j) \geq J -\varepsilon(d-c)
		\end{split}\end{equation*}
		donde se ha usado $(\star)$. En suma
		\[\forall \varepsilon > 0 : J\leq L(\Pi) + \varepsilon(d-c)\geq L + \varepsilon(d-c) \implies J \leq L\]
	\end{dem}
\end{teo}
\begin{ejem}[Longitud gráfica de una función] 
	$\ds \forall t \in \left[c, d\right] : \gamma(t)=(t, f(t))$ donde $\appl{f}{\left[c, d\right]}{\R}$ es una función deribable.
	\[\norm{\gamma'(t)}=\sqrt{1+(f'(t))^2} \implies \tex{Long gráfica }=\int_c^d \sqrt{1+(f'(t))^2} \odif{t}\]
\end{ejem}



\subsubsection{Reparametrizaciones}

Dada una curva $\appl{\gamma}{I}{\R^3}$, queremos recorrer su traza con otra curva.
\begin{defn}
	Sean $\appl{h}{J}{I}$ un difeomorfismo (biyección tal que tanto $h$ como $h^{-1}$ son $C^\infty$) y $\appl{\gamma}{I}{\R^3}$ una curva, $\appl{\eta}{J}{\R^3}$ es una reparametrización de $\gamma$
	\[ \iff \forall u \in J : \eta(u)=\gamma(h(u))\]
	\begin{itemize}[topsep=1pt, itemsep=1pt,parsep=3pt]
		\item $\eta$ es curva porque $\gamma$ lo es y $h$ es $C^\infty$.
		\item $\eta$ y $\gamma$ tienen la misma traza $(\trz{\eta}=\trz{\gamma})$.
		\item $\gamma$ es también reparametrización de $\eta$, pues $\gamma = \eta \circ h^{-1}$.
		\item $\eta$ regular $\iff \gamma$ regular, pues $\eta(u) = h'(u)\cdot\gamma'(h(u))$.
		\item En coordenadas, $\ds \frac{d}{du}\eta(u)=\left(x'(h(u))h'(u), y'(h(u))h'(u), z'(h(u))h'(u)\right)$
	\end{itemize}
\end{defn}
\noindent \textbf{Orientación:}
\begin{itemize}[topsep=0pt, itemsep=1pt,parsep=3pt]
	\item Si $h$ es función creciente, entonces $\eta$ recorre la traza en el mismo sentido que $\gamma$.
	\item Si $h$ es decreciente, entonces $\eta$ recorre la traza en sentido contrario a como lo hace $\gamma$.
\end{itemize}
Supongamos que $h$ es creciente. Sean $\left[p, q\right]\subset J$. Llamamos $h(p)=c<d=h(q)$. La curva $\eta$ recorre de $p$ a $q$ la misma traza que $\gamma$ desde $c$ a $d$. La longitud de esa porción de traza se puede calcular con $\eta$ o con $\gamma$:
\[L(\gamma; c, d)=L(\eta; p, q) \iff \int_c^d\norm{\gamma'(t)}\odif{t}=\int_p^qh'(u)\norm{\gamma'(h(u))}\odif{u}\]

\subsubsection{Parametrización por longitud de arco}
\begin{defn}[Curva parametrizada por longitud de arco]
	Sea $\appl{\eta}{I}{\R^3}$ una curva regular, es parametrizada por longitud de arco
	\[\iff \forall s_1, s_2 \in I : s_1<s_2 : s_2-s_1=L(\eta; s_1, s_2)\]
	\[\iff \forall s_1, s_2 \in I : s_1<s_2 : s_2-s_1=\int_{s_1}^{s_2}\norm{\eta'(d)}\odif{s} \iff \forall s \in I : \norm{\eta'(s)}=1\]
\end{defn}
\begin{teo}
	Toda curva regular se puede reparametrizar por longitud de arco.
	\begin{dem}
		Sea $\appl{\gamma}{I}{\R^3}$ una curva regular, fijamos $t_0\in I$ y defininimos: \[\forall t \in I : g(t)\defeq\int_{t_0}^t\norm{\gamma'(u)}\odif{u}\]
		\[\begin{cases}
			\tex{Para } t >t_0\tex{, se tiene que } g(t) \tex{ es la longitud de la curva } \gamma \tex{ entre } \gamma(t_0) \tex{ y } \gamma(t).\\
			\tex{Para } t <t_0\tex{, la función } g(t) \tex{ nos da el negativo de la longitud entre } \gamma(t_0) \tex{ y } \gamma(t).
		\end{cases}\]
		Queremos usar $s \defeq g(t)$ como etiqueta del punto $\gamma(t)$, en otras palabras, queremos despejar $t$ en términos de $s$, es decir, invertir $g$.
		\begin{itemize}
			\item Como $\gamma$ es regular, se tiene $g'(t)=\norm{\gamma'(t)}>0$. Así que $g$ es función creciente.
			\item Supongamos que $I=(a,b)$. Llamaremos $\ds c\defeq \lim_{t \rightarrow a}g(t)$ y $\ds d\defeq \lim_{t \rightarrow b}g(t)$ 
		\end{itemize}
		\[\implies \tex{La función $g$ es difeomorfismo entre $I=(a,b)$ y $J\defeq(c,d)$}\]
		\begin{lem}
			Sea $\appl{\alpha}{I}{\R^3}$ una función $C^\infty : \forall t\in I : \alpha(t)\ne 0 \implies \norm{\alpha(t)} \tex{es }C^\infty$
			\begin{dem}
				La función $\norm{\alpha(t)}^2=\alpha(t)\cdot\alpha(t)$ es $C^\infty$ y $\norm{\alpha(t)}$ es la raíz cuadrada de una función $C^\infty$ que no se anula, por tanto, es $C^\infty$.
			\end{dem}
		\end{lem}
		El lema nos da que $g'$ es $C^\infty$.
		Tomamos $\appl{h}{J}{I}$ dada por $h(s)\defeq g^{-1}(s)$, y consideramos la curva $\forall s \in J : \eta(s)\defeq\gamma(h(s))$. \\
		Se tiene que $\ds \eta'(s)=h'(s)\cdot\gamma'(h(s))=\frac{1}{g'(h(s))}\gamma'(h(s))=\frac{1}{\norm{\gamma'(h(s))}}\gamma'(h(s))$.
		\[\implies \forall s \in J : \norm{\eta'(s)}=1\implies \eta\tex{ está parametrizada por longitud de arco.}\]
	\end{dem}
\end{teo}
La relevancia del teorema de reparametrización por longitud de arco radica sobre todo en que nos permite suponer que hay tal reparametrización, no tanto en que sea un procedimiento para conseguirla.

\subsection{Curvatura y torsión}
La curvatura de una curva en un punto mide
\begin{itemize}[topsep=1pt, itemsep=1pt,parsep=3pt]
	\item cuán lejana está la curva de ser recta en ese punto.
	\item cuán rápidamente está cambiando de dirección en ese punto.
\end{itemize}
La torsión de una curva en un punto mide
\begin{itemize}[topsep=1pt, itemsep=1pt,parsep=3pt]
	\item cuán próxima está la curva a ser curva plana en ese punto.
	\item cuan rápidamente está cambiando su plano “osculador” en ese punto.
\end{itemize}
\begin{lem}
	\begin{enumerate}[topsep=0pt, itemsep=1pt,parsep=3pt]
		\item[] 
		\item Sea $\appl{u}{I}{\R^3}$ una aplicación derivable tal que $\norm{u(t)}$ es constante en $I$.
		\[\implies \forall t \in I : u'(t)\cdot u(t)=0 \iff \forall t \in I : u'(t) \perp u(t)\]
		\item Sean $\appl{u}{I}{\R^3}, \appl{v}{I}{\R^3}$ aplicaciones derivables tales que $u(t)\cdot v(t)=\tex{cte}$ en $I$.
		\[\implies \forall t \in I : u'(t) \cdot v(t)=-u(t)\cdot v'(t)\]
	\end{enumerate}
	Uso típico: $\norm{u(t)}=1 \we u(t)\cdot v(t)=0$.
	\begin{dem}
		La primera parte del lema se deduce de la segunda, más general, tomando $v=u$.
		Como el producto escalar $u(t)\cdot v(t)$ es constante, derivando se tiene que
		\[\forall t \in I : u'(t)\cdot v(t) + u(t)\cdot v'(t) = 0 \implies \forall t \in I : u'(t)\cdot v(t) = - u(t)\cdot v'(t)\]
	\end{dem}
\end{lem}
\begin{defn}[Vector tangente]
	Sea $\appl{\gamma}{I}{\R^3}$ una curva parametrizada por longitud de arco ($\forall s \in I : \norm{\gamma'(s)}=1$), $\tngnt(s)$ es el vector tangente a $\gamma$ en $s\in I \iff \boxed{\tngnt(s)\defeq \gamma'(s)}$
\end{defn}
\begin{defn}[Curvatura] \label{defnCurvatura}
	Sea $\appl{\gamma}{I}{\R^3}$ una curva parametrizada por longitud de arco ($\forall s \in I : \norm{\gamma'(s)}=1$), $\kappa(s)$ es la curvatura de $\gamma$ en $s \in I \iff \boxed{\kappa(s) \defeq \norm{\tngnt'(s)}=\norm{\gamma''(s)}}$
\end{defn}

\begin{ejem}[Curvatura nula significa que la traza es (parte de) una recta]
	
\end{ejem}

\begin{prop}
	Una curva en el plano de curvatura constante $M > 0$ está contenida en una circunferencia de radio $\sfrac{1}{M}$
\end{prop}

\begin{defn}[Curva birregular]
	Sea $\appl{\gamma}{I}{\R^3}$ una curva regular, es birregular \[\ds\iff \forall s \in I : \gamma''(s) \ne 0 \iff\forall s \in I : \kappa(s)\ne 0\]
\end{defn}

\begin{prop}
	Una curva regular $\appl{\gamma}{I}{\R^3}$ (no necesariamente parametrizada por longitud de arco) es birregular \[\iff \tex{su reparametrización por longitud de arco es birregular.}\]
	\[\iff \forall i \in I : \gamma'(t)\tex{ y }\gamma''(t) \tex{ son linealmente independientes.}\]
	\begin{dem}
		
	\end{dem}
\end{prop}
\begin{defn}[Vector normal] \label{defnNormal}
	Sea $\appl{\gamma}{I}{\R^3}$ una curva birregular parametrizada por longitud de arco, $\nrml(s)$ es el vector normal a $\gamma$ en $s\in I$
	\[\iff\nrml(s)\defeq\frac{\tngnt'(s)}{\norm{\tngnt'(s)}}= \frac{\gamma''(s)}{\norm{\gamma''(s)}}=\frac{\gamma''(s)}{\kappa(s)}\]
	La aplicación $\appl{\operatorname{n}}{I}{\R^3}$ es $C^\infty$, pues $\kappa$ es $C^\infty$ y no se anula. \\
	Como $\norm{\tngnt}=1\implies \tngnt'(s)\perp \tngnt(s) \implies \boxed{\nrml(t) \perp \tngnt(s)}$ \\
	El vector $\nrml(s)$ es unitario y perpendicular a $\tngnt(s)$. Hay infinitos vectores en $\R^3$ (toda una circunferencia) con esa propiedad: $\nrml(s)$ es uno de ellos, pero canónico.
\end{defn}
\begin{defn}[Plano osculador]
	Sea $\appl{\gamma}{I}{\R^3}$ una curva parametrizada por longitud de arco, el subespacio afín $\pi$ es el plano osculador a la curva $\gamma$ en $s \in I$
	\[\iff \pi = \mathcal{L}\left\{\tngnt(s)=\gamma'(s), \nrml(s)=\frac{\gamma''(s)}{\kappa(s)}\right\} + \gamma(s) \]
	Para una curva en $\R^2$, el plano osculador es $\R^2$.
\end{defn}
Sea $\appl{\gamma}{I}{\R^3}$ una curva birregular parametrizada por longitud de arco y supongamos $0 \in I$. \\
La aproximación de primer orden a $\gamma$ en $\gamma(0)$ es la recta tangente a $\gamma$ en $\gamma(0)$:
\[\gamma(s)\approx\gamma(0) + s\gamma'(0)=\gamma(0)+s\tngnt()\]
La aproximación de segundo orden a $\gamma$ en $\gamma(0)$ se denomina \textbf{parábola osculatriz}:
\[\gamma(s)\approx\gamma(0) + s\tngnt(0) + \frac{1}{2}s^2\gamma''(0)=\gamma(0) + s\tngnt(0) + \frac{\kappa(0)}{2}s^2\nrml(0)\]
Entonces, ``hasta orden $2$'' la curva está contenida en su plano osculador.
\begin{defn}[Vector binormal]
	Sea $\appl{\gamma}{I}{\R^3}$ una curva birregular, $\bnrml(s)$ es el vector binormal a $\gamma$ en $s\in I \iff \boxed{\bnrml(s)\defeq\tngnt(s)\times\nrml(s)}$ \\
	\[\tex{Por las propiedades del producto vectorial } \implies \bnrml(s) \perp \tngnt(s) \we \bnrml(s) \perp \nrml(s) \we \norm{\bnrml(s)}=1\]
	Entonces, el vector binormal es vector unitario normal al plano osculador a la curva en $s$.
\end{defn}
\begin{defn}[Triedro de Frenet]
	Sea $\appl{\gamma}{I}{\R^3}$ una curva birregular, el triedro de Frenet en $s\in I$ es la tripleta $\left(\tngnt(s), \nrml(s), \bnrml(s)\right)$
	\begin{itemize}
		\item $\forall s \in I : \left(\tngnt(s), \nrml(s), \bnrml(s)\right)$ es un triedro positivamente orientado.
		\item $\forall s \in I : \{\tngnt(s), \nrml(s), \bnrml(s)\}$ es una base ortonormal de $\R^3$.
	\end{itemize}
\end{defn}
La torsión de $\gamma$ en $s$ va a medir cómo varía el plano osculador a $\gamma$ en $s$ y, alternativamente, cuán ``plana'' es la curva $\gamma$ en $s$.

Como $\bnrml(s)$ es el vector normal al plano osculador a $\gamma$ en $s$, se trata de estudiar $\bnrml'(s)$. Sabiendo que $\bnrml(s)=\tngnt(s)\times\nrml(s)$, al derivar obtenemos:
\[\begin{aligned}
	\bnrml'(s)&=\tngnt'(s)\times\nrml(s)+\tngnt(s)\times\nrml'(s) \\
	&= \kappa(s)\nrml(s)\times\nrml(s)+\tngnt(s)\times\nrml'(s) = \tngnt(s)\times\nrml'(s)
\end{aligned}\]
Se deduce que $\bnrml'(s) \perp \tngnt(s)$. Además, como $\bnrml(s)\cdot\bnrml(s)\equiv1 \implies \bnrml'(s)\perp\bnrml(s) \implies \bnrml'(s) \parallel \nrml(s)$.

\begin{defn}[Torsión] \label{defnTorsion}
	Sea $\appl{\gamma}{I}{\R^3}$ una curva birregular, $\tau(s)$ es la torsión de $\gamma$ en $\ds s\in I \iff \boxed{\bnrml'(s)=\tau(s)\nrml(s)} \implies \boxed{\tau(s)=\bnrml'(s)\cdot\nrml(s)} \we \norm{\bnrml(s)}=\abs{\tau(s)}$
	
	Para una curva birregular la curvatura es siempre positiva, pero la torsión en un punto dado puede ser positiva, negativa o 0. El signo de la torsión tiene significado geométrico.
\end{defn}
\begin{prop}[Torsión $\equiv 0$]
	Sea $\gamma$ una curva birregular,
	\[\gamma \tex{ está contenida en un plano} \iff \tau \equiv 0\]
	\begin{dem}
		$(\implies)$ Si la curva está contenida en un plano, entonces $\bnrml$ es constantemente uno de los dos vectores normales a dicho plano, por lo que $\bnrml' \equiv 0 \implies \tau \equiv 0$.

		$(\impliedby)$ Si $\tau \equiv 0$, entonces $\bnrml' \equiv 0 \implies \bnrml$ es constante. Por tanto, si fijamos $s_0\in I \implies \forall s \in I : b(s)=b(s_0)$. Queremos probar ahora que $\forall s \in I : (\gamma(s)- \gamma(s_0)) \perp b(s_0)$. \\
		Pero derivando el producto escalar,
		\[\begin{aligned}
			((\gamma(s)-\gamma(s_0))\cdot\bnrml(s_0))' &= \gamma'(s)\cdot\bnrml(s_0) \\
			&= \tngnt(s)\cdot\bnrml(s_0) =  \tngnt(s)\cdot\bnrml(s) = 0, \quad \forall s \in I
		\end{aligned}\]
		Por tanto, $(\gamma(s)-\gamma(s_0))\cdot\bnrml(s_0)$ es una constante cuando $s$ recorre $I$. Como en $s=s_0$ se anula, se tiene que $\forall s \in I : (\gamma(s)-\gamma(s_0))\perp\bnrml(s_0)$ y $\gamma$ está contenida en un plano.
	\end{dem}
\end{prop}
\begin{defn}[Planos]
	Sean $\appl{\gamma}{I}{\R^3}$ una curva parametrizada por longitud de arco y $(\tngnt, \nrml, \bnrml)$ su triedro de Frenet. Se definen
	\[\begin{aligned}
		\tex{Plano osculador, } & \tex{determinado por} & \tngnt \we \nrml \\
		\tex{Plano normal, } & \tex{determinado por} & \nrml \we \bnrml \\
		\tex{Plano rectificante, } & \tex{determinado por} & \tngnt \we \bnrml \\
	\end{aligned}\]
\end{defn}
\subsubsection{Fórmulas de Frenet-Serret}
Vamos a expresar las derivadas $\tngnt'(s), \nrml'(s), \bnrml'(s)$ respecto de la base ortonormal $\beta = \{\tngnt(s), \nrml(s), \bnrml(s)\}$.

Ya tenemos $\tngnt'(s) = \kappa(s)\nrml(s) \we \bnrml'(s)=\tau(s)\nrml(s)$ de las definiciones \ref{defnCurvatura}, \ref{defnTorsion} y \ref{defnNormal}.
\[\bnrml(s)=\tngnt(s)\times\nrml(s) \implies \nrml(s) = \bnrml(s)\times\tngnt(s)\]
\[\implies \nrml'(s)=\bnrml'(s)\times\tngnt(s) + \bnrml(s)\times\tngnt'(s) = \tau(s)\nrml(s)\times\tngnt(s) + \kappa(s)\bnrml(s)\times\nrml(s)\]
\[\implies \nrml'(s)=-\tau(s)\bnrml(s)-\kappa(s)\tngnt(s)\]
\[\implies \begin{pmatrix}
	\tngnt'(s) \\ \nrml'(s) \\ \bnrml'(s)
\end{pmatrix} = \underbrace{\begin{pmatrix}
	0 & \kappa(s) & 0 \\
	-\kappa(s) & 0 & -\tau(s) \\
	0 & \tau(s) & 0
\end{pmatrix}}_{\eqdef FS}\begin{pmatrix}
	\tngnt(s) \\ \nrml(s) \\ \bnrml(s)
\end{pmatrix}\]
Estas son las ecuaciones de Frenet-Serret donde la matriz escalar $FS$ es antisimétrica.

Si escribimos los vectores de arriba en coordenadas $(\tngnt=(t_1, t_2, t_3), \tngnt'=(t_1', t_2', t_3'), \tex{ etc.})$, la matriz anterior es de dimensiones $9\times9$: cada símbolo en la matriz $FS$ se ha de sustituir por una caja diagonal.

\subsubsection{Cambio de sentido y Triedro de Frenet}
Sea $\appl{\gamma}{I}{\R^3}$ una curva birregular parametrizada por longitud de arco. Supongamos por comodidad que $I = (-a, a) \implies 0\in I$ para $a>0$.

Consideramos la curva $\bar{\gamma}(s)\defeq\gamma(-s)$, que recorre la misma traza que $\gamma$, pero en sentido contrario. Queremos relacionar el tiedro de $\bar{\gamma}$ y su curvatura $\bar{\kappa}$ y torsión $\bar{\tau}$ con los de $\gamma$ en $0$, como $\bar{\gamma} (0)=\gamma(0)$:
\[\bar{\tngnt}(0)=-\tngnt(0) \we \bar{\nrml}(0)=\nrml(0) \we \bar{\kappa}(0)=\kappa(0) \we \bar{\bnrml}(0)=\bnrml(0) \we \bar{\tau}(0)=-\tau(0)\] 

\subsubsection{Movimientos rígidos y triedros}
Los movimientos rígidos de $\R^3$ se obtienen componiendo traslaciones con rotaciones. Si $M$ es movimiento rígido de $\R^3$, entonces $M$ viene dado por un vector $p$ y una matriz ortogonal $O$ $\left(O^T=O^{-1}\right)$ con determinante $1$:
\[\forall v \in \R^3 : M(v)=Ov+p\]
Como $|O|=1$, se tiene que $u, v \in \R^3 \implies Ou\times Ov=O(u\times v)$.
\begin{enumerate}
	\item Traslaciones \\
	Sea $\gamma$ una curva birregular parametrizada por longitud de arco y $\bar{\gamma}$ la curva $\bar{\gamma}(s)\defeq\gamma(s)+p$.

	Como $\bar{\gamma}'(s)=\gamma'(s)$, se tiene que $\bar{\gamma}$ está parametrizada por longitud de arco y que $\forall s \in I : \bar{\tngnt}(s) = \tngnt(s) \implies \forall s \in I : \bar{\kappa}(s)=\kappa(s) \we \bar{\nrml}(s)=\nrml(s) \we \bar{\tau}(s)=\tau(s) \we \bar{\bnrml}(s)=\bnrml(s)$
	\item Rotaciones \\
	Consideremos ahora la curva $\bar{\gamma}(s)=O\gamma(s)$.

	Se tiene $\forall s \in I : \bar{\gamma}'(s)=O\gamma'(s)$. Así que $\bar{\gamma}$ está parametrizada por longitud de arco y $\forall s \in I : \bar{\tngnt}(s)=O\tngnt(s)$.

	Además, $\forall s \in I : \bar{\tngnt}'(s)=O\tngnt'(s) \implies \bar{\kappa}(s)=\kappa(s) \we \bar{\nrml}(s)=O\nrml(s)$
	\[\bar{\bnrml}(s)=\bar{\tngnt}(s)\times\bar{\nrml}(s) = O\tngnt(s)\times O\nrml(s)=O\left(\tngnt(s)\times \nrml(s)\right)=O\bnrml(s) \implies \bar{\tau}(s)=\tau(s)\]
\end{enumerate}
Es decir, al trasladar o rotar la curva, sin deformarla, las longitudes se conservan: los triedros van en triedros y sus variaciones $(\kappa, \tau)$ se conservan.

\subsubsection{Otros triedros sobre curvas}
Sea $\appl{\gamma}{I}{\R^3}$ una curva parametrizada por longitud de arco. Sea $\left[P(s), Q(s), R(s)\right]$ un triedro positivo sobre la curva $\gamma$.
\[\iff \begin{cases}
	\appl{P, Q, R}{I}{\R^3} \quad C^\infty \\
	\forall s \in I : \norm{P(s)}=\norm{Q(s)}=\norm{R(s)}=1 \\
	\forall s \in I : P(s) \perp Q(s) \perp R(s) \we P(s) \perp P(s)
\end{cases}\]
Para cada $s \in I$ tenemos una base ortonormal $[P(s), Q(s), R(s)]$ de $\R^3$ positivamente orientada, i.e. $\forall s \in I : R(s)=P(s)\times Q(s)$.

Vamos a expresar la variación de $[P(s), Q(s), R(s)]$ respecto de sí mismo.
\[\begin{pmatrix}
	P'(s) \\ Q'(s) \\ P'(s)
\end{pmatrix} = \begin{pmatrix}
	\star & \star & \star \\
	\star & \star & \star \\
	\star & \star & \star \\
\end{pmatrix}\begin{pmatrix}
	P(s) \\ Q(s) \\ R(s)
\end{pmatrix}\]
Como $\norm{P(s)}\equiv1$, tenemos que $P'(s)\cdot P(s)=0$, análogamente para $Q$ y $R$. Así que la matriz tiene ceros en la diagonal.

Por otro lado, como $P(s)\cdot Q(s)=0 \implies P'(s)\cdot Q(s)=-Q'(s)\cdot P(s)$ y análogamente para $P(s)\cdot R(s)=0$ y $Q(s)\cdot R(s)=0$. Así que la matriz es antisimétrica.
\[\implies \begin{pmatrix}
	P'(s) \\ Q'(s) \\ P'(s)
\end{pmatrix} = \begin{pmatrix}
	0 & A(s) & B(s) \\
	-A(s) & 0 & C(s) \\
	-B(s) & -C(s) & 0 \\
\end{pmatrix}\begin{pmatrix}
	P(s) \\ Q(s) \\ R(s)
\end{pmatrix}\]

\subsection{Curvatura y torsión. Complementos}
\subsubsection{Curvas no parametrizadas por longitud de arco}
Sea $\appl{\gamma}{I}{\R}$ una curva con parámetro natural $t$ pero que no es longitud de arco. Queremos obtener la curvatura, torsión y el triedro de Frenet de $\gamma(t)$.

La parametrización $\eta$ por longitud de arco viene dada por $\forall t \in I : \gamma(t) = \eta(s(t))$ donde $s=s(t)$ es la longitud de arco de $\gamma$ desde $\gamma(t_0)$ hasta $\gamma(t)$.
\begin{center}
	\includegraphics[width=8cm]{img/longarco}	
\end{center}
Cuando $t$ recorre el intervalo $I$, el parámetro $s$ recorre un intervalo $J$. En notación general de (re)parametrización, la función $t \to s(t)$ se denota por $g(t)$.

Como $\tngnt, \nrml, \bnrml, \kappa, \tau$ se expresan y calculan a través de derivadas de $\eta$ y queremos hacerlo en términos de $\gamma$, aplicamos la regla de la cadena.

\[\forall t \in I : s(t) = \int_{t_0}^t \norm{\odv{\gamma}{t}(u)}\odif{u} \implies \forall t \in I : \odv{s}{t}(t)=\norm{\odv{\gamma}{t}(t)}\]
\[\odv{\gamma}{t}(t)=\odv{\eta}{s}\cdot\odv{s}{t}(t)=\odv{s}{t}\cdot\tngnt(s)\]
\[\implies \odv[2]{\gamma}{t} (t) = \odv{s}{t}\cdot\odv{\tngnt}{s}(s) + \odv[2]{s}{t}\cdot\tngnt(s) = \kappa(s) {\left(\odv{s}{t}\right)}^2\nrml(s) + \odv[2]{s}{t}\cdot\tngnt(s)\]
\[\begin{aligned}
	\implies \odv[3]{\gamma}{t}(t) &= \kappa(s){\left(\odv{s}{t}\right)}^3\odv{n}{s}(s) &+ \tex{ combinación lineal de $\tngnt$ y $\nrml$} \\
	&= - \kappa(s){\left(\odv{s}{t}\right)}^3\tau(s)\bnrml(s) &+ \tex{ combinación lineal de $\tngnt$ y $\nrml$}
\end{aligned}\]

\textbf{Fórmula general para la curvatura}
\[\left(\odv{\gamma}{t}\right)\times\left(\odv[2]{\gamma}{t}\right)=\kappa(s)\left(\odv{s}{t}\right)\bnrml(s) \implies \boxed{\kappa(s)=\frac{\gamma'\times \gamma''}{\norm{\gamma'}^3}}\]
Para $\gamma$ parametrizada por longitud de arco $\implies \norm{\gamma'}=1 \we \gamma'\perp\gamma'' \implies \kappa(s)=\norm{\gamma''(s)}$.

\textbf{Fórmula general para la torsión}
\[\begin{aligned}
	\left(\odv{\gamma}{t}\times\odv[2]{\gamma}{t}\right) \cdot \odv[3]{\gamma}{t} &= \left(\kappa(s)\left(\odv{s}{t}\right)^3\bnrml(s)\right)\cdot\left(-\kappa(s)\left(\odv{s}{t}\right)^3\tau(s)\bnrml(s) + \tex{ combinacnión lineal de $\tngnt$ y $\nrml$}\right) \\
	&= - \kappa^2\left(\odv{s}{t}\right)^6\tau=-\tau\norm{\odv{\gamma}{t}\times\odv[2]{\gamma}{t}}^2
\end{aligned}\]
\[\implies \boxed{\tau(s)=-\frac{(\gamma'\times\gamma'')\cdot\gamma'''}{\norm{\gamma'\times\gamma''}^2}}\]
En cuanto a $\tngnt$, $\nrml$, $\bnrml$:
\[\boxed{t=\frac{\gamma'}{\norm{\gamma'}}} \we \boxed{\bnrml=\frac{\gamma'\times\gamma''}{\norm{\gamma'\times\gamma''}}} \we \boxed{\nrml=\bnrml\times\tngnt=\frac{\left(\gamma'\times\gamma''\right)\times\gamma'}{\norm{\gamma'\times\gamma''}}\norm{\gamma'}}\]

\subsubsection{Curvatura de curvas planas}

Para curvas planas: podemos darle signo a la curvatura, y darle a ese signo un significado geométrico.
\begin{defn}
	Sea $u \in \R^2$ un vector unitario $(\norm{u}=1)$, $u^\perp$ es el vector que se obtiene al girarlo $\sfrac{\pi}{2}$ en sentido positivo (antihorario) $$\iff u=(\cos{\theta}, \sin{\theta}) \implies u^\perp=(-\sin{\theta}, \cos{\theta})$$
\end{defn}

Sea $\gamma$ una curva regular en $\R^2$ parametrizada por longitud de arco. \\
Denotamos con $\hat{\nrml}(s)\defeq{\tngnt(s)}^\perp$.
\[\implies \appl{\hat{\nrml}}{I}{\R^2} \quad C^\infty \we \forall s \in I : \left(\tngnt(s) \perp \hat{\nrml}(s)\right) \we \left(\nrml(s)=\pm \hat{\nrml}(s)\right)\]
Como $\tngnt'(s) \perp \tngnt(s)$, tenemos que $\tngnt'(s) \parallel \hat{\nrml}(s) \implies \tngnt'(s) = \hat{\kappa}(s) \cdot \hat{\nrml}(s)$. Donde $\hat{\kappa}(s)$ es la curvatura con signo de $\gamma$ en $s$.
La curvatura con signo solo requiere que la curva sea regular.
\[\implies \boxed{\hat{\kappa}(s)=\tngnt'(S)\cdot\hat{\nrml}(s)} \we \abs{\hat{\kappa}(s)}=\kappa(s)\]
El correspondiente vector binormal sería $\hat{\bnrml}=\tngnt\times\hat{\nrml}=\tngnt\times\tngnt^\perp=(0, 0, 1) \perp \R^2$.
\begin{center}
	\includegraphics[width=10cm]{img/curvaturaR2}	
\end{center}
Para una gráfica en el plano $t\mapsto (t, f(t))$ que se recorre de izquierda a derecha ($t$ creciente) la convexidad $(f'' > 0)$ corresponde con $\hat{\kappa} > 0$ y la concavidad corresponde con $\hat{\kappa} < 0$.

\textbf{Relación entre curvatura y ángulo con eje $OX$}

Vamos a describir la relación entre la curvatura $\hat{\kappa}$ y (la variación de) el ángulo que (el vector tangente de) la curva con el eje $OX$.

\begin{lem}
	Sea $\appl{I}{\R^2}$ una aplicación $C^\infty : \forall t\in I : \norm{u(t)} = 1$.
	\[\implies \exists\, \appl{\theta}{I}{\R^2} \quad C^\infty : \forall t \in I : u(t)=(\cos{\theta(t)}, \sin{\theta(t)})\]
	La función $\theta$ es única salvo por adición de un múltiplo entero de $2\pi$. Por tanto, la derivada $\theta'$ está unívocamente determinada.
\end{lem}
Dada una curva $\gamma$ regular plana y parametrizada por longitud de arco, podemos escribir $\tngnt(s)=(\cos{\theta(t)}, \sin{\theta(t)})$ con $\theta$ función $C^\infty$. Así que, $\hat{\nrml}(s)=(-\sin{\theta(t)}, \cos{\theta(t)})$ donde $\theta(s)$ es el ángulo que forma $\tngnt(s)$ con el eje $OX$. El ángulo $\theta(s)$ determina la dirección de $\tngnt(s)$.
\[\implies \tngnt'(s)=\theta'(s)(-\sin{\theta(s)}, \cos{\theta(s)})=\theta'(s) \hat{\nrml}(s) \implies \forall s : \hat{\kappa}(s) = \theta'(s)\]

\textbf{Reconstrucción de curva plana a partir de $\hat{\kappa}$}

Sea $\appl{\gamma}{I}{\R^2}$ una curva plana parametrizada por longitud de arco y $M$ un movimiento rígido del plano que conserva orientación, es decir, una traslación compuesta con una rotación.

Consideremos la curva $\bar{\gamma}$ dada por $\forall s \in I : \bar{\gamma}(s)=M\gamma(s)$. Como $M$ es movimiento rígido, se tiene que $\bar{\gamma}(s)$ está parametrizada por longitud de arco y $\forall s \in I : \hat{\bar{\kappa}}(s)=\hat{\kappa}(s)$.

Ahora, como $\hat{\kappa}(s)=\theta'(s)$ y partimos de $\theta(0)=0$:
\[\implies \theta(s)=\int_{0}^{s}\theta'(u)\odif{u} = \int_0^s\hat{\kappa}(u) \odif{u}\] 
Como $\gamma'(s)=\tngnt(s)=(\cos{\theta(s)}, \sin{\theta(s)})$, si establecemos $\gamma(0)=(0, 0)$
\[\implies\gamma(s)=\left(\int_0^s\cos{\theta(u)} \odif{u},\int_0^s\sin{\theta(u)} \odif{u}\right) \, \implies \, \boxed{\hat{\kappa} \Rrightarrow \theta' \Rrightarrow \theta \Rrightarrow \tngnt \Rrightarrow \gamma}\]
En general si conocemos la función $\ds \hat{\kappa}(s) \implies \theta(s) = \int \hat{\kappa}(s)\odif{s} + c_1$ para cierta constante $c_1$.
\[\implies \gamma(s)=\left(\int \cos{\theta(s)} \odif{s}, \int \sin{\theta(s)}\odif{s}\right) + (c_2, c_3) \tex{ para $c_2, c_3$ constantes.}\]
Los valores de $c_1, c_2, c_3$ quedan determinados, por ejemplo, por los valores de $\gamma$ y $\gamma'$ es un cierto $s_0\in I$.

\subsubsection{Forma canónica local}

Sea $\appl{\gamma}{I}{\R^3}$ una curva birregular parametrizada por longitud de arco.
Analizamos localmente, mediante aproximación de Taylor de tercer orden, la curva $\gamma$ en un entorno de $0 \in I$ centrada en $0$.
\[\gamma(s)=\gamma(0) + s\gamma'(0) + \frac{s^2}{2} \gamma''(0) + \frac{s^3}{6}\gamma'''(0) + E(s), \tex{ con }\lim_{s\to0}\frac{\norm{E(s)}}{s^3} = 0\]
\[\gamma(s)=\gamma(0) + \left(s-\frac{\kappa(0)^2s^3}{6}\right)\tngnt(0) + \left(\frac{\kappa(0)s^2}{2} + \frac{\kappa'(0)s^3}{6}\right)\nrml(0) + \left(-\frac{\kappa(0)\tau(0)s^3}{6}\right)\bnrml(0) + E(s)\]
Tras un movimiento rígido o con cambio de sistema de referencia, suponemos que
\[\gamma(0)=0 \we \tngnt(0)=i \we \nrml(0)=j \we \bnrml(0)=k\]
Si $\gamma(s)=(x(s), y(s), z(s))$ y si $E(s)=(e_x(s), e_y(s), e_z(s))$, entonces
\[\begin{cases}
	\ds x(s) &= s - \frac{\kappa(0)^2s^3}{6} + e_x(s) \\
	\ds y(s) &= \frac{\kappa(0)s^2}{2} + \frac{\kappa'(0)s^3}{6} + e_y(s) \\
	\ds z(s) &= -\frac{\kappa(0)\tau(0)s^3}{6} + e_z(s)
\end{cases}\]
Esta es la forma canónica local de $\gamma$ cerca de $s=0$.

La torsión $\tau$ solo aparece en $z(s)$:
\begin{itemize}
	\item Si $\tau > 0$, para $s>0$ la curva $\gamma$ tiende a meterse debajo del plano osculador a $\gamma$ en $s=0$, donde ``debajo'' $=$ en dirección opuesta a $\bnrml$.
	\item Si $\tau < 0$, para $s>0$ la curva $\gamma$ tiende a pasar por encima del plano osculador a $\gamma$ en $s=0$, donde ``encima'' $=$ en la dirección de $\bnrml$.
\end{itemize}

\textbf{Forma canónica local. Curvas planas}

Sea $\appl{\gamma}{I}{\R^2}$ una curva parametrizada por longitud de arco con $0\in I$.
\[\implies \gamma(s)=\gamma(0)+s\tngnt(0)+\frac{s^2}{2}\hat{\kappa}(0)\hat{\nrml}(0) + F(s), \tex{ con } \lim_{s\to 0}\frac{\norm{F(s)}}{s^2} = 0\]
Con $\gamma(s)=(x(s), y(s)) \we \gamma(0)=(0,0) \we \tngnt(0)=(1, 0) \we \nrml(0)=(0,1)$ se tiene:
\[\begin{cases}
	x(s)=s-\frac{\kappa(0)s^3}{6}+o(s^3) \\
	y(s)=\frac{\kappa(0)}{2}s^2 + o(s^2)
\end{cases}\]

\textbf{Circunferencia osculatriz}

\begin{itemize}
	\item Centro de curvatura $\gamma(0) + \frac{1}{\kappa(0)}\nrml(0)$
	\item Radio de curvatura $\frac{1}{\kappa(0)}$
\end{itemize}
La circunferencia osculatriz aproxima a la curva $\gamma$ cerca de $s=0$:
\[\norm{\gamma(s)-\left(\gamma(0) + \frac{1}{\kappa}\nrml(0)\right)}=\frac{1}{\kappa(0)} + o(s^2)\]

\textbf{Curvatura y comparación arco/cuerda en el plano}
\[D(s)=\norm{\gamma(s_0+s) - \gamma(s_0)}\]
\[\implies \begin{cases}
	\ds D(s) \leq \tex{longitud de $\gamma$ entre $s_0$ y $s_0+s$} = s \\
	\ds \lim_{s\to 0} \frac{D(s)}{s} = \norm{\gamma'(s_0)} = 1
\end{cases}\]
Poniendo $s_0=0$, y $\gamma(0)=(0,0) \we \tngnt(0)=(1, 0) \we \nrml(0)=(0, 1)$ se tiene
\[\begin{aligned} D(s)^2 = x(s)^2+y(s)^2 &= s^2-2\frac{\kappa^2}{6}s^4 + \frac{\kappa^2}{4}s^4 + o(s^4) \\
&= s^2\left(1-\frac{1}{12}\kappa^2s^2+o(s^2)\right)\end{aligned}\]
Así que $\ds \frac{D(s)^2}{s^2} = 1-\frac{1}{12}\kappa^2s^2+o(s^2)$. Cuando más grande es $\kappa$, más pequeño es $\ds \frac{D(s)}{s}$

\subsection{Teorema fundamental de la teoría de curvas}

\subsection{Desigualdad isoperimétrica. Geometría global de curvas planas}