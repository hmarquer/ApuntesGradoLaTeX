\section{EDOs autónomas}

\begin{defn}
	Una ecuación diferencial ordinaria (EDO) de primer orden se dice autónoma si no depende explícitamente de la variable independiente. Es decir, 
	\[\iff \tex{es de la forma } y'=f(y)\]
\end{defn}

\begin{prop}[Propiedades de EDOs autónomas]
	\begin{enumerate}
		\item[]
		\item (Isoclinas) Todos los puntos de cada recta horizontal $y=c$ pertenecen a la misma isoclina. ¡Cuidado! A veces una isoclina puede contener más de una recta horizontal.
		\begin{ejem}[$y'=y^2$]
			\[\{(x,y)\in \R:y^2=c\}=\{(x,\sqrt{c}):x\in\R\}\cup\{(x,-\sqrt{c}):x\in\R\}\]
		\end{ejem}
		\item (Traslaciones) Si y es solución $\implies \forall c \in \R : w(x)\defeq y(x+c)$ es solución.
		\item (Soluciones triviales) Si $\exists a \in \operatorname{Dom}(f) : f(a)=0 \implies y(x)=a$ es solución.
		\begin{dem}
			$y'(x)=0=f(a)=f(y(x))$
		\end{dem}
	\end{enumerate}
\end{prop}
% \begin{teo}
%     Sean $a\in [-\infty, \infty) \we b\in (-\infty, \infty] \we \appl{f}{(a,b)}{\R}$ continua $ \we y_0 \in (a,b) : f(y_0) \ne 0$
%     \[\implies \exists \varepsilon : (y_0-\varepsilon, y_0+\varepsilon) \subset (a,b) \we \forall z \in (y_0-\varepsilon, y_0+\varepsilon) : f(z)\ne 0\]
%     Sea $\appl{y}{I}{(a,b)}$ una función derivable con $I\subset \R$ intervalo abierto tal que
%     $$\forall x \in I : y'(x)=f(y(x)) \we y(x_0)=y_0 \tex{ para algún }x_0 \in I$$.
%     Como $y$ es continua, $\exists \delta > 0 : \forall x \in (x_0-\delta, x_0+\delta) :y(x)\in(y_0-\varepsilon, y_0+\varepsilon)$
%     \[\implies \forall x \in (x_0-\delta, x_0+\delta) : f(y(x))\ne 0\]
%     Así, $\ds \forall x \in (x_0-\delta, x_0+\delta) : \frac{y'(x)}{f(y(x))}=1 \implies \forall x \in (x_0-\delta, x_0+\delta) : \int_{x_0}^{x} \frac{y'(z)}{f(y(z))}\odif{z} = x-x_0$
%     \[\implies \boxed{\forall x \in (x_0-\delta, x_0+\delta) : \int_{y_0}^{y(x)} \frac{1}{f(z)} \odif{z} + x_0 = x}\]
% \end{teo}

\fecha{22/02/2024}

\begin{teo}[Existencia de soluciones]
	Sean $a\in [-\infty, \infty) \we b\in (-\infty, \infty] \we \appl{f}{(a,b)}{\R}$ continua \\
	Supongamos que $\forall x \in (a,b) : f(x) \ne 0$ y que $\begin{cases}
		a>-\infty \implies f(a)=0 \\
		b<\infty \implies f(b)=0
	\end{cases}$ \\
	Sea $x_0 \in (a,b)$ definimos $\ds \forall x \in (a,b) : F(x)\defeq\int_{x_0}^{x} \frac{1}{f(s)} \odif{s}$ \\
	Si $f(x)>0$ en $(a,b)$, definimos 
	$$\ds T_-\defeq \lim_{x\to a^+}F(x)\in [-\infty, 0) \we T_+\defeq \lim_{x\to b^-}F(x) \in (0, \infty]$$
	Si $f(x) < 0$ en $(a,b)$, intercambiamos $T_-$ por $T_+$.
	\[\implies \exists \appl{x}{(T_-, T_+)}{(a,b)} \tex{ derivable}: \begin{cases}
		x'(t)=f(x(t)) \\
		x(0)=x_0
	\end{cases}\]
	\begin{dem}
		Supongamos sin pérdida de generalidad que $\forall x \in (a,b) : f(x)>0$
		\[\implies \forall x \in (a,b) : F'(x)=\frac{1}{f(x)}>0 \implies F \tex{ es creciente en }(a,b)\]
		\[\implies F \tex{ tiene inversa en } (a, b) \implies \exists x\defeq \appl{F^{-1}}{(T_-,T_+)}{(a,b)}\]
		\[\tex{Por un lado, }\quad x'(t) = (F^{-1})'(t)=\frac{1}{F'(F^{-1}(t))}=\frac{1}{F'(x(t))}\]
		\[\tex{Por otro lado, }\quad F(x_0)=0 \implies x_0= F^{-1}(F(x_0))=F^{-1}(0)=x(0)\]
	\end{dem}
\end{teo}

\begin{teo}[Unicidad local]
	Sean $a\in [-\infty, \infty) \we b\in (-\infty, \infty] \we \appl{f}{(a,b)}{\R}$ continua. Supongamos que $f(x)\ne 0$ en $(a,b)$. Sea $x_0 \in (a,b)$, sea $I\subset \R$ un intervalo abierto tal que $0\in I$ y sean $\appl{x}{I}{(a,b)} \we \appl{y}{I}{(a,b)}$ cumpliendo
	\[\begin{cases}
		x'(t)=f(x(t)) \\
		y'(t)=f(y(t)) \\
		x(0)=x_0=y(0)
	\end{cases} \implies \forall t \in I : x(t)=y(t)\]
	\begin{dem}
		\[\forall s \in (a,b) : F(s)=\int_{x_0}^{s} \frac{1}{f(r)}\odif{r} \implies \forall t \in I : F(x(t))=t=F(y(t))\]
		\[\implies \forall t \in I : F^{-1}(F(x(t)))=F^{-1}(F(y(t))) \implies \implies \forall t \in I : x(t)=y(t)\]
	\end{dem}
\end{teo}

\begin{cor}
	En las condiciones del teorema de unicidad local, sea $\ds a\in \R : f(a)\defeq \lim_{x\to a^-} f(x)=0$. Supongamos que
	\[\forall k \in (a,b) : \lim_{x\to a^+}\int_{k}^{x} \frac{1}{f(s)} \odif{s} = \begin{cases}
		-\infty \iff f>0 \tex{ en } (a,b) \\
		\infty \iff f<0 \tex{ en } (a,b)
	\end{cases}\]
	\[\implies \tex{Para cada intervalo} \begin{cases}
		I=[0, t_0) \iff f>0 \tex{ en } (a,b) \\
		I=(-t_0, 0] \iff f<0 \tex{ en } (a,b)
	\end{cases} x=a\tex{ es la única solución.}\]
\end{cor}

\fecha{26/02/2024}

\fecha{27/02/2024}

\begin{ejem}[$y'=\sqrt{1-y^2}$]
	\[\appl{f}{[-1, 1]}{\R} \we f(y)\defeq\sqrt{1-y^2} \we \begin{cases}
		f(y)>0 \iff y\in (-1, 1) \\
		f(-1)=f(1)=0
	\end{cases}\]
	Si $y(0)\eqdef y_0 \in (-1, 1)$, entonces existe una única solución del PVI. Esa solución está definida en $(T_-, T_+)$, donde 
	\[T_-=\lim_{y\to -1^+}\int_{y_0}^{y} \frac{1}{\sqrt{1-s^2}}\odif{s}=\lim_{y\to -1^+}\arcsin(y)-\arcsin(y_0)=-\frac{\pi}{2}-\arcsin(y_0)\]
	\[T_+=\lim_{y\to 1^-}\int_{y_0}^{y} \frac{1}{\sqrt{1-s^2}}\odif{s}=\frac{\pi}{2}-\arcsin(y_0)\]
	\[\tex{Si $y_0=1$, }\lim_{y\to 1^-} \int_{k}^{y} \frac{1}{\sqrt{1-s^2}} \odif{s} = \frac{\pi}{2} -\arcsin(k) \in \R \implies \exists \tex{ un solución \underline{no trivial} del PVI}\]	
	\[\tex{Si $y_0=-1$, }\lim_{y\to -1^+} \int_{k}^{y} \frac{1}{\sqrt{1-s^2}} \odif{s}  \in \R \implies \exists \tex{ un solución \underline{no trivial} del PVI}\]
	Por tanto, la solución general del PVI es
	\[y_k(x)=\begin{cases}
		-1 &\iff x\leq -\frac{\pi}{2}-k \\
		\sin(x+k) &\iff x \in \left(-\frac{\pi}{2}-k, \frac{\pi}{2}-k\right) \\
		1 &\iff x\geq \frac{\pi}{2}-k \\
	\end{cases}\]
	\begin{enumerate}
		\item La única $y_k$ que satisface $y_k(0)=0 \in (-1, 1)$ es $y_k(x)=y_0$
		\item Las funciones $y_k$ con $\ds k > \frac{\pi}{2}$ cumplen $y_k(0)=1$
		\item Las funciones $y_k$ con $\ds k < -\frac{\pi}{2}$ cumplen $y_k(0)=-1$ 
	\end{enumerate}
\end{ejem}

\begin{obs}
	Sea $a\in \R$, $b \in (-\infty, \infty] : b>a$ y $\appl{f}{[a,b)}{\R}$ continua tal que $\forall x \in (a,b) : f(x)\ne 0$ y $f(a)=0$. \\
	Supongamos que $\exists\, c >0, \delta \in (0, b-a) : \forall s \in [a, a+\delta) : |f(s)|\leq C(s -a)$ \\
	Vamos a comprobar que se cumplen las condiciones de unicidad para el PVI con $x(0)=x_0=a$ tanto en el caso $f>0$ como en el caso $f<0$.
	\begin{itemize}
		\item $\boxed{f>0}$ Queremos ver que $\ds \lim_{z\to a^+}\int_{a+\delta}^{z} \frac{1}{f(s)}\odif{s} = -\infty$
		\[\int_{a+\delta}^{z} \frac{1}{f(s)}\odif{s} = \int_{a+\delta}^{z} \frac{1}{\abs{f(s)}}\odif{s}=-\int_{z}^{a+\delta} \frac{1}{\abs{f(s)}}\odif{s} \leq -\frac{1}{C} \int_{z}^{a+\delta} \frac{1}{s-a}\odif{s}\]
		\[\implies \int_{a+\delta}^{z} \frac{1}{f(s)}\odif{s} \leq -\frac{1}{C}\left(\log(\delta)-\log(z-a)\right)\]
		\[\implies \lim_{z\to a^+}\int_{a+\delta}^{z} \frac{1}{f(s)}\odif{s} \leq -\infty \implies \tex{ Hay unicidad de PVI con }x(0)=a \tex{ en } [0, \tilde{t})\]
		\item $\boxed{f<0}$ De forma análoga $\ds \lim_{z\to a^+}\int_{a+\delta}^{z} \frac{1}{f(s)}\odif{s} = \cdots = \infty$
	\end{itemize}
	Si $f$ derivable con $f'$ acotada
	\[\implies \forall s \in [a, a+\delta) : \abs{f(s)}=\abs{f(s)-f(a)}=\abs{f'(r)}\abs{s-a}\leq C(s-a)\]
\end{obs}