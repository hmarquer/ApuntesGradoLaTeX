\section{Ecuaciones diferenciales ordinarias autónomas}

\subsection{Propiedades básicas}

\begin{defn}[EDO autónoma]
	Una ecuación diferencial ordinaria (EDO) de primer orden se dice autónoma si no depende explícitamente de la variable independiente. Es decir,
	\[\iff \tex{es de la forma } y'=f(y)\]
\end{defn}

\allbold{Propiedades de EDOs autónomas}
\begin{enumerate}
	\item \textbf{Isoclinas:} Todos los puntos de cada recta horizontal $y=c$ pertenecen a la misma isoclina. ¡Cuidado! A veces una isoclina puede contener más de una recta horizontal.
	      \begin{ejem}[$y'=y^2$]
		      \[\{(x,y)\in \R:y^2=c\}=\{(x,\sqrt{c}):x\in\R\}\cup\{(x,-\sqrt{c}):x\in\R\}\]
	      \end{ejem}
	\item \textbf{Traslaciones:} Si y es solución $\implies \forall c \in \R : w(x)\defeq y(x+c)$ es solución.
	      \begin{dem}
		      $w'(x)=y'(x+c)=f(y(x+c))=f(w(x))$
	      \end{dem}
	\item \textbf{Soluciones triviales:} Si $\exists \, a \in \operatorname{Dom}(f) : f(a)=0 \implies y(x)=a$ es solución.
	      \begin{dem}
		      $y'(x)=0=f(a)=f(y(x))$
	      \end{dem}
\end{enumerate}

\subsection{Teorema de existencia de soluciones}
Sean $a\in [-\infty, \infty) \we b\in (-\infty, \infty] \we \appl{f}{(a,b)}{\R}$ continua$ \we y_0 \in (a,b) : f(y_0) \ne 0$
\[\implies \exists \, \varepsilon : \bigg((y_0-\varepsilon, y_0+\varepsilon) \subset (a,b) \we \forall z \in (y_0-\varepsilon, y_0+\varepsilon) : f(z)\ne 0\bigg)\]

Sea $\appl{y}{I}{(a,b)}$ una función derivable con $I\subset \R$ intervalo abierto tal que
\[\forall x \in I : y'(x)=f(y(x)) \quad \we \quad y(x_0)=y_0 \tex{ para algún }x_0 \in I\]

Como $y$ es continua, $\exists \, \delta > 0 : \forall x \in (x_0-\delta, x_0+\delta) :y(x)\in(y_0-\varepsilon, y_0+\varepsilon)$
\[\implies \forall x \in (x_0-\delta, x_0+\delta) : f(y(x))\ne 0\]

Así, $\ds \forall x \in (x_0-\delta, x_0+\delta) : \frac{y'(x)}{f(y(x))}=1 \implies \forall x \in (x_0-\delta, x_0+\delta) : \int_{x_0}^{x} \frac{y'(z)}{f(y(z))}\odif{z} = x-x_0$.

Por tanto, mediante el cambio de variable $y(z) = s$, obtenemos
\[\implies \boxed{\forall x \in (x_0-\delta, x_0+\delta) : \int_{y_0}^{y(x)} \frac{1}{f(s)} \odif{s} + x_0 = x}\]
Es decir, si tenemos $f(y(x_0)) \ne 0$, entonces en un entorno de $x_0$ tenemos $x$ en función de $y$. $\ds x(y) = \int_{y(x_0)}^{y} \frac{1}{f(s)} \odif{s} + x_0$. Por tanto, tenemos una ``inversa'' de $y$.

\fecha{22/02/2024}

% \begin{teo}[Existencia de soluciones]
% 	Sea $a\in [-\infty, \infty) \we b\in (-\infty, \infty] \we \appl{f}{(a,b)}{\R}$ continua
% 	\[\implies \exists \, \appl{x}{\left(\lim_{x\to a^+}\int_{x_0}^{x} \frac{1}{f(s)} \odif{s}, \lim_{x\to b^-}\int_{x_0}^{x} \frac{1}{f(s)} \odif{s}\right)}{(a,b)} \tex{ derivable}: \begin{cases}
% 		x'(t)=f(x(t)) \\
% 		x(0)=x_0
% 	\end{cases}\]
% \end{teo}

\begin{teo}[Existencia de soluciones] \mbox{} \\
	Sean $a\in [-\infty, \infty) \we b\in (-\infty, \infty] \we \appl{f}{(a,b)}{\R}$ continua \\
	Supongamos que $\forall x \in (a,b) : f(x) \ne 0$ y que $\begin{cases}
			a>-\infty \implies f(a)=0 \\
			b<\infty \implies f(b)=0
		\end{cases}$ \\
	Sea $x_0 \in (a,b)$ definimos $\ds \forall x \in (a,b) : F(x)\defeq\int_{x_0}^{x} \frac{1}{f(s)} \odif{s}$ \\
	\begin{itemize}
		\item Si $f(x)>0$ en $(a,b)$, $\ds T_-\defeq \lim_{x\to a^+}F(x)\in [-\infty, 0) \we T_+\defeq \lim_{x\to b^-}F(x) \in (0, \infty]$.
		\item Si $f(x) < 0$ en $(a,b)$, $\ds T_+\defeq \lim_{x\to a^+}F(x)\in [-\infty, 0) \we T_-\defeq \lim_{x\to b^-}F(x) \in (0, \infty]$.
	\end{itemize}
	\[\implies \exists \, \appl{x}{(T_-, T_+)}{(a,b)} \tex{ derivable}: \begin{cases}
			x'(t)=f(x(t)) \\
			x(0)=x_0
		\end{cases}\]
	\begin{dem}
		Supongamos sin pérdida de generalidad que $\forall x \in (a,b) : f(x)>0$
		\[\implies \forall x \in (a,b) : F'(x)=\frac{1}{f(x)}>0 \implies F \tex{ es estrictamente creciente en }(a,b)\]
		\[\implies F \tex{ tiene inversa en } (a, b) \implies \exists\, x\defeq \appl{F^{-1}}{(T_-,T_+)}{(a,b)}\]
		\[\implies \begin{aligned}         & \tex{Por un lado, } x'(t) = \left(F^{-1}\right)'(t)=\frac{1}{F'(F^{-1}(t))}=\frac{1}{F'(x(t))} = f(x(t)) \\
                        & \tex{Por otro lado, } F(x_0)=0 \implies x_0= F^{-1}(F(x_0))=F^{-1}(0)=x(0)\end{aligned}\]
	\end{dem}
\end{teo}

\subsection{Teorema de unicidad local de soluciones}
\begin{teo}[Unicidad local] \mbox{} \\
	Sean $a\in [-\infty, \infty) \we b\in (-\infty, \infty] \we \appl{f}{(a,b)}{\R}$ continua con $\forall x \in (a,b) : f(x)\ne 0$. \\
	Sea $x_0 \in (a,b)$ y sean $\appl{x}{I}{(a,b)} \we \appl{y}{I}{(a,b)}$ dos funciones ($I\subset \R$ abierto $ : 0 \in I$) tales que
	\[\begin{cases}
			x'(t)=f(x(t)) \we y'(t)=f(y(t)) \\
			x(0)=x_0=y(0)
		\end{cases} \implies \forall t \in I : x(t)=y(t)\]
	\begin{dem}
		\[\forall s \in (a,b) : F(s)\defeq\int_{x_0}^{s} \frac{1}{f(r)}\odif{r} \implies \forall t \in I : F(x(t))=t=F(y(t))\]
		\[\implies \forall t \in I : F^{-1}(F(x(t)))=F^{-1}(F(y(t))) \implies \forall t \in I : x(t)=y(t)\]
	\end{dem}
\end{teo}

\begin{cor}
	Con las condiciones del teorema de existencia, supongamos $\ds a\in \R$ y\\
	$\ds f(a)\defeq \lim_{x\to a^-} f(x)=0$. Supongamos que
	\[\forall k \in (a,b) : \lim_{x\to a^+}\int_{k}^{x} \frac{1}{f(s)} \odif{s} = \begin{cases}
			-\infty \iff f>0 \tex{ en } (a,b) \\
			\infty \iff f<0 \tex{ en } (a,b)
		\end{cases}\]
	\[\implies \forall I \subset \R : \begin{cases}
			I=[0, t_0) \iff f>0 \tex{ en } (a,b) \\
			I=(-t_0, 0] \iff f<0 \tex{ en } (a,b)
		\end{cases} : x\equiv a\tex{ es la única solución.}\]
	\fecha{26/02/2024}
	\begin{dem}
		Sin pérdida de generalidad, suponemos que $\forall z \in (a,b) : f(z) > 0$. \\
		Entonces suponemos que $\ds \lim_{x\to a^+}\int_{k}^{x} \frac{1}{f(s)} \odif{s} = -\infty$. Procedemos por contradicción asumiendo que $\exists \appl{x}{[0, \varepsilon)}{(a,b)} \tex{ solución del PVI con } x_0 = a \tex{ que no es constante.}$
		\[\implies \forall t \in [0, \varepsilon) : x'(t)=f(x(t)) \geq 0 \implies x \tex{ es creciente en } [0, \varepsilon)\]
		Sin pérdida de generalidad, podemos suponer que $\forall t \in (0, \varepsilon) : x(t) > a$ porque, si no fuese así, entonces $\exists \, t_0 \in (0, \varepsilon) : \forall t \in [0, t_0) : x(t)=a \implies \forall t \in (t_0, \varepsilon) : x(t) > a$. Entonces consideramos $\hat{x}(t) = x(t + t_0)$ que es solución del PVI en $[0, \varepsilon - t_0)$ que además cumple $\forall t \in (0, \varepsilon - t_0) : \hat{x}(t) > a$.

		Por tanto, $\ds \forall t \in (0, \varepsilon) : f(x(t)) > 0 \implies \forall t \in (0, \varepsilon) : \frac{x'(t)}{f(x(t))} = 1$
		\[\implies \forall t \in (0, \varepsilon) : \int_{\varepsilon/2}^{t} \frac{x'(s)}{f(x(s))} \odif{s} = t -\frac{\varepsilon}{2} \implies \int_{x(\varepsilon/2)}^{x(t)} \frac{1}{f(s)} \odif{s} = t -\frac{\varepsilon}{2}\]
		Entonces, pasando al límite tenemos
		\[-\infty = \lim_{t\to 0} \int_{x(\varepsilon/2)}^{x(t)} \frac{1}{f(s)} \odif{s} = \lim_{t\to 0} \left(t - \frac{\varepsilon}{2}\right) = - \frac{\varepsilon}{2} \quad \contr\]
	\end{dem}
\end{cor}

\begin{cor}
	Con las hipótesis del corolario anterior, si $\ds \lim_{x\to a^+} \int_{k}^{x} \frac{1}{f(s)} \odif{s} \in \R$ \[\implies \exists \, \appl{x}{[0, \infty)}{(a,b)} \tex{ solución no trivial del PVI con } x(0) = a\]
	% \begin{dem}
	% \end{dem}
\end{cor}

\fecha{27/02/2024}

\begin{ejem}[$y'=\sqrt{1-y^2}$]
	\[\appl{f}{[-1, 1]}{\R} \we f(y)\defeq\sqrt{1-y^2} \we \begin{cases}
			f(y)>0 \iff y\in (-1, 1) \\
			f(-1)=f(1)=0
		\end{cases}\]
	Si $y(0)\eqdef y_0 \in (-1, 1)$, entonces existe una única solución del PVI. Esa solución está definida en $(T_-, T_+)$, donde
	\[T_-=\lim_{y\to -1^+}\int_{y_0}^{y} \frac{1}{\sqrt{1-s^2}}\odif{s}=\lim_{y\to -1^+}\arcsin(y)-\arcsin(y_0)=-\frac{\pi}{2}-\arcsin(y_0)\]
	\[T_+=\lim_{y\to 1^-}\int_{y_0}^{y} \frac{1}{\sqrt{1-s^2}}\odif{s}=\frac{\pi}{2}-\arcsin(y_0)\]
	\[\tex{Si $y_0=1$, }\lim_{y\to 1^-} \int_{k}^{y} \frac{1}{\sqrt{1-s^2}} \odif{s} = \frac{\pi}{2} -\arcsin(k) \in \R \implies \exists \tex{ una solución \underline{no trivial} del PVI}\]
	\[\tex{Si $y_0=-1$, }\lim_{y\to -1^+} \int_{k}^{y} \frac{1}{\sqrt{1-s^2}} \odif{s}  \in \R \implies \exists \tex{ una solución \underline{no trivial} del PVI}\]
	Por tanto, la solución general del PVI es
	\[y_k(x)=\begin{cases}
			-1        & \iff x\leq -\frac{\pi}{2}-k                               \\
			\sin(x+k) & \iff x \in \left(-\frac{\pi}{2}-k, \frac{\pi}{2}-k\right) \\
			1         & \iff x\geq \frac{\pi}{2}-k                                \\
		\end{cases}\]
	\begin{enumerate}
		\item La única $y_k$ que satisface $y_k(0)=0 \in (-1, 1)$ es $y_k(x)=y_0$
		\item Las funciones $y_k$ con $\ds k > \frac{\pi}{2}$ cumplen $y_k(0)=1$
		\item Las funciones $y_k$ con $\ds k < -\frac{\pi}{2}$ cumplen $y_k(0)=-1$
	\end{enumerate}
\end{ejem}

\begin{obs}
	Sea $a\in \R$, $b \in (-\infty, \infty] : b>a$ y $\appl{f}{[a,b)}{\R}$ continua tal que $\forall x \in (a,b) : f(x)\ne 0$ y $f(a)=0$. \\
	Supongamos que $\exists\, C >0, \delta \in (0, b-a) : \forall s \in [a, a+\delta) : |f(s)|\leq C(s -a)$ \\
	Vamos a comprobar que se cumplen las condiciones de unicidad para el PVI con $x(0)=x_0=a$ tanto en el caso $f>0$ como en el caso $f<0$.
	\begin{itemize}
		\item $\boxed{f>0}$ Queremos ver si $\ds \lim_{z\to a^+}\int_{a+\delta}^{z} \frac{1}{f(s)}\odif{s} = -\infty$
		      \[\int_{a+\delta}^{z} \frac{1}{f(s)}\odif{s} = \int_{a+\delta}^{z} \frac{1}{\abs{f(s)}}\odif{s}=-\int_{z}^{a+\delta} \frac{1}{\abs{f(s)}}\odif{s} \leq -\frac{1}{C} \int_{z}^{a+\delta} \frac{1}{s-a}\odif{s}\]
		      \[\implies \int_{a+\delta}^{z} \frac{1}{f(s)}\odif{s} \leq -\frac{1}{C}\left(\log(\delta)-\log(z-a)\right)\]
		      \[\implies \lim_{z\to a^+}\int_{a+\delta}^{z} \frac{1}{f(s)}\odif{s} \leq -\infty \implies \tex{ Hay unicidad de PVI con }x(0)=a \tex{ en } [0, \tilde{t})\]
		\item $\boxed{f<0}$ De forma análoga $\ds \lim_{z\to a^+}\int_{a+\delta}^{z} \frac{1}{f(s)}\odif{s} = \cdots \leq \infty$
	\end{itemize}
	Si $f$ derivable con $f'$ acotada
	\[\implies \forall s \in [a, a+\delta) : \abs{f(s)}=\abs{f(s)-f(a)}=\abs{f'(r)}\abs{s-a}\leq C(s-a)\]
\end{obs}

\fecha{04/03/2024}

\subsection{Teorema de existencia global/asíntotas de soluciones}

\begin{teo}
	Sean $a\in [-\infty, \infty) \we b\in (-\infty, \infty] \we a < b \we y_0 \in (a, b) \we \appl{f}{(a,b)}{\R}$ continua tal que $\forall x \in (a, b) : f(x) > 0$. Si $b \in \R$, $f(b)=0$, si no, $\ds \lim_{x\to\infty} f(x) = \infty$.

	Así tenemos $\ds T_-=\lim_{x\to a^+} F(x) \we T_+=\lim_{x\to b^-} F(x)$ con $\ds F(x)\defeq\int_{y_0}^{x} \frac{1}{f(s)}\odif{s}$.
	\begin{enumerate}
		\item $\left(T_+ = \infty\right) \ve \left(T_+ \in \R \we b \in \R\right) \implies$ la única sol del PVI se puede definir en $(T_-, \infty)$.
		\item $T_+ \in \R \we b = \infty \implies$ la única sol del PVI tiene una asíntota en $T_+$ con $\ds \lim_{x\to T_+^-} y(x) = \infty$
	\end{enumerate}
	\begin{dem}
		\begin{enumerate}
			\item[]
			\item Si $T_+ = \infty$ el resultado es inmediato.

			      Si $T_+ \in \R \we b \in \R$, es decir, $\ds \lim_{x\to b^-} \int_{y_0}^{x} \frac{1}{f(s)} \odif{s} \in \R$, entonces $\appl{y}{(T_-, T_+)}{\R}$ se puede extender de forma diferenciable tomando $\forall x \geq T_+ : y(x)=b$.

			      Ahora tenemos que comprobar que cumple las dos condiciones necesarias.
			      \[\begin{aligned}
					      y \tex{ es continua: }      & \lim_{x\to T_+^-} y(x) = b = y(T_+)                            \\
					      y \tex{ es diferenciable: } & \lim_{x\to T_+^-} y'(x) = \lim_{x\to T_+^-} f(y(x)) = f(b) = 0
				      \end{aligned}\]
			\item Si $T_+ \in \R \we b = \infty$, es decir, $\ds \lim_{x\to\infty} F(x) \in \R$, como sabemos que $\appl{y}{(T_-, T_+)}{\R}$ se define como $y\defeq F^{-1}$, entonces $\ds \lim_{x\to T_+^-} y(x) = \infty$ dado que $F$ es creciente.
		\end{enumerate}
	\end{dem}
\end{teo}

\begin{obs}
	Se cumple para los casos análogos con $\begin{cases}
			f(y) > 0 \tex{ con asíntota en } T_- \\
			f(y) < 0 \tex{ con asíntota en } T_+ \\
			f(y) < 0 \tex{ con asíntota en } T_- \\
		\end{cases}$
\end{obs}
\pagebreak
\fecha{05/03/2024}
\subsection{Estabilidad de soluciones}

Sea $\appl{f}{(\alpha,\beta)}{\R}$ continua, con el PVI $\{y'=f(y) \, \we \, y(x_0) = y_0\}$.
\begin{defn}[Equilibrio]
	Sea $y$ una solución del PVI, $y$ es de equilibrio
	\[\iff \tex{es de la forma } \forall x \in \R : y(x)=a \in (\alpha, \beta) : f(a) = 0\]
\end{defn}

\begin{defn}[Estabilidad]
	Si $\forall y_0 \in (\alpha, \beta)$ el PVI tiene solución única, la solución $\appl{y}{[x_0, \infty)}{\R}$ es estable
	\[\iff \forall \varepsilon >0 : \exists \delta > 0 : \bigg(y_0^*\in (\alpha, \beta) : \abs{y_0^*-y_0} < \delta \implies \forall x \geq x_0 :\abs{y(x)-y^*(x)}<\varepsilon\bigg)\]
	donde $\appl{y^*}{[x_0, \infty)}{\R}$ es cualquier solución del PVI con $y^*(x_0)=y_0^*$.
\end{defn}

\begin{ejem}[$\{y'=0 \we y(x_0)=1\}$]
	Vemos a ojo que la función $y(x)=1$ es solución de equilibrio. Además, vamos a ver que es estable.

	Sea $\varepsilon > 0$, tomamos $\delta = \varepsilon$ y la única solución de $\{y'=0 \we y(x_0) = y_0^*\}$ con $\abs{y_0^* -1} < \delta$ que claramente es $y^*(x) = y_0^*$ y cumple las condiciones pedidas de que \[\forall x \geq x_0 : \abs{y(x) - y^*(x)} = \abs{y_0^* - 1} = \delta < \varepsilon\]
\end{ejem}

\begin{defn}[Estabilidad asintótica]
	Sea $\appl{y}{[x_0, \infty)}{\R}$ una solución del PVI, $y$ es asintóticamente estable (a futuro) $\ds \iff y \tex{ es estable } \we \lim_{x\to \infty}\abs{y(x)-y^*(x)}=0$ \\
	donde $\appl{y^*}{[x_0, \infty)}{\R}$ es cualquier solución del PVI con $y^*(x_0)=y_0^*$.
\end{defn}

\begin{ejem}
	\[\begin{cases}
			y'=y(1-y) \\
			y(x_0)=y_0
		\end{cases} \implies \begin{cases}
			f(y)\defeq y(1-y) \\
			f(y) = 0 \iff y \in \{0, 1\}
		\end{cases} \implies \begin{cases}
			y \equiv 0 \\
			y \equiv 1 \\
		\end{cases}\tex{ son de equilibrio}\]
	\[\xRightarrow{\tex{H2: ej. 1}} \big(\forall y < 0 :f(y) < 0\big) \we \big(\forall y \in (0, 1) : f(y) > 0\big) \we \big(\forall y > 1 : f(y) < 0\big)\]
	Por tanto, si una solución tiene límite, entonces ese límite es solución estable.

	$\boxed{y_0 < 0}$ \hspace{1cm}$\ds \forall y < 0 : F(y)=\int_{y_0}^{y}\frac{1}{s(1-s)}\odif{s} \implies \forall y<0 : F'(y)<0$

	Tomamos $y\in (y_0, 0) \implies F(y)<F(y_0)=0$
	\[\implies s \geq y_0 \implies 0<1-s \leq 1-y_0 \implies \frac{1}{1-s}\geq\frac{1}{1-y_0} \implies \frac{1}{(1-s)s}\leq \frac{1}{(1-y_0)s}\]
	\[\implies F(y) \leq \frac{1}{1-y_0}\int_{y_0}^{y} \frac{1}{s} \odif{s} = \frac{\log{(-y)} - \log{(-y_0)}}{1-y_0} \implies \boxed{\lim_{y\to0^-} F(y) = -\infty}\]
	\begin{enumerate}
		\item La única solución tal que $y(x_0)=y_0<0$ está definida globalmente hacia el pasado.
		\item El equilibrio $y=0$ es único ``por abajo''.
	\end{enumerate}
	$(y\to-\infty)$
	\[y<y_0<0 \implies 0 \leq F(y)=-\int_{y}^{y_0} \frac{1}{s(1-s)} \odif{s} = \int_{y}^{y_0} \frac{1}{s^2-s} \odif{s} \leq \int_{y}^{y_0} \frac{1}{s^2} \odif{s} = \frac{1}{y} -\frac{1}{y_0}\]
	\[\implies 0\leq F(y) < -\frac{1}{y} \implies \boxed{\lim_{y\to -\infty} F(y) \in \R} \implies \tex{Hay una asíntota}\]
	$\boxed{y_0>1}$ Si $y$ es solución con $y(x_0)=y_0>1$, entonces $z(x)=-y(-x)+1$ también es solución con $z(x_0)=1-y_0<0$.
	% \begin{center}
	% 	\newcommand{\numXSamples}{201}
\newcommand{\minimumX}{-2}
\newcommand{\maximumX}{2}
\newcommand{\minimumY}{-2}
\newcommand{\maximumY}{2}
\newcommand{\arrowSamples}{15}
\newcommand{\initialy}{0.1}
\pgfmathsetmacro{\arrowScaler}{0.7*(\maximumX-\minimumX)/\arrowSamples}
\newcommand{\yprime}[2]{((#2)*(1-#2))}%(abs(#1^2)+abs(#2^2)-1)}


\pgfplotstableset{
	create on use/x/.style={
		create col/expr={
			\pgfplotstablerow/\numXSamples*(\maximumX-\minimumX)+\minimumX
		}
	},
	create on use/y/.style={
		create col/expr accum={
			\pgfmathaccuma+((\maximumX-\minimumX)/\numXSamples)*
			(\yprime{\thisrow{x}}{\pgfmathaccuma})
			%(abs(\pgfmathaccuma^2)+abs(\thisrow{x}^2)-1)
		}{\initialy}
	}
}

\pgfplotstablenew{\numXSamples}\loadedtable

\begin{center}
\begin{tikzpicture}[scale=1] % Adjust the scale factor as needed
\begin{axis}[
	width=10cm, % Adjust the width of the graph
	height=10cm, % Adjust the height of the graph
	view={0}{90},
	domain=\minimumX:\maximumX,
	y domain=\minimumY:\maximumY,
	xmax=\maximumX, ymax=\maximumY,
	samples=\arrowSamples,
	% x={(3cm,0cm)}, y={(0cm,3cm)}, 
]
\draw[black, thick] (axis cs: \minimumX,0,0) -- (axis cs: \maximumX,0,0);
\draw[black, thick] (axis cs: 0,\minimumY,0) -- (axis cs: 0,\maximumY,0);
\addplot3 [gray, quiver={u={1}, v={\yprime{x}{y}},
	scale arrows=\arrowScaler,
	every arrow/.append style={-latex}}] (x,y,0);
\addplot [line width=1.75pt, red] table [x=x, y=y] {\loadedtable};
\end{axis}
\end{tikzpicture}
\end{center}
	% \end{center}
\end{ejem}

\fecha{06/03/2024}
\subsection{Bifurcación}
\begin{defn}[Puntos de bifurcación]
	Sea $\appl{f_\mu}{\R}{\R}$ una función continua que depende de un parámetro $\mu \in \R$. El comportamiento cualitativo de la EDO $y'=f_\mu(y)$ puede cambiar dependiendo de la $\mu$. Los valores de $\mu$ que dan lugar a un cambio de este tipo* son los puntos de bifurcación.

	\hfill *Este tipo: de cortar dos veces al equilibrio a cortarlo una o ninguna vez.
\end{defn}

\begin{ejem}[$\{y'=y(1-y)-\mu \we y(x_0) = y_0\}$]
	\[f_\mu(y)=y(1-y)-\mu=0 \iff y^2-y+\mu=0 \iff y=\frac{1\pm \sqrt{1-4\mu}}{2} \we \mu \leq \frac{1}{4}\]
	Curvas de equilibrios: $\ds \gamma_1(\mu)\defeq\frac{1+\sqrt{1-4\mu}}{2} \we \gamma_2(\mu)\defeq\frac{1-\sqrt{1-4\mu}}{2}$
	\[\implies \begin{cases}
			\mu > \frac{1}{4} \tex{ no hay equilibrios}  \\
			\mu < \frac{1}{4} \tex{ hay dos equilibrios} \\
			\mu = \frac{1}{4} \tex{ hay un equilibrio}   \\
		\end{cases} \implies \mu =\frac{1}{4} \tex{ es un punto de bifurcación.}\]
	\begin{itemize}
		\item $\mu > \frac{1}{4} \implies $la solución $y$ es decreciente porque la derivada nunca toca el eje $X$.
		\item $\mu = \frac{1}{4} \implies $las soluciones son siempre decrecientes con un punto de silla $(y'=0)$.
		\item $\mu < \frac{1}{4} \implies y$ es estrictamente creciente para $y_0 \in \left(\frac{1-\sqrt{1-4\mu}}{2}, \frac{1+\sqrt{1-4\mu}}{2}\right)$, constante para $y_0 = \frac{1\pm\sqrt{1-4\mu}}{2}$, y estrictamente decreciente en otro caso.
	\end{itemize}
\end{ejem}
\fecha{07/03/2024}
\begin{ejem}[$\{y'=y(\mu - y^2) \we y(x_0)=y_0\}$]
	% \[\implies \begin{cases}
	% 		\mu < 0 & \tex{$y$ pasa por el 0 con derivada negativa} \\
	% 		\mu = 0 & \tex{$y$ pasa por el 0 con derivada nula}     \\
	% 		\mu > 0 & \tex{$y$ pasa por el 0 con derivada positiva}
	% 	\end{cases}\]
	\[f_\mu (y) \defeq y(\mu - y^2) = 0 \iff y=0 \ve y=\pm\sqrt{\mu} \implies \forall \mu \leq 0 : \begin{cases}
			\gamma_1 (\mu) \defeq \sqrt{\mu} \\
			\gamma_2 (\mu) \defeq -\sqrt{\mu}
		\end{cases}\]
	\[\implies \begin{cases}
			\mu \leq 0 & \tex{hay un equilibrio } (y = 0)                          \\
			\mu > 0    & \tex{hay tres equilibrios } (y = 0 \ve y = \pm\sqrt{\mu})
		\end{cases}\]
	Definimos $\ds F(y) \defeq \int_{y_0}^{y} \frac{1}{s(\mu - s^2)}\odif{s}$ y establecemos $\mu < 0$.
	\begin{itemize}
		\item $\ds y_0< 0 \implies \forall y < 0 : F'(y)=\frac{1}{y(\mu - y^2)} >0 \implies y$ es estrictamente creciente.

		      Como $\ds \lim_{y\to-\infty} F(y) \in \R$, entonces hay una asíntota hacia atrás y para abajo.
		\item $\ds y_0 > 0 \implies \forall y > 0 : F'(y) = \frac{1}{y(\mu - y^2)} < 0 \implies y$ es estrictamente decreciente.
		      Como $\ds \lim_{y\to \infty} F(y) \in \R$, entonces hay una asíntota hacia atrás y para arriba.
	\end{itemize}
	Si establecemos $\mu > 0$, entonces
	\begin{itemize}
		\item $\ds y_0 \in (-\sqrt{\mu}, 0) \implies \forall y \in (-\sqrt{\mu}, 0) : F'(y) < 0 \implies y$ es estrictamente decreciente.
		\item $\ds y_0 \in (0, \sqrt{\mu}) \implies \forall y \in (0, \sqrt{\mu}) : F'(y) > 0 \implies y$ es estrictamente creciente.
	\end{itemize}
\end{ejem}
\begin{prop}
	Sea $y'=f_\mu(y)$ una EDO autónoma con $\mu^* \in \R$ punto de bifurcación
	\[\varepsilon^*\defeq\{y\in \R : f_{\mu^*} (y) = 0\} \ne \phi \implies \exists \, y^*\in \varepsilon^* : f_{\mu^*}(y^*)=0 \we f_{\mu^*}'(y^*) = 0\]
\end{prop}
\begin{ejem}[$y'=e^{-y} - \mu = f_\mu(y)$]
	\[f_\mu(y) = 0 \iff y = -\ln{\mu} \implies \begin{cases}
			\mu \leq 0 & \tex{no hay equilibrios} \\
			\mu > 0    & \tex{hay un equilibrio}
		\end{cases} \implies \mu^*=0 \tex{ es de bifurcación}\]
	\[\implies \varepsilon^* = \{y\in \R : e^{-y}-0 = 0\} = \phi \implies \tex{no se puede aplicar la proposición (vaya ejemplo)}\]
\end{ejem}