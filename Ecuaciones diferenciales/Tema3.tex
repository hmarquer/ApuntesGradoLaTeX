\fecha{11/03/2024}

\section{Teoremas fundamentales}

\subsection{Introducción}

Sean $I\subset \R$, $\Omega \subset \R$ dos intervalos abiertos y $\appl{f}{I\times \Omega}{\R^2}$ una función continua. Consideramos el PVI $\begin{cases}
		x'=f(t, x(t)) , & t\in I            \\
		x(t_0)=\hat{x}, & \hat{x}\in \Omega
	\end{cases}$.
Hasta ahora sabemos trabajar con EDOs autónomas: si  $f(t, x)=f(x)$, entonces el PVI tiene siempre solución en un entorno de $t_0$.

Vamos a intentar darle condiciones a la $f$ de ahora para poder comprobar si hay existencia y unicidad, igual que en el tema anterior. Vamos a integrar a ver qué pasa:
\[\int_{t_0}^{t} x'(s) \odif{s} = \int_{t_0}^{t} f(s, x(s)) \odif{s} \implies x(t) - x(t_0) = \int_{t_0}^{t} f(s, x(s)) \odif{s}\]
Por tanto, si existe solución, esta es de la forma $\ds x(t) = \hat{x} + \int_{t_0}^{t} f(s, x(s)) \odif{s}$.

Definimos el operador $\ds \forall z \in \mathcal{C}(I, \Omega) : T[z]\defeq z + \int_{t_0}^{t} f(s, z(s)) \odif{s} \implies x=T[x]$.

Hemos reducido el problema a encontrar un punto fijo de $T$, procedemos calculando las iteradas de Picard de $T$:
\[x_0 = \hat{x} \we x_1 = T[x_0] = \hat{x} + \int_{t_0}^{t} f(s, \hat{x}) \odif{s} \we \cdots \we x_{k+1} = T[x_k]= \hat{x} + \int_{t_0}^{t} f(s, x_k(s)) \odif{s}\]
El objetivo es encontrar el límite de esta sucesión y demostrar que es una solución del PVI.
\begin{ejem}
	$\ds\{x'=x \we x(0)=1\} \iff x(t)=1+\int_{0}^{t} x(s) \odif{s}$
	\[\implies x_0 = 1 \we x_1 = 1 + t \we \cdots \we x_k = \sum_{i=0}^{k} \frac{t^i}{i!} \implies x_k(t)\xrightarrow{k\to\infty} e^t\]
	Como es una EDO autónoma y en el dato inicial el lado derecho no se anula, sabemos que la solución es única y podemos afirmar que es la que acabamos de encontrar. Este es un método de encontrar el límite que en este caso ha funcionado.
\end{ejem}

Pero necesitamos formalizar todo esto.
\begin{enumerate}
	\item Concepto de límite de series de funciones.
	\item ¿Toda sucesión de Cauchy es convergente?
	\item $\ds \lim_{k\to\infty} \int_{t_0}^t f(s, x_k(s)) \odif{s} \stackrel{?}{=} \int_{t_0}^t \lim_{k\to\infty} f(s, x_k(s)) \odif{s}$
	\item $\ds \lim_{k\to\infty} f(s, x_k(s)) \stackrel{?}{=} f\left(s, \lim_{k\to\infty} x_k(s)\right)$
\end{enumerate}
Los dos últimos puntos son equivalentes a $\ds \lim_{k\to\infty} T\left[x_k\right] \stackrel{?}{=} T\left[\lim_{k\to\infty} x_k\right]$
\subsection{Conceptos de análisis}
\subsubsection{Convergencia puntual y uniforme}
\begin{defn}[Convergencia puntual]
	Sea $(f_k)_{k\in\N}$ con $\appl{f_k}{I}{\R}$ e $I\subset \R$ abierto una sucesión de funciones, $(f_k)$ converge puntualmente a $f$
	\[\begin{aligned}
			\iff & \forall t\in I : \lim_{k\to\infty} f_k(t) = f(t)                                                                              \\
			\iff & \forall t \in I : \forall \varepsilon > 0 :  \exists \kappa \in \N : \forall k\geq \kappa : \abs{f_k(t) - f(t)} < \varepsilon
		\end{aligned}\]
\end{defn}

\begin{ejem} %TODO: gráfica
	La sucesión de funciones continuas $(f_k)$ definida a continuación converge puntualmente a $f$ que no es continua.
	\[\forall t \in\R : f_k(t) \defeq \begin{cases}
			0,                & t < -\frac{1}{k}        \\
			k(t+\frac{1}{k}), & -\frac{1}{k} \leq t < 0 \\
			k(\frac{1}{k}-t), & 0\leq t < \frac{1}{k}   \\
			0,                & t\geq \frac{1}{k}
		\end{cases} \implies f_k(t)\xrightarrow{k\to\infty}\begin{cases}
			0, & t\ne 0 \\
			1, & t=0
		\end{cases}\]
\end{ejem}
\fecha{12/03/2024}
\begin{ejem} %TODO: gráfica
	La sucesión de funciones $(x_k)$ definida a continuación no intercambia el límite con la integral.
	\[x_k(t) \defeq \begin{cases}
			2k^2t,                          & t\in\left[0, \frac{1}{2k}\right)           \\
			2k^2\left(\frac{1}{k}-t\right), & t\in\left[\frac{1}{2k}, \frac{1}{k}\right) \\
			0,                              & t\in\left[\frac{1}{k}, 1\right)            \\
		\end{cases}\implies \begin{aligned}
			 & x_k(0)=0 \we x_k(1)=0                                  \\
			 & t\in (0,1] \implies x_k(t)  \xrightarrow{k\to\infty} 0
		\end{aligned}\]
	$\ds \lim_{k\to\infty} \int_{0}^1 x_k(t) \odif{t} \stackrel{?}{=} \int_{0}^1 \lim_{k\to\infty} x_k(t) \odif{t}$:
	\[ \int_{0}^1 \lim_{k\to\infty} x_k(t) \odif{t} = \int_{0}^{1} 0 \odif{t} = 0 \ne \frac{1}{2} = \lim_{k\to\infty}\int_{0}^{1} x_k(t) \odif{t}\]
\end{ejem}
\begin{defn}[Convergencia uniforme]
	Sea $(f_k)_{k\in\N}$ con $\appl{f_k}{I}{\R}$ e $I\subset \R$ abierto una sucesión de funciones, $(f_k)$ converge uniformemente a $f$
	\[\iff \forall \varepsilon > 0 : \exists \kappa \in \N : \forall k\geq \kappa : \forall t\in I : \abs{f_k(t) - f(t)} < \varepsilon\]
	Es decir, $\ds \forall \varepsilon > 0 : \exists \kappa \in \N : \forall k \geq \kappa : \sup_{t\in I} \abs{x_k(t)-x(t)} \leq \varepsilon$
\end{defn}
\begin{obs}\begin{itemize}
		\item Los dos ejemplos anteriores no convergen uniformemente.
		\item La convergencia uniforme implica convergencia puntual.
	\end{itemize}\end{obs}
\begin{prop}
	Sea $(x_k)_{k\in\N}$ una sucesión de funciones uniformemente convergente a $x$.
	\[\implies x \tex{ es continua}\]
	\begin{dem}
		Sean $\varepsilon > 0 \we t_0 \in I$. Buscamos $\delta > 0$ tal que, dado $t\in I$
		\[\abs{t-t_0} < \delta \implies \abs{x(t)-x(t)}< \varepsilon\]
		Para cada $k\in\N$, $\abs{x(t)-x_k(t)+x_k(t)-x_k(t_0)+x_k(t_0)-x_k(t_0)} \leq$
		\[\leq \abs{x(t)-x_k(t)} + \abs{x_k(t)-x_k(t_0)}  + \abs{x_k(t_0)-x(t_0)}\]
		Como $(x_k)$ converge uniformemente a $x$, $\kappa = \kappa(\varepsilon) \in \N$ tal que
		\[\forall \varepsilon > 0 : \exists \kappa \in \N : \forall k\geq \kappa : \forall t\in I : \abs{x_k(t) - x(t)} < \frac{\varepsilon}{3}\]
		Entonces, para $k=\kappa$, se tiene
		\[\begin{aligned}
				\abs{x(t)-x(t_0)} & \leq \abs{x(t)-x(t_0)}+\abs{x(t)-x_\kappa(t_0)} + \abs{x_\kappa(t_0)-x(t_0)}         \\
				                  & \leq \frac{\varepsilon}{3} + \abs{x_\kappa(t)-x_\kappa(t_0)} + \frac{\varepsilon}{3}
			\end{aligned}\]
		Como $x_\kappa$ es continua, $\exists \delta > 0 : \abs{t-t_0} < \delta \implies \abs{x_\kappa(t)-x_\kappa(t_0)} < \frac{\varepsilon}{3}$
		\[\implies\abs{x(t)-x(t_0)} < \varepsilon\]
	\end{dem}
\end{prop}
\begin{prop}
	Sea $(x_k)_{k\in\N}$ una sucesión de funciones continuas $\appl{x_i}{(a,b)}{\R}$ que converge uniformemente a $\appl{x}{(a,b)}{\R}$ donde $(a,b) \subset \R$ está acotado.
	\[\implies \lim_{k\to\infty} \int_{a}^{b} x_k(t) \odif{t} = \int_{a}^{b} x(t) \odif{t}\]
	\begin{dem}
		Sea $\varepsilon > 0$. La convergencia uniforme nos asegura que
		\[\exists \kappa_\varepsilon \in \N : \forall k \geq \kappa : \forall t \in (a, b) : \abs{x_k(t) - x(t)} < \frac{\varepsilon}{b-a}\]
		\[\begin{aligned}
				\implies\forall k\geq \kappa_\varepsilon : \abs{\int_{a}^{b} x_k(t) \odif{t} - \int_{a}^{b} x(t) \odif{t}} & = \abs{\int_{a}^{b} \left(x_k(t) - x(t)\right) \odif{t}}                                                     \\
				                                                                                                           & \leq \int_{a}^{b} \abs{x_k(t) - x(t)} \odif{t} < \int_{a}^{b} \frac{\varepsilon}{b-a} \odif{t} = \varepsilon
			\end{aligned}\]
	\end{dem}
\end{prop}

\fecha{13/03/2024}
\begin{ejem} %TODO: gráfica
	Consideramos la sucesión $x_k(t) = \begin{cases}
			\frac{1}{k} & \tex{ si } t\leq k \\
			0           & \tex{ si } t>k
		\end{cases}$
	\[\implies \int_{0}^{\infty} x_k(t) \odif{t} = \int_{0}^{k} \frac{1}{k} \odif{t} = 1 \we \lim_{k\to\infty} x_k(t) = 0\]
\end{ejem}
\subsubsection{Espacios normados y contracciones}
Sea $\mathcal{C}([a,b], \R)=\mathcal{C}([a,b]) \defeq \left\{\appl{x}{[a,b]}{\R} : x\tex{ continua}\right\}$ el espacio vectorial de funciones continuas.
\[\tex{con la norma }\forall x \in \mathcal{C}([a,b]) : \norm{x}_{\infty} \defeq \max_{t\in [a,b]} \abs{x(t)} \tex{ es un espacio vectorial normado }\]
\[\implies \big(\mathcal{C}([a,b]), \norm{\cdot}_\infty\big) \tex{ también es métrico con } \operatorname{d}(x,y)\defeq\norm{x-y}_\infty\]
\[\operatorname{d}(x_k, x) = \norm{x_k-x}_\infty=\max_{t\in [a,b]} \abs{x_k(t)-x(t)} \rightarrow0 \iff x_k\to x \tex{ uniformemente}\]
\begin{teo}
	Sean $a, b \in \R : a<b \implies \ds \bigg(\mathcal{C}([a,b]), \norm{\cdot}_\infty\bigg)$ es un espacio completo.\\Es decir, toda sucesión de Cauchy es convergente. El recíproco también es cierto, toda sucesión convergente es de Cauchy.
	\begin{dem}
		Sea $(x_k)_{k\in \N}\subset \mathcal([a,b])$ una sucesión de Cauchy. Esto significa que
		\[\forall \varepsilon > 0 : \exists \kappa \in \N : \big(k, l\geq \kappa \implies \norm{x_k-x_l}_\infty <\varepsilon\big) \]
		Recordemos que $\ds \norm{x_k-x_l}_\infty = \max_{t\in[a,b]}\abs{x_k(t)-x_l(t)}$\\
		Fijamos $\varepsilon > 0$ y $\ds t\in [a,b] \implies \forall k, l \geq \kappa : \abs{x_k(t) - x_l(t)} \leq \norm{x_k-x_l}_\infty < \varepsilon$

		Esto demuestra que la sucesión de números reales $(x_k(t))_{k\in\N}$ es de Cauchy. \\
		Como $\R$ es completo, $\exists x(t) \in \R : x_k(t) \xrightarrow{k\to\infty} x(t)$.

		Veamos que el límite es uniforme: En efecto, sean $k\geq \kappa, l \geq 1, k, l \in \N$.
		\[\implies \forall t \in [a,b] : \abs{x_k(t)-x_{k+l}(t)} \leq \norm{x_k-x_{k+l}} < \varepsilon \tex{ porque $(x_k)$ es de Cauchy}\]
		Haciendo tender $l\to\infty$, tenemos $\ds \forall t \in [a,b] : \forall k \geq \kappa : \abs{x_k(t)-x(t)}\leq \varepsilon$

		Recordemos que $x\in\mathcal{C}([a,b])$ por ser límite de funciones continuas.

		Para el recíproco:
		\[\norm{x_k-x_l}_\infty \leq \norm{x_k-x+x-x_l}_\infty \leq \norm{x_k-x}_\infty + \norm{x_l-x}_\infty < \frac{\varepsilon}{2} + \frac{\varepsilon}{2} = \varepsilon\]
	\end{dem}
\end{teo}
\begin{defn}[Punto fijo]
	Sea $\appl{T}{\mathcal{C}([a,b])}{\mathcal{C}([a,b])}$ un operador, $\appl{x}{[a,b]}{\R}$ es un punto fijo de $T \iff T[x]=x$.
\end{defn}
\begin{defn}[Contracción]
	Sea $C \subset X \we C \ne \phi$ donde $(X, \norm{\cdot})$ es un espacio vectorial normado y $\appl{T}{C}{C}$ una aplicación, $T$ es una contracción en $C$ \[\iff \exists \,\alpha \in (0,1) : \forall x, y \in C : \norm{T[x]-T[y]}\leq \alpha \norm{x-y}\]
\end{defn}
\begin{obs}
	Claramente, toda contracción es continua (con respecto a la norma correspondiente). En efecto, si $(x_k)_{k\in\N} \subset C : x_k\xrightarrow{k\to\infty} x \in C$
	\[\implies \norm{T[x_k]-T[x]}\leq \alpha \norm{x_k-x} \xrightarrow{k\to\infty} 0\]
\end{obs}
\begin{teo}[Punto fijo de Banach]
	Sea $C \subset X \we C\ne \phi$ cerrado donde $(X, \norm{\cdot})$ es un espacio normado completo y sea $\appl{T}{C}{C}$ una contracción $\implies \exists!\,x\in C : T[x]=x$
	\fecha{14/03/2024}
	\begin{dem}
		Sea $x_0 \in C$. Definimos la sucesión $\forall n \in \N : x_n\defeq T\left[x_{n-1}\right]$.
		\[\implies \norm{x_{n+1}-x_n} = \norm{T[x_n]+T[x_{n-1}]} \leq \alpha \norm{x_n - x_{n-1}} \leq \dots \leq \alpha^n \norm{x_1-x_0}\]
		Sean $n,m \in \N \we n>m$.
		\[\begin{aligned}
				\implies \lbox{\|x_n-x_m\|} & = \|(x_n-x_{n-1}) + (x_{n-1}-x_{n-2}) + \dots + (x_{m+1}-x_m)\|                                                            \\
				                            & \leq \|x_n-x_{n-1}\| + \|x_{n-1}-x_{n-2}\| + \dots + \| x_{m+1}-x_m\|                                                      \\
				                            & \leq \left(\alpha ^{n-1} + \alpha^{n-2} + \dots + \alpha^m \right) \|x_1-x_0\|                                             \\
				                            & \leq \alpha ^{n-1} (1+\alpha + \alpha ^2 + \dots + \alpha ^{m-n+1} ) \|x_1-x_0\|                                           \\
				                            & \leq \alpha^m \ds \left(\sum _{i=0}^{\infty} \alpha^i\right) \|x_1-x_0\| = \rbox{\ds \frac{\alpha^m}{1-\alpha}\|x_1-x_0\|}
			\end{aligned}\]
		\[\implies \forall \varepsilon >0 : \exists \kappa = \kappa (\varepsilon ) \in \N : \big(n>m \geq \kappa (\varepsilon )\implies \|x_n-x_m\| < \varepsilon\big) \implies \text{es de Cauchy}\]
		Por tanto, $\left(x_n\right)$ tiene límite, es decir, $\ds \exists x \in X : x = \lim_{n\to \infty} x_n$.\\
		Como $x_n \in C\we C$ es cerrado $\implies x \in C$.

		Hay que ver que $x$ es punto fijo de $T$. En efecto, tomando límites en $x_{n+1}=T[x_n]$ y usando que $T$ es continua por ser contractiva, se tiene
		\[x = \displaystyle \lim_{n \to \infty} x_n = \ds \lim_{x \to \infty} T[x_n] = T[x]\]

		Falta probar la unicidad. Para ello, sean $x,y \in C$ puntos fijos. Se tiene
		\[\|x-y\|=\|T[x]-T[y]\|\leq \alpha \|x-y\| \implies (1-\alpha)\|x-y\| \leq 0 \implies \|x-y\|= 0 \implies x=y\]
	\end{dem}
\end{teo}

Para poder aplicar el teorema de punto fijo a nuestra aplicación $T[x]$, hay que ver que ésta es contractiva:
\[\begin{aligned}
		\|T[x]-T[y]\|_{\infty} & = \max_{t \in [a,b]} \left | \int_{t_0}^t f(s,x(s))\odif{s} - \int_{t_0}^t f(s,y(s)) \odif{s} \right |                                                                  \\
		                       & \leq \max_{t\in [a,b]} \left ( \int_{t_0}^t \left | f(s,x(s))-f(s,y(s))\right | \odif{s} \right )                                                                       \\
		                       & \overset{(*)}{\leq} \max_{t\in[a,b]} \int_{t_0}^t L\left |x(s) - y(s) \right | \odif{s} \leq L\|x-y\|_{\infty} \left( \max_{t\in [a,b]} \int_{t_0}^t \odif{s} \right )= \\
		                       & = L \|x-y\|_{\infty} \max_{t\in[a,b]}(t-t_0) =L(b-t_0)\|x-y\|_{\infty} \leq L(b-a)\|x-y\|_{\infty}
	\end{aligned}\]
$(*) \iff t>t_0 \we \left |f(s,x)-f(s,y)\right | \leq L|x-y|$
\\ \hspace*{\fill} ($f$ es Lipschitz con respecto a la segunda variable y es uniforme con respecto a la primera)

Así concluimos que $\ds \frac{(b-a)}{L}<1 \implies $ T es una contracción.

$\hfill \blacksquare$


\fecha{18/03/2024}
\subsubsection{Funciones Lipschitz}
\begin{defn}[Función Lipschitz]
	Sea $\appl{f}{\Omega}{\R}$ una función con $\Omega\subset \R$ abierto, es Lipschitz en $\Omega$
	\[\iff\exists L \in \R : \forall x, \hat{x} \in \Omega : \abs{f(x)-f(\hat{x})}\leq L\abs{x-\hat{x}}\]
\end{defn}
\textbf{Interpretación geométrica: }Si $f$ es Lipschitz en $\Omega$, suponiendo que $x > \hat{x}$:
\[\abs{f(\hat{x})-f(x)}\leq L(x-\hat{x}) \implies - L(x-\hat{x}) \leq f(\hat{x})-f(x) \leq L(x-\hat{x}) \]
\[\implies f(\hat{x})-L(x-\hat{x}) \leq f(x) \leq f(\hat{x})+L(x-\hat{x})\] %TODO: gráfica / dibujo
Es como si en cada punto $(x, f(x))$ de la gráfica se abriese un ``cono'' (dos rectas de pendientes $L$ y $-L$) que contiene el resto de la gráfica para $\hat{x} > x$.
\begin{obs}
	\begin{enumerate}
		\item[]
		\item Si $f$ es Lipschitz en $\Omega$, entonces $f$ es uniformemente continua en $\Omega$. \\
		      \hspace*{\fill}(Sale al tomar $\ds \delta = \frac{\varepsilon}{L}$ en la definición de continuidad uniforme).
		\item A la constante $L$ se le llama constante Lipschitz de $f$ en $\Omega$.
		\item Funciones con derivada no acotada no pueden ser Lipschitz.
	\end{enumerate}
\end{obs}
\begin{ejem}
	\begin{enumerate}
		\item[]
		\item Sea $\appl{f}{\R}{\R}$ dada por $f(x)=\sin{x}$ y $x, \hat{x} \in \R$, $x<\hat{x}$. Por el TVM
		      \[\exists y \in (x, \hat{x}) : \abs{f(x) - f(\hat{x})} = f'(y) \abs{x-\hat{x}} \implies \abs{f(x) - f(\hat{x})} = \cos{(y)} \abs{x-\hat{x}} \leq \abs{x-\hat{x}}\]
		      Por tanto $f(x)=\sin{x}$ es Lipschitz en $\R$ con constante $L=1$.
		\item En realidad, si $\appl{f}{I\subset\R}{\R}$ es derivable con $\forall x \in I : \abs{f'(x)} \leq L \implies f$ es Lipschitz en $I$ con constante $L$.
	\end{enumerate}
\end{ejem}

\begin{prop}
	\begin{enumerate}
		\item[]
		\item Sean $\appl{f, g}{\R^m}{\R^d}$ dos funciones Lipschitz con constantes de Lipschitz $L_f, L_g$ y sean $\alpha, \beta \in \R \implies \alpha f + \beta g \tex{ es Lipschitz.}$
		\item Sean $\appl{f}{\R^m}{\R^l}$ y $\appl{f}{\R^l}{\R^d}$ dos funciones Lipschitz con constantes de Lipschitz $L_f, L_g \implies g\circ f \tex{ es Lipschitz con constante } L_fL_g$.
	\end{enumerate}
	\begin{dem}
		\begin{enumerate}
			\item[]
			\item Sean $x, \hat{x} \in \Omega$ y $\alpha, \beta \in \R$:
			      \[\begin{aligned}
					      \abs{\alpha f(x) + \beta g(x) - \left(\alpha f(\hat{x}) - \beta g(\hat{x})\right)} & = \abs{\alpha\left(f(x) - f(\hat{x})\right) + \beta\left(g(x) - g(\hat{x})\right)} \\
					                                                                                         & \leq \abs{\alpha}\abs{f(x) - f(\hat{x})} + \abs{\beta}\abs{g(x) - g(\hat{x})}      \\
					                                                                                         & \leq \left(\abs{\alpha}L_f + \abs{\beta}L_g \right)\abs{x-\hat{x}}
				      \end{aligned}\]
			\item Sean $x, \hat{x} \in \Omega$:
			      \[\abs{(g \circ f)(x) - (g \circ f)(\hat{x}) } = \abs{g(f(x)) - g(f(\hat{x}))} \leq L_g\abs{f(x)-f(\hat{x})} \leq L_gL_f\abs{x-\hat{x}}\]
		\end{enumerate}
	\end{dem}
\end{prop}
\begin{defn}[Función Lipschitz en la segunda variable]  \mbox{} \\
	Sea $\appl{f}{(a,b)\times \Omega \subset \R^n}{\R^d}$ una función, $f$ es Lipschitz respecto a su segunda variable, y uniforme con respecto a la primera
	\[\iff \exists L \in \R : \forall x, \hat{x} \in \Omega : |f(t,x)-f(t,\hat{x})| \leq L |x-\hat{x}|\]
\end{defn}

\subsection{Existencia y unicidad de soluciones}
\begin{teo}[Existencia y unicidad global de soluciones]
	Sea $\appl{f}{[a,b]\times \R^n}{\R^n}$ con $a, b \in \R \we a < b$ una función continua y Lipschitz en la segunda variable y uniforme en la primera
	\[\implies \forall t_0\in [a,b] : \forall x \in \R : \exists! \, x \in \mathcal{C}^1\left([a,b]\right) : \begin{cases}
			\forall t \in [a,b] : x'(t)=f(t, x(t)) \\
			x(t_0) = \hat{x}
		\end{cases}\]
	\begin{dem}
		Estamos buscando un punto fijo de $\ds T[x] = \hat{x} + \int_{t_0}^{t} f(s, x(s)) \odif{s}$.

		Sea $L$ la constante Lipschitz de $f$ en la segunda variable. Dividimos $[t_0, b]$ en subintervalos cerrados de longitud $< \sfrac{1}{L}$:
		\[[t_0, b] = [t_0, t_1] \cup [t_1, t_2] \cup \cdots \cup [t_{N-1}, t_N=b]\]
		Queremos encontrar una solución en $[t_0, t_1]$, para ello definimos $\appl{T_1}{\mathcal{C}([t_0, t_1])}{\mathcal{C}([t_0, t_1])}$
		\[\forall x \in \mathcal{C}([t_0, t_1]) : \forall t \in [t_0, t_1] : T_1[x](t) \defeq \hat{x} + \int_{t_0}^{t} f(s, x(s)) \odif{s}\]
		\[\begin{aligned}
				\implies \abs{T_1[x](t) - T_1[y](t)} & = \abs{\int_{t_0}^{t} f(s, x(s)) - f(s, y(s)) \odif{s}} \leq L\int_{t_0}^{t} \abs{x(s)-y(s)} \odif{s} \\
				                                     & \leq L(t-t_0)\norm{x-y}_\infty \leq L(t_1-t_0)\norm{x-y}_\infty
			\end{aligned}\]
		\[\implies \forall x, y \in \mathcal{C}([a,b])  \norm{T_1[x]-T_1[y]}_\infty \leq L(t_1-t_0)\norm{x-y}_\infty\]
		Por tanto, $T_1$ es una contracción con $\alpha = L(t_1-t_0) < 1$ en $\mathcal{C}([t_0, t_1])$ y por el teorema de punto fijo de Banach, $\exists! \, x_1 \in \mathcal{C}([t_0, t_1]) : T_1[x_1]=x_1$. Por tanto, existe una solución única al PVI en $[t_0, t_1]$.

		Ahora queremos encontrar una solución en $[t_1, t_2]$ que ``empiece donde acaba $x_1$''. Definimos $\appl{T_2}{\mathcal{C}([t_1, t_2])}{\mathcal{C}([t_1, t_2])}$ como $\ds T_2[x](t) \defeq x_1(t_1) + \int_{t_1}^{t} f(s, x(s)) \odif{s}$ y procedemos igual que antes para obtener $x_2$, la solución única en $[t_1, t_2]$.

		Repetimos este proceso $\left(\tex{sacando }\left(x_j\right)_{j=1}^{N}\right)$ hasta llegar a $b$ y obtenemos una única función $x \in \mathcal{C}^1([a,b])$ continua y derivable por construcción que es solución del PVI.
		\[\forall j \in \N_N : \forall t \in [t_{j-1}, t_j] : x(t) = x_j(t)\]
	\end{dem}
\end{teo}

\begin{cor}
	Sea $\appl{f}{\R\times \R^d}{\R^d}$ una función continua y Lipschitz en la segunda variable y uniforme en la primera
	\[implies \forall \hat{x} \in \R^d : \forall t_0 \in \R : \exists! \, x \in \mathcal{C}^1(\R) : \begin{cases}
			\forall t \in \R : x'(t) = f(t, x(t)) \\
			x(t_0) = \hat{x}
		\end{cases}\]
	\begin{dem}
		Podemos dividir $\R$ en intervalos cerrados de longitud $<\sfrac{1}{L}$ y aplicar el teorema anterior.
	\end{dem}
\end{cor}

\begin{defn}[Lipschitz local]
	Sea $\appl{f}{\Omega \subset \R^m}{\R^d}$ una función, $f$ es localmente Lipschitz $\iff \forall K \subset \Omega : \big.f\big|_K \tex{ es Lipschitz}$.
\end{defn}
\begin{defn}
	Sea $\appl{f}{\Omega \tex{ abto }\subset \R \times \R^m}{\R^d}$ una función, $f$ es localmente Lipschitz en la segunda variable y uniforme en la primera
	\[\iff \forall K \subset \Omega : \exists L_K \in \R : \forall (t, x), (\hat{t}, \hat{x}) \in K : \abs{f(t, x) - f(\hat{t}, \hat{x})} \leq L_K \abs{x-\hat{x}}\]
\end{defn}
\begin{teo}[Picard-Lindelöf]
	Sea $\appl{f}{\Omega \subset \R \times \R^d}{\R^d}$ una función continua y localmente Lipschitz en la segunda variable y uniforme en la primera
	\[\implies \forall (t_0, \hat{x}) \in \Omega : \exists \varepsilon > 0 : \exists! \, x \in \mathcal{C}^1([t_0-\varepsilon, t_0+\varepsilon]) : \begin{cases}
			\forall t \in [t_0-\varepsilon, t_0+\varepsilon] : x'(t) = f(t, x(t)) \\
			x(t_0) = \bar{x}
		\end{cases}\]
	\begin{dem}
		Como $\Omega$ es abierto, $\exists \rho, \delta > 0 : \mathcal{R}_{\rho, \delta} \defeq (t_0-\rho, t_0+\rho) \times B_{\delta}(\bar{x}) \subset \Omega$.

		Como $f$ es continua, $\ds \implies \abs{f(t, x)} \leq \max_{(t, x) \in \mathcal{R}_{\rho, \delta}} \abs{f(t, x)} \eqdef C_{\rho, \delta}$.

		Como $f$ es localmente Lipschitz en la segunda variable,
		\[\implies \exists L_{\rho, \delta} \in \R : \forall (t, x), (\hat{t}, \hat{x}) \in \mathcal{R}_{\rho, \delta} : \abs{f(t, x) - f(\hat{t}, \hat{x})} \leq L_{\rho, \delta} \abs{x-\hat{x}}\]

		Para cada $\varepsilon \in (0, \rho]$ consideramos el conjunto
		\[\mathcal{X}_{\rho, \delta} \defeq \big\{x \in \mathcal{C}([t_0-\varepsilon, t_0+\varepsilon]) : \forall t \in [t_0-\varepsilon, t_0 + \varepsilon] :x(t) \in B_\delta(\bar{x})\big\}\]
		Definimos $\appl{T}{\mathcal{X}_{\rho, \delta}}{\mathcal{C}([t_0-\varepsilon, t_0+\varepsilon])}$ como
		\[\forall x \in \mathcal{X}_{\rho, \delta} : \forall t \in [t_0-\varepsilon, t_0+\varepsilon] : T[x](t) \defeq \bar{x} + \int_{t_0}^{t} f(s, x(s)) \odif{s}\]

		Queremos un $\varepsilon > 0$ lo suficientemente pequeño para que $T$ sea una contracción en $\mathcal{X}_{\rho, \delta}$.
		\begin{enumerate}
			\item $\mathcal{X}_{\rho, \delta} \ne \phi \tex{ porque } \bar{x} \in \mathcal{X}_{\rho, \delta}$
			\item Sea $(x_k) \subset \mathcal{X}_{\rho, \delta}$ una sucesión $: x_k \xrightarrow{k\to \infty} x \in \mathcal{C}([t_0-\varepsilon, t_0+\varepsilon])$
			      \[\implies \forall t \in [t_0-\varepsilon, t_0+\varepsilon] : \abs{x_k(t) - \bar{x}} \leq \delta \implies \forall t \in [t_0-\varepsilon, t_0+\varepsilon] : \abs{x(t) - \bar{x}} \leq \delta\]
			      \[\implies \forall t \in [t_0-\varepsilon, t_0+\varepsilon] : x(t) \in B_\delta(\bar{x}) \implies x \in \mathcal{X}_{\rho, \delta}\]
			\item Hay que ver que $T[\mathcal{X}_{\rho, \delta}] \subset \mathcal{X}_{\rho, \delta}$. Como $\forall x \in \mathcal{X}_{\rho, \delta} : \forall t \in [t_0-\varepsilon, t_0+\varepsilon] :$
			      \[\begin{aligned}
					      \abs{T[x](t) - \bar{x}} & = \abs{\int_{t_0}^{t} f(s, x(s)) \odif{s}} \leq \int_{\min\{t_0,t_1\}}^{\max\{t_0, t_1\}} \abs{f(s, x(s))} \odif{s}                                                              \\
					                              & \leq \int_{t_0}^{t} C_{\rho, \delta} \odif{s} = C_{\rho, \delta} \abs{t-t_0} \leq \varepsilon C_{\rho, \delta} \leq \delta \iff \varepsilon \leq \frac{\delta}{C_{\rho, \delta}}
				      \end{aligned}\]
			\item Veamos que $T$ es una contracción en $\mathcal{X}_{\rho, \delta}$:
			      \[\begin{aligned}
					      \abs{T[x](t) - T[\hat{x}](t)} & = \abs{\int_{t_0}^{t} f(s, x(s)) - f(s, \hat{x}(s)) \odif{s}} \leq L_{\rho, \delta} \abs{t - t_0} \leq \varepsilon L_{\rho, \delta} \abs{x - \hat{x}} \\
						  & \leq \abs{x - \hat{x}} \iff \varepsilon < \frac{1}{L_{\rho, \delta}}
				      \end{aligned}\]
		\end{enumerate}
		Por tanto, $T$ es contractiva en $\mathcal{X}_{\rho, \delta}$ y por el teorema de punto fijo de Banach, 
		\[\exists! \, x \in \mathcal{X}_{\rho, \delta} : x \tex{ es solución del PVI.}\]
	\end{dem}
\end{teo}