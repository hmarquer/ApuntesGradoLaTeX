\fecha{11/03/2024}

\section{Tema 3: Teoremas fundamentales}

\subsection{Introducción}

Sean $I\subset \R$, $\Omega \subset \R$ dos intervalos abiertos y $\appl{f}{I\times \Omega}{\R^2}$ una función continua. Consideramos el PVI $\begin{cases}
		x'=f(t, x(t)) , & t\in I            \\
		x(t_0)=\hat{x}, & \hat{x}\in \Omega
	\end{cases}$. Recordamos que si $f(t, x)=f(x)$, entonces el PVI tiene solución en un entorno de $\ds t_0$: $\ds x(t) = \hat{x} + \int_{t_0}^{t} f(s, x(s)) \odif{s}$.

Definimos el operador $\ds \forall x \in \mathcal{C}(I, \Omega) : T[x]\defeq x + \int_{t_0}^{t} f(s, x(s)) \odif{s} \implies x=T[x]$.
\[x_0 = \hat{x} \we x_1 = T[x_0] = \hat{x} + \int_{t_0}^{t} f(s, \hat{x}) \odif{s} \we \cdots \we x_{k+1} = T[x_k]= \hat{x} + \int_{t_0}^{t} f(s, x_k(s)) \odif{s}\]
\begin{ejem}
	$\ds\{x'=x \we x(0)=1\} \iff x(t)=1+\int_{0}^{t} x(s) \odif{s}$
	\[\implies x_0 = 1 \we x_1 = 1 + t \we \cdots x_k = \sum_{i=0}^{k} \frac{t^i}{i!} \implies x_k(t)\xrightarrow{k\to\infty} e^t\]
\end{ejem}

Pero necesitamos formalizar todo esto.
\begin{enumerate}
	\item Concepto de límite de series de funciones.
	\item ¿Toda sucesión de Cauchy es convergente?
	\item $\ds \lim_{k\to\infty} \int_{t_0}^t f(s, x_k(s)) \odif{s} \stackrel{?}{=} \int_{t_0}^t \lim_{k\to\infty} f(s, x_k(s)) \odif{s}$
	\item $\ds \lim_{k\to\infty} f(s, x_k(s)) \stackrel{?}{=} f\left(s, \lim_{k\to\infty} x_k(s)\right)$
\end{enumerate}

\subsection{Conceptos de análisis}
\subsubsection{Convergencia puntual y uniforme}
\begin{defn}[Convergencia puntual]
	Sea $(f_k)_{k\in\N}$ una sucesión de funciones $\appl{f_k}{I}{\R}$ con $I\subset \R$ abierto, $(f_k)$ converge puntualmente a $f$
	\[\iff \forall t\in I : \lim_{k\to\infty} f_k(t) = f(t)\]
	Es decir, $\forall \varepsilon > 0 : \forall t\in I : \exists \kappa \in \N : \forall k\geq \kappa : \abs{f_k(t) - f(t)} < \varepsilon$.
\end{defn}

\begin{ejem}
	La sucesión $(f_k)$ definida a continuación converge puntualmente a $f$ pero su límite es una función no continua.
	\[\forall t \in\R : f_k(t) \defeq \begin{cases}
			0,                & t < -\frac{1}{k}        \\
			k(t+\frac{1}{k}), & -\frac{1}{k} \leq t < 0 \\
			k(\frac{1}{k}-t), & 0\leq t < \frac{1}{k}   \\
			0,                & t\geq \frac{1}{k}
		\end{cases} \implies f_k(t)\xrightarrow{k\to\infty}\begin{cases}
			0, & t\ne 0 \\
			1, & t=0
		\end{cases}\]
\end{ejem}
\fecha{12/03/2024}
\begin{ejem}
	\[x_k(t) \defeq \begin{cases}
			2k^2t,                          & t\in\left[0, \frac{1}{2k}\right)           \\
			2k^2\left(\frac{1}{k}-t\right), & t\in\left[\frac{1}{2k}, \frac{1}{k}\right) \\
			0,                              & t\in\left[\frac{1}{k}, 1\right)            \\
		\end{cases}\implies \begin{aligned}
			 & x_k(0)=0 \we x_k(1)=0                                  \\
			 & t\in (0,1] \implies x_k(t)  \xrightarrow{k\to\infty} 0
		\end{aligned}\]
	$\ds \lim_{k\to\infty} \int_{0}^1 x_k(t) \odif{t} \stackrel{?}{=} \int_{0}^1 \lim_{k\to\infty} x_k(t) \odif{t}$:
	\[ \int_{0}^1 \lim_{k\to\infty} x_k(t) \odif{t} = \int_{0}^{1} 0 \odif{t} = 0 \ne \frac{1}{2} = \lim_{k\to\infty}\int_{0}^{1} x_k(t) \odif{t}\]
\end{ejem}
\begin{defn}[Convergencia uniforme]
	Sea $(f_k)_{k\in\N}$ una sucesión de funciones $\appl{f_k}{I}{\R}$ con $I\subset \R$ abierto, $(f_k)$ converge uniformemente a $f$
	\[\iff \forall \varepsilon > 0 : \exists \kappa \in \N : \forall k\geq \kappa : \forall t\in I : \abs{f_k(t) - f(t)} < \varepsilon\]
	Es decir, $\ds \forall \varepsilon > 0 : \exists \kappa \in \N : \forall k \geq \kappa : \sup_{t\in I} \abs{x_k(t)-x(t)} \leq \varepsilon$
\end{defn}
\begin{obs}\begin{itemize}
		\item Los dos ejemplos anteriores no convergen uniformemente.
		\item La convergencia uniforme implica convergencia puntual.
	\end{itemize}\end{obs}
\begin{prop}
	Sea $(x_k)_{k\in\N}$ una sucesión de funciones uniformemente convergente a $x$.
	\[\implies x \tex{ continua}\]
	\begin{dem}
		Sea $\varepsilon > 0$, $t_0 \in I$. Buscamos $\delta > 0$ tal que, dado $t\in I$
		\[\abs{t-t_0} < \delta \implies \abs{x(t)-x(t)}< \varepsilon\]
		Para cada $k\in\N$, $\abs{x(t)-x_k(t)+x_k(t)-x_k(t_0)+x_k(t_0)-x_k(t_0)} \leq$
		\[\leq \abs{x(t)-x_k(t)} + \abs{x_k(t)-x_k(t_0)}  + \abs{x_k(t_0)-x(t_0)}\]
		Como $(x_k)$ converge uniformemente a $x$, $\kappa = \kappa(\varepsilon) \in \N$ tal que
		\[\forall \varepsilon > 0 : \exists \kappa \in \N : \forall k\geq \kappa : \forall t\in I : \abs{x_k(t) - x(t)} < \frac{\varepsilon}{3}\]
		Entonces, para $k=\kappa$, se tiene
		\[\begin{aligned}
			\abs{x(t)-x(t_0)} &\leq \abs{x(t)-x(t_0)}+\abs{x(t)-x_\kappa(t_0)} + \abs{x_\kappa(t_0)-x(t_0)} \\
			&\leq \frac{\varepsilon}{3} + \abs{x_\kappa(t)-x_\kappa(t_0)} + \frac{\varepsilon}{3} 
		\end{aligned}\]
		Como $x_\kappa$ es continua, $\exists \delta > 0 : \abs{t-t_0} < \delta \implies \abs{x_\kappa(t)-x_\kappa(t_0)} < \frac{\varepsilon}{3}$. Entonces, $\abs{x(t)-x(t_0)} < \varepsilon$.
	\end{dem}
\end{prop}
\begin{prop}
	Sea $(x_k)_{k\in\N}$ una sucesión de funciones continuas $\appl{x_i}{(a,b)}{\R}$ que converge uniformemente a $\appl{x}{(a,b)}{\R}$.
	\[\implies \lim_{k\to\infty} \int_{a}^{b} x_k(t) \odif{t} = \int_{a}^{b} x(t) \odif{t}\]
\end{prop}