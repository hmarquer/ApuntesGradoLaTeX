\fecha{11/03/2024}

\section{Tema 3: Teoremas fundamentales}

\subsection{Introducción}

Sean $I\subset \R$, $\Omega \subset \R$ dos intervalos abiertos y $\appl{f}{I\times \Omega}{\R^2}$ una función continua. Consideramos el PVI $\begin{cases}
		x'=f(t, x(t)) , & t\in I            \\
		x(t_0)=\hat{x}, & \hat{x}\in \Omega
	\end{cases}$. Recordamos que si $f(t, x)=f(x)$, entonces el PVI tiene solución en un entorno de $\ds t_0$: $\ds x(t) = \hat{x} + \int_{t_0}^{t} f(s, x(s)) \odif{s}$.

Definimos el operador $\ds \forall x \in \mathcal{C}(I, \Omega) : T[x]\defeq x + \int_{t_0}^{t} f(s, x(s)) \odif{s} \implies x=T[x]$.
\[x_0 = \hat{x} \we x_1 = T[x_0] = \hat{x} + \int_{t_0}^{t} f(s, \hat{x}) \odif{s} \we \cdots \we x_{k+1} = T[x_k]= \hat{x} + \int_{t_0}^{t} f(s, x_k(s)) \odif{s}\]
\begin{ejem}
	$\ds\{x'=x \we x(0)=1\} \iff x(t)=1+\int_{0}^{t} x(s) \odif{s}$
	\[\implies x_0 = 1 \we x_1 = 1 + t \we \cdots x_k = \sum_{i=0}^{k} \frac{t^i}{i!} \implies x_k(t)\xrightarrow{k\to\infty} e^t\]
\end{ejem}

Pero necesitamos formalizar todo esto.
\begin{enumerate}
	\item Concepto de límite de series de funciones.
	\item ¿Toda sucesión de Cauchy es convergente?
	\item $\ds \lim_{k\to\infty} \int_{t_0}^t f(s, x_k(s)) \odif{s} \stackrel{?}{=} \int_{t_0}^t \lim_{k\to\infty} f(s, x_k(s)) \odif{s}$
	\item $\ds \lim_{k\to\infty} f(s, x_k(s)) \stackrel{?}{=} f\left(s, \lim_{k\to\infty} x_k(s)\right)$
\end{enumerate}

\subsection{Conceptos de análisis}
\subsubsection{Convergencia puntual y uniforme}
\begin{defn}[Convergencia puntual]
	Sea $(f_k)_{k\in\N}$ una sucesión de funciones $\appl{f_k}{I}{\R}$ con $I\subset \R$ abierto, $(f_k)$ converge puntualmente a $f$
	\[\iff \forall t\in I : \lim_{k\to\infty} f_k(t) = f(t)\]
	Es decir, $\forall \varepsilon > 0 : \forall t\in I : \exists \kappa \in \N : \forall k\geq \kappa : \abs{f_k(t) - f(t)} < \varepsilon$.
\end{defn}

\begin{ejem} %TODO: gráfica
	La sucesión $(f_k)$ definida a continuación converge puntualmente a $f$ pero su límite es una función no continua.
	\[\forall t \in\R : f_k(t) \defeq \begin{cases}
			0,                & t < -\frac{1}{k}        \\
			k(t+\frac{1}{k}), & -\frac{1}{k} \leq t < 0 \\
			k(\frac{1}{k}-t), & 0\leq t < \frac{1}{k}   \\
			0,                & t\geq \frac{1}{k}
		\end{cases} \implies f_k(t)\xrightarrow{k\to\infty}\begin{cases}
			0, & t\ne 0 \\
			1, & t=0
		\end{cases}\]
\end{ejem}
\fecha{12/03/2024}
\begin{ejem} %TODO: gráfica
	\[x_k(t) \defeq \begin{cases}
			2k^2t,                          & t\in\left[0, \frac{1}{2k}\right)           \\
			2k^2\left(\frac{1}{k}-t\right), & t\in\left[\frac{1}{2k}, \frac{1}{k}\right) \\
			0,                              & t\in\left[\frac{1}{k}, 1\right)            \\
		\end{cases}\implies \begin{aligned}
			 & x_k(0)=0 \we x_k(1)=0                                  \\
			 & t\in (0,1] \implies x_k(t)  \xrightarrow{k\to\infty} 0
		\end{aligned}\]
	$\ds \lim_{k\to\infty} \int_{0}^1 x_k(t) \odif{t} \stackrel{?}{=} \int_{0}^1 \lim_{k\to\infty} x_k(t) \odif{t}$:
	\[ \int_{0}^1 \lim_{k\to\infty} x_k(t) \odif{t} = \int_{0}^{1} 0 \odif{t} = 0 \ne \frac{1}{2} = \lim_{k\to\infty}\int_{0}^{1} x_k(t) \odif{t}\]
\end{ejem}
\begin{defn}[Convergencia uniforme]
	Sea $(f_k)_{k\in\N}$ una sucesión de funciones $\appl{f_k}{I}{\R}$ con $I\subset \R$ abierto, $(f_k)$ converge uniformemente a $f$
	\[\iff \forall \varepsilon > 0 : \exists \kappa \in \N : \forall k\geq \kappa : \forall t\in I : \abs{f_k(t) - f(t)} < \varepsilon\]
	Es decir, $\ds \forall \varepsilon > 0 : \exists \kappa \in \N : \forall k \geq \kappa : \sup_{t\in I} \abs{x_k(t)-x(t)} \leq \varepsilon$
\end{defn}
\begin{obs}\begin{itemize}
		\item Los dos ejemplos anteriores no convergen uniformemente.
		\item La convergencia uniforme implica convergencia puntual.
	\end{itemize}\end{obs}
\begin{prop}
	Sea $(x_k)_{k\in\N}$ una sucesión de funciones uniformemente convergente a $x$.
	\[\implies x \tex{ continua}\]
	\begin{dem}
		Sea $\varepsilon > 0$, $t_0 \in I$. Buscamos $\delta > 0$ tal que, dado $t\in I$
		\[\abs{t-t_0} < \delta \implies \abs{x(t)-x(t)}< \varepsilon\]
		Para cada $k\in\N$, $\abs{x(t)-x_k(t)+x_k(t)-x_k(t_0)+x_k(t_0)-x_k(t_0)} \leq$
		\[\leq \abs{x(t)-x_k(t)} + \abs{x_k(t)-x_k(t_0)}  + \abs{x_k(t_0)-x(t_0)}\]
		Como $(x_k)$ converge uniformemente a $x$, $\kappa = \kappa(\varepsilon) \in \N$ tal que
		\[\forall \varepsilon > 0 : \exists \kappa \in \N : \forall k\geq \kappa : \forall t\in I : \abs{x_k(t) - x(t)} < \frac{\varepsilon}{3}\]
		Entonces, para $k=\kappa$, se tiene
		\[\begin{aligned}
			\abs{x(t)-x(t_0)} &\leq \abs{x(t)-x(t_0)}+\abs{x(t)-x_\kappa(t_0)} + \abs{x_\kappa(t_0)-x(t_0)} \\
			&\leq \frac{\varepsilon}{3} + \abs{x_\kappa(t)-x_\kappa(t_0)} + \frac{\varepsilon}{3} 
		\end{aligned}\]
		Como $x_\kappa$ es continua, $\exists \delta > 0 : \abs{t-t_0} < \delta \implies \abs{x_\kappa(t)-x_\kappa(t_0)} < \frac{\varepsilon}{3}$. Entonces, $\abs{x(t)-x(t_0)} < \varepsilon$.
	\end{dem}
\end{prop}
\begin{prop}
	Sea $(x_k)_{k\in\N}$ una sucesión de funciones continuas $\appl{x_i}{(a,b)}{\R}$ que converge uniformemente a $\appl{x}{(a,b)}{\R}$ donde $(a,b) \subset \R$ está acotado.
	\[\implies \lim_{k\to\infty} \int_{a}^{b} x_k(t) \odif{t} = \int_{a}^{b} x(t) \odif{t}\]
\end{prop}

\fecha{13/03/2024}
\begin{ejem} %TODO: gráfica
	Consideramos la sucesión $x_k(t) = \begin{cases}
		\frac{1}{k} & \tex{ si } t\leq k \\
		0 & \tex{ si } t>k 
	\end{cases}$
	\[\implies \int_{0}^{\infty} x_k(t) \odif{t} = \int_{0}^{k} \frac{1}{k} \odif{t} = 1 \we \lim_{k\to\infty} x_k(t) = 0\]
\end{ejem}

Sea $\mathcal{C}([a,b], \R)=\mathcal{C}([a,b]) \defeq \left\{\appl{x}{[a,b]}{\R} : x\tex{ continua}\right\}$ el espacio vectorial de funciones continuas.
\[\forall x \in \mathcal{C}([a,b]) : \norm{x}_{\infty} \defeq \max_{t\in [a,b]} \abs{x(t)}\]
\[\implies \big(\mathcal{C}([a,b]), \norm{\cdot}_\infty\big) \tex{ es un espacio vectorial normado }\]
\[\implies \tex{también es métrico con} \operatorname{d}(x,y)\defeq\norm{x-y}_\infty\]
\[\operatorname{d}(x_k, x) = \norm{x_k-x}_\infty=\max_{t\in [a,b]} \abs{x_k(t)-x(t)} \rightarrow0 \iff x_k\to x \tex{ unif.}\]
\begin{teo}
	Sean $a, b \in \R : a<b \implies \ds \bigg(\mathcal{C}([a,b]), \norm{\cdot}_\infty\bigg)$ es un espacio completo.\\Es decir, toda sucesión de Cauchy es convergente.
	\begin{dem}
		Sea $(x_k)_{k\in \N}\subset \mathcal([a,b])$ una sucesión de Cauchy. Esto significa que
		\[\forall \varepsilon > 0 : \exists \kappa \in \N : \big(k, l\geq \kappa \implies \norm{k_k-x_l}_\infty <\varepsilon\big) \]
		Recordemos que $\norm{x_k-x_l}_\infty = \max_{t\in[a,b]}\abs{x_k(t)-x_l(t)}$\\
		Fijamos $\varepsilon > 0$ y $t\in [a,b]$
		\[\implies \forall k, l \geq \kappa : \abs{k_k(t) - x_l(t)} \leq \norm{x_k-x_l}_\infty < \varepsilon\]
		Esto demuestra que la sucesión de números reales $(x_k(t))_{k\in\N}$ es de Cauchy. Como $\R$ es completo, $\exists x(t) \in \R : x_k(t) \xrightarrow{k\to\infty} x(t)$. \\
		Veamos que el límite es uniforme: En efecto, sean $k\geq \kappa, l \geq 1, k, l \in \N$.
		\[\implies \forall t \in [a,b] : \abs{x_k(t)-x_{k+l}(t)} \leq \norm{x_k-x_{k+l}} < \varepsilon \tex{ porque $(x_k)$ es de Cauchy}\]
		Haciendo tender $l\to\infty$
		\[\implies \forall t \in [a,b] : \forall k \geq \kappa : \abs{x_k(t)-x(t)}\leq \varepsilon\]
		Recordemos que $x\in\mathcal{C}([a,b])$ por ser límite de funciones continuas.
	\end{dem}
\end{teo}
\begin{defn}[Punto fijo]
	Sea $\appl{T}{\mathcal{C}([a,b])}{\mathcal{C}([a,b])}$ un operador, $\appl{x}{[a,b]}{\R}$ es un punto fijo de $T \iff T[x]=x$.
\end{defn}
\begin{defn}[Contracción]
	Sea $C \subset X \we C \ne \phi$ donde $(X, \norm{\cdot})$ es un espacio vectorial normado y $\appl{T}{C}{C}$ una aplicación, $T$ es una contracción en $C$ \[\iff \exists \alpha \in (0,1) : \forall x, y \in C : \norm{T[x]-T[y]}\leq \alpha \norm{x-y}\]
\end{defn}
\begin{teo}
	Sea $C \subset X\we C\ne \phi$ cerrado donde $(X, \norm{\cdot})$ es un espacio normado completo y $\appl{T}{C}{C}$ una contracción $\implies \exists! x\in C : T[x]=x$
	\begin{dem}

	\end{dem}
\end{teo}
\begin{obs}
	Claramente, toda contracción es continua (con respecto a la norma correspondiente). En efecto, si $(x_k)_{k\in\N} \subset C : x_k\xrightarrow{k\to\infty} x \in C$
	\[\implies \norm{T[x_k]-T[x]}\leq \alpha \norm{x_k-x} \xrightarrow{k\to\infty} 0\]
\end{obs}

\fecha{14/03/2024}
\fecha{18/03/2024}
\subsubsection{Funciones Lipschitz}
\begin{defn}[Función Lipschitz]
	Sea $\appl{f}{\Omega}{\R}$ una función con $\Omega\subset \R$ abierto, es Lipschitz en $\Omega$ 
	\[\iff\exists L \in \R : \forall x, \hat{x} \in \Omega : \abs{f(x)-f(\hat{x})}\leq L\abs{x-\hat{x}}\]
\end{defn}
Si $f$ es Lipschitz en $\Omega$, suponiendo que $x > \hat{x}$:
\[\abs{f(\hat{x})-f(x)}\leq L(x-\hat{x}) \implies - L(x-\hat{x}) \leq f(\hat{x})-f(x) \leq L(x-\hat{x}) \]
\[\implies f(\hat{x})-L(x-\hat{x}) \leq f(x) \leq f(\hat{x})+L(x-\hat{x})\] %TODO: gráfica / dibujo

\begin{obs}
\begin{enumerate}
	\item[]
	\item Si $f$ es Lipschitz en $\Omega$, entonces $f$ es uniformemente continua en $\Omega$. (Sale al tomar $\ds \delta = \frac{\varepsilon}{L}$ en la definición de continuidad uniforme).
	\item A la constante $L$ se le llama constante Lipschitz de $f$ en $\Omega$.
\end{enumerate}
\end{obs}
\begin{ejem}
	\begin{enumerate}
		\item[]
		\item Sea $\appl{f}{\R}{\R}$ dada por $f(x)=\sin{x}$ y $x, \hat{x} \in \R$, $x<\hat{x}$. Por el TVM
		\[\exists y \in (x, \hat{x}) : \abs{f(x) - f(\hat{x})} = f'(y) \abs{x-\hat{x}} \implies \abs{f(x) - f(\hat{x})} = \cos{(y)} \abs{x-\hat{x}} \leq \abs{x-\hat{x}}\]
		Por tanto $f(x)=\sin{x}$ es Lipschitz en $\R$ con constante $L=1$.
		\item En realidad, si $\appl{f}{I\subset\R}{\R}$ es derivable con $\forall x \in I : \abs{f'(x)} \leq L \implies f$ es Lipschitz en $I$ con constante $L$.   
	\end{enumerate}
\end{ejem}

\begin{prop}
\begin{enumerate}
	\item[]
	\item Sean $\appl{f, g}{\R^m}{\R^d}$ dos funciones Lipschitz con constantes de Lipschitz $L_f, L_g$ y sean $\alpha, \beta \in \R \implies \alpha f + \beta g \tex{ es Lipschitz.}$
	\item Sean $\appl{f}{\R^m}{\R^l}$ y $\appl{f}{\R^l}{\R^d}$ dos funciones Lipschitz con constantes de Lipschitz $L_f, L_g \implies g\circ f \tex{ es Lipschitz con constante } L_fL_g$.
\end{enumerate}
\begin{dem}
\begin{enumerate}
	\item[]
	\item Sean $x, \hat{x} \in \Omega$ y $\alpha, \beta \in \R$:
	\[\begin{aligned}
		\abs{\alpha f(x) + \beta g(x) - \left(\alpha f(\hat{x}) - \beta g(\hat{x})\right)} &= \abs{\alpha\left(f(x) - f(\hat{x})\right) + \beta\left(g(x) - g(\hat{x})\right)} \\
		&\leq \abs{\alpha}\abs{f(x) - f(\hat{x})} + \abs{\beta}\abs{g(x) - g(\hat{x})} \\
		&\leq \left(\abs{\alpha}L_f + \abs{\beta}L_g \right)\abs{x-\hat{x}} 
	\end{aligned}\]
	\item Sean $x, \hat{x} \in \Omega$:
	\[\abs{(g \circ f)(x) - (g \circ f)(\hat{x}) } = \abs{g(f(x)) - g(f(\hat{x}))} \leq L_g\abs{f(x)-f(\hat{x})} \leq L_gL_f\abs{x-\hat{x}}\]
\end{enumerate}
\end{dem}
\end{prop}