\section{Ejercicios}

%\renewcommand{\theenumi}{(\alph{enumi})}

\subsection{Hoja 1}

\subsubsection{Conceptos básicos}

\subsubsection{Algunos métodos de resolución de EDOs}
\textbf{2.18} \begin{enumerate}
	\item $yy'' + {(y')}^2=0$ \\
	Resulta razonable buscar soluciones en forma de polinomios $y(x)=x^\alpha$ porque:
	\[\implies y'=\alpha x ^{(\alpha-1)} \we y''=\alpha(\alpha-1)x^{(\alpha-2)}\]
	\[\implies x^{\alpha}\alpha(\alpha-1)x^{(\alpha-2)} + \alpha^2 x ^{2(\alpha-1)} = 0\implies (2\alpha^2-\alpha)x^\alpha=0\]
	\[\implies 2\alpha^2-\alpha=0 \implies \alpha(2\alpha-1)=0 \implies \alpha=0 \ve \alpha=\frac{1}{2}\]
	Opción 1: Integramos la EDO:%\@
	\[\int_{0}^{t} y(s) y''(s)\odif{s} + \int_{0}^{t} {(y'(s))}^2 \odif{s} = 0\]
	\[-\int_{0}^t {(y'(s))}^2 \odif{s} + {[y(s)y'(s)]}^t_{s=0} + \int_{0}^{t} {(y'(s))}^2 \odif{s}\]
	\[\implies y(t)y'(t)-y(0)y'(0)=0 \implies y(t)y'(t)=y(0)y'(0)\eqdef C\]
	\[\implies \int_{0}^{t} y(s)y'(s)\odif{s}=Ct \implies \frac{{(y(t))}^2}{2}  -\frac{{(y(0))}^2}{2}=Ct\]
	\[\implies y(t)=\sqrt{2Ct+{(y(0))}^2}\]
	\item $xy''=y'+{(y')}^2$ \\
	No depende de $y \implies$ Hacemos un cambio de variable $x=y'$:
	\[\implies xz'=z+z^3 \tex{ que es de variables separadas.}\]
	\item $x^2y''=2xy'+{(y')}^2$ \\
	Nuevamente hacemos un cambio de variable $z=y' \implies x^2z'=2xz+z^2$
	\[\forall x \ne 0 : z'=2\frac{z}{x} + {\left(\frac{z}{x}\right)}^2 \implies \tex{ mediante el cambio de variables } \omega = \frac{z}{x}\]
	Obtenemos una EDO de variables separadas en $\omega$.
	\item $2yy''-{(y')}^2=1$ \\
	Otra vez resulta razonable buscar soluciones de la forma $y(x)=Ax^2 + Bx +C$
\end{enumerate}

\subsubsection{Modelización}

\textbf{3.4} $C(t)=\tex{``Cantidad de sal''}$ en el tanque en el tiempo $t$. 
\[C'(t)=10-\frac{1}{10}C(t) \implies \int_{C(0)}^{C(t)} \frac{1}{100-y}\odif{y} = \int_{0}^{t}\frac{1}{10}\odif{t}\]
\[\implies \log{(100-C(t))} - \log{(100-C(0))}=-\frac{1}{10}\implies \log{\frac{100-C(t)}{100}}=-\frac{t}{10}\]
\[\implies \frac{100-C(t)}{100}=e^{-\frac{t}{10}}\implies C(t)=100(1-e^{-\frac{t}{10}})\]
\[\implies C(1)=100(1-e^{-\frac{1}{10}}) \we \lim_{t\to \infty} C(t) = 100\]

% \textbf{3.7} \begin{enumerate}
% 	\item $xy=C \implies y(x)=\frac{C}{x}$ \\
% 	\begin{enumerate}
% 		\item Sacamos la EDO asociada $\odv{}{x}(xy(x))=\odv{}{x}(C)$
% 		\[\implies y(x) + xy'(x) = 0 \implies y'(x) = -\frac{y(x)}{x}=f(x,y(x))\]
% 		\item Las soluciones de $z'=-\frac{1}{f(x,y(x))}$ son ortogonales a las soluciones de la EDO $y'=f(x,y(x))$.
% 		\item \[z'=-\frac{1}{\frac{z}{x}}=\frac{x}{z} \implies zz'=x \implies \left(\frac{}{}\right)^2\]
% 	\end{enumerate}
% \end{enumerate}
\subsubsection{Análisis cualitativo y campos de pendientes}

\textbf{4.6} $\ds \forall t > \frac{5}{4} : \forall x\left(\frac{5}{4}\right) \in \left(-\sqrt{\frac{5}{4}}, -\frac{1}{2}\right) : x'=x^2-t \implies -\sqrt{t} < x(t) < -\sqrt{t-1}$
\[f_1(t)\defeq-\sqrt{t} \we f_2 \defeq -\sqrt{t-1} \implies f\left(\frac{5}{4}\right)=-\sqrt{\frac{5}{4}} \we f_2\left(\frac{5}{4} - 1\right)=-\frac{1}{2}\]
Sabemos que $\ds \tilde{t} = \frac{5}{4} \implies -\sqrt{\tilde{t}} < x(\tilde{t}) < -\sqrt{\tilde{t}-1}$
\[\implies \tex{Por contunuidad, al menos en un tiempo, estas cotas se siguen manteniendo.}\]
Atendiendo a las isoclinas de este ejercicio $(\{x^2-t=C : C\in \R\})$, observamos que:
\begin{itemize}
	\item $C=0 \implies x^2=t \implies x=\pm\sqrt{t}$ que es precisamente la cota inferior que buscábamos.
	\item $C=-1 \implies x^2-t=-1 \implies x=\pm\sqrt{t-1}$ que es la cota superior.
\end{itemize}
Inicialmente $x(t)>-\sqrt{t}$ ``durante un rato''.
\begin{enumerate}
	\item Supongamos que $\exists t^* : x(t^*)=-\sqrt{t^*}$.
	\item Por un lado, la isoclina nos dice que $x'(t^*)=0$.
	\item Por otro lado, $\ds x'(t^*)\leq {\left[\odv{}{t} (-\sqrt{t})\right]}_{t=t^*} = -\frac{1}{2}\frac{1}{\sqrt{t^*}}<0$.
\end{enumerate}

\textbf{4.7} $\ds \begin{cases}
	x' = x^2 + t^2, t >0 \\
	x(0)>0
\end{cases}$ Sea $\appl{x}{[0, T)}{\R}$ derivable.
\begin{enumerate}
	\item Queremos ver si $\ds \forall t \in [0, T) : x(t)> \frac{t^3}{3}$ \\
	Como $\ds x'\geq t^2 \implies \int_{0}^{t} x'(s) \odif{s} \geq \int_{0}^{t} s^2 \odif{s} \implies x(t)-x(0)\geq \frac{t^3}{3}$
	\[\implies \boxed{\forall t \in [0, T) : x(t)>\frac{t^3}{3}}\]
	\item Queremos ver si $\ds \forall t \in \left(\sqrt{3}, T\right), T>\sqrt{3} : x(t)> \frac{1}{\frac{4}{\sqrt{3}}-t}$ \\
	Como $\ds x'\geq x^2 \implies x^{-2}x'\geq 1 \implies \int_{\sqrt{3}}^{t} \frac{x'(s)}{{x(s)}^2} \odif{s} \geq t-\sqrt{3}$
	\[\implies -\frac{1}{x(t)}+ \frac{1}{x(\sqrt{3})} \geq t-\sqrt{3} \implies t \leq \sqrt{3} +\frac{1}{x(\sqrt{3})}- \frac{1}{x(t)}\]
	\[\implies t< \sqrt{3} + \frac{3}{{\left(\sqrt{3}\right)}^3} - \frac{1}{x(t)}= \sqrt{3} + \frac{1}{\sqrt{3}}- \frac{1}{x(t)}=\frac{4}{\sqrt{3}}- \frac{1}{x(t)}\]
	\[\implies t < \frac{4}{\sqrt{3}}-\frac{1}{x(t)} \implies \frac{1}{x(t)} < \frac{4}{\sqrt{3}}-t \implies \boxed{\forall t > \sqrt{3} : x(t) > \frac{1}{\frac{4}{\sqrt{3}}-t}}\]
\end{enumerate}

\subsection{Hoja 2}

\textbf{1.1} $\ds \begin{cases}
	x' = f(x) \\
	x(t_0) = x_0
\end{cases}$ tiene sol única.
\begin{enumerate}
	\item Toda solución que no sea constante es estrictamente monótona.
	\begin{dem}
		Por contrarecíproco, veamos que
		\[\ds \exists t^* : x'(t)=0 \implies x(t)\equiv C\defeq x(t^*)\]
		Definimos $\forall t \in \R:y(t)=x(t^*)$. Por hipótesis, $f(C)=f(x(t^*))=x'(t)=0$
		\[\implies y'(t)=\odv{}{t}(C)=0=f(C)=f(y(t))\implies y \tex{ es solución}\]
		\[\begin{cases}
			y'(t)=f(y(t)) \\
			y(t^*)=C
		\end{cases} \implies \tex{Por unicidad, } x(t)=y(t)\equiv C\]
	\end{dem}
	\item $\ds \lim_{t\to \infty} x(t) = C_0 \implies u(t) \equiv C$ es solución.
	\begin{dem}
		\begin{enumerate}
			\item $\ds \lim_{t\to \infty} x'(t)=\lim_{t\to \infty}f(x(t)) = f\left(\lim_{t\to \infty} x(t)\right) = f(C_0)$
			\item Veamos que $f(C_0)=0$. Por contradicción, supongamos que $f(C_0)=A>0$
			\[\implies \lim_{t\to \infty} x'(t)=A \implies \exists \, \til{t} : \forall t \geq \til{t} : x'(t)>\frac{A}{2}\]
			\[\implies \int_{\til{t}}^{t} x'(\tau) \odif{\tau} > \int_{\til{t}}^{t} \frac{A}{2} \odif{\tau} \implies x(t)-x\left(\til{t}\right)> \frac{A}{2}(t-\til{t})\]
			\[\implies \lim_{t\to\infty} x(t) = \lim_{t\to\infty}\left(x\left(\til{t}+\frac{A}{2}\left(t-\til{t}\right)\right)\right)=\infty \contr\]
			\item $\ds u'(t) = \odv{}{t}(C_0)=0=f(C_0)=f(u(t)) \implies u \tex{ es solución}$.
		\end{enumerate}
	\end{dem}
\end{enumerate}

\textbf{1.2} $x'=f(x)$ La unicidad solo se puede perder cuando $f(x)=0$
\begin{enumerate}
	\item $\ds f(x)\defeq x'=\begin{cases}
		\sqrt{-x} & x<0 \\
		x^2 & x\geq 0
	\end{cases}$
	\begin{obs}\begin{enumerate}
		\item[]
		\item $x\equiv 0$ es solución $\left(f(0)=0\right)$ y solo puede haber problemas de unicidad en $x=0$.
		\item $x(t)$ es estrictamente creciente si $x(t)\ne 0$
	\end{enumerate}\end{obs}
	No habría unicidad en $\ds x=0 \iff \lim_{x\to 0} \int_{0}^{x_0} \frac{1}{f(\tau)} \odif{\tau} \in \R$ con $x_0>x$ \\
	En nuestro caso, $\ds \int_{x}^{x_0} \frac{1}{\tau^2} \odif{\tau} = \frac{1}{x} - \frac{1}{x_0} \xrightarrow{x\to 0^+} \infty \implies \tex{ hay unicidad por arriba.}$ \\
	Para la unicidad por abajo, $\ds \int_{x_0}^{x} \frac{1}{\sqrt{-\tau}} \odif{\tau} = -2\left(\sqrt{-x} - \sqrt{-x_0}\right) \xrightarrow{x\to 0^-} 2 \sqrt{x_0} \in \R$
	\[\implies \tex{No hay unicidad por abajo.}\]
	Por tanto, podemos encontrar una solución de la siguiente forma:
	\[y(t)\defeq \begin{cases}
		-\frac{t^2}{4} & t<0 \\
		0 & t\geq 0
	\end{cases} \implies y'(t)=\begin{cases}
		-\frac{t}{2} & t<0 \\
		0 & t\geq 0
	\end{cases}\]
	Es solución porque:
	\[f(y(t))=\begin{cases}
		f(-\frac{t^2}{4}) & t<0 \\
		f(0) & t\geq 0
	\end{cases} = \begin{cases}
		\sqrt{-\left(-\frac{t^2}{4}\right)} & t<0 \\
		0 & t\geq 0
	\end{cases} = \begin{cases}
		-\frac{t}{2} & t<0 \\
		0 & t\geq 0
	\end{cases} = y'(t)\]
	Por un lado, $\ds x(t_0)=0 \implies \lim_{t\to \infty}x(t)=0 \tex{ por la unicidad por arriba.}$ \\
	Por otro lado, si $x(t_0)<0$ sabemos que
	\begin{itemize}
		\item $x(t)$ no decrece.
		\item $x(t)$ está acotada por arriba por 0.
	\end{itemize}
	\[\implies \exists \lim_{t\to \infty} x(t) \leq 0\]
	Supongamos que $\ds \exists A > 0 : \lim_{x(t)} = -A$. \\
	Como $x(t)$ no decrece,
	\[\ds \implies x(t) \leq -A \implies x'(t) = \sqrt{-x(t)}\geq \sqrt{A} \implies x(t)\geq x(c) + \sqrt{A}t \xrightarrow{t\to \infty} \infty \contr\]
	\[\implies \lim_{t\to \infty} x(t) =0 \]
	\item
\end{enumerate}

\textbf{1.3} $\ds x' = f(t,x) \we x$ es solución.
\[f\tex{ no depende de } t \iff \forall b \in \R : y(t)\defeq x(t+b) \tex{ es sol.}\]
\begin{dem}
	$(\implies)$ Supongamos que $f$ no depende de $t$.
	\[\implies x'=f(x) \implies y(t)=x(t+b)\implies y'(t)=\odv{}{t} (x(t+b)) = x'(t+b)\]
	\[\implies y'(t)=f(x(t+b))=f(y(t))\]
	$(\impliedby)$ Supongamos que $\forall b \in \R : y(t)\defeq x(t+b) \tex{ es sol.}$
	\[x'=f(t, x(t)) \implies y(t)=x(t+b) \tex{ también es solución}\]
	\[x'(t+b)=f(t, x(t+b)) \implies x'(t) = f(t-b, x(t))\]
	\[\implies \forall b \in \R : f(t-b, x(t))=f(t, x(t))\]
	En particular, $x(0)=x_0 \in \R \implies \forall b \in \R : \forall x_0 \in \R : f(-b,x_0)=f(0,x_0)$
	\[\implies f\tex{ no depende de su primera variable}\]
\end{dem}